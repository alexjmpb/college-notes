\documentclass{report}
\usepackage[spanish]{babel}

\usepackage[most,many,breakable]{tcolorbox}
\usepackage{xcolor}

\definecolor{defBoxBorder}{HTML}{395144}
\newtcolorbox{defBox}{colback=white,colframe=defBoxBorder,arc=3pt, boxrule=0.5pt, drop fuzzy shadow, title=Definition}
\definecolor{thBoxBorder}{HTML}{AC8441}
\newtcolorbox{thBox}{colback=white,colframe=thBoxBorder,arc=3pt, boxrule=0.5pt, drop fuzzy shadow, title=Theorem}
\definecolor{noteBoxBorder}{HTML}{4E6C50}
\newtcolorbox{noteBox}{colback=white,colframe=noteBoxBorder,arc=3pt, boxrule=0.5pt, drop fuzzy shadow, title=Note}
\definecolor{axBoxBorder}{HTML}{AA5656}
\newtcolorbox{axBox}{colback=white,colframe=axBoxBorder,arc=3pt, boxrule=0.5pt, drop fuzzy shadow, title=Axiom/Postulate}
\definecolor{corBoxBorder}{HTML}{8B7E74}
\newtcolorbox{corBox}{colback=white,colframe=corBoxBorder,arc=3pt, boxrule=0.5pt, drop fuzzy shadow, title=Corollary}
\definecolor{lemBoxBorder}{HTML}{B99B6B}
\newtcolorbox{lemBox}{colback=white,colframe=lemBoxBorder,arc=3pt, boxrule=0.5pt, drop fuzzy shadow, title=Lemma}
\definecolor{asBoxColor}{HTML}{FDFDF9}
\definecolor{asBoxBorder}{HTML}{DEB881}
\newtcolorbox{asBox}{coltext=black, colback=asBoxColor,colframe=asBoxBorder,arc=3pt, boxrule=0.5pt, drop fuzzy shadow, title=Aside}


\input{setup.tex}

\begin{document}
    \coverPage{Matemáticas}{Teoría de Números}{Fundamentos}{Ejemplo de Subtítulo.}{Alexander Mendoza}{\today}
    \tableofcontents

    \pagebreak
    \chapter{Fundamentos}

    \section{Definición de los números naturales y sus operaciones}
    Comencemos definiendo nuestro banco de trabajo, los números naturales. Definiremos los números naturales de la manera más estándar, utilizando los axiomas de Peano.

    \subsection*{Axiomas de Peano}
    Existe el conjunto de números naturales $\mathbb{N}$
    % Enumeración correcta con axiomas enumerate
    \begin{enumerate}
        \item $0 \in \mathbb{N}$.
        \item Si $x$ es un número natural, entonces $s(x)\in \mathbb{N}$.
        \item No existe $x \in \mathbb{N}$ tal que $s(x) = 0$.
        \item Sean $x,y \in \mathbb{N}$, si $s(x) = s(y)$, entonces $x=y$.
        \item Sea $B \subseteq \mathbb{N}$ tal que:
            \begin{itemize}
                \item $0 \in B$
                \item Si $n\in B$, entonces $s(n)=B$
            \end{itemize}
        \begin{noteBox}
            % MEJORAR LA GRAMÁTICA
            Se dice que $B$ es un conjunto inductivo. Los Axiomas originales establecían que $B$ podía ser cualquier conjunto en lugar de un subconjunto de $\mathbb{N}$, y luego, mediante un teorema, se demostraría que $\mathbb{N}$ sería el más pequeño de todos los conjuntos inductivos. Para evitar todo este trabajo,  establecemos que $B$ es un subconjunto de $\mathbb{N}$ y a partir de esto podemos concluir que $B = \mathbb{N}$
        \end{noteBox}
    \end{enumerate}

    El último axioma se conoce como el Principio de Inducción Matemática (PIM), profundizaremos más en este tema ya que se utilizará en todas partes a partir de ahora.

    \subsection*{Adición}

    La adición es una función recursiva $+$ con las siguientes propiedades:

    $$+: \mathbb{N}\times \mathbb{N} \to \mathbb{N}$$ donde dados $m,n \in \mathbb{N}$:

    $$+(m,0) = m$$
    $$+(m,s(n)) = s(+(m,n))$$

    La salida de la función se llama la suma de $m$ y $n$ y se denota como $m+n$ por conveniencia, con esta notación nuestra definición puede ser vista como:

    $$m+0 = m$$
    $$m + s(n) = s(m, n)$$

    Damos un par de ejemplos para verificar que esta función es de hecho la suma con la que estamos familiarizados.

    \begin{Example}
        Sea $m=2$ y $n=3$, entonces por definición

        \begin{align*}
            2+3 &= s(2+2)\\
            &= s(s(2+1))\\
            &= s(s(s(2+0)))\\
            &= s(s(s(2)))\\
            &= s(s(3))\\
            &= s(4)\\
            &= 5
        \end{align*}
    \end{Example}
\end{document}