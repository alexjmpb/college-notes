\documentclass{report}
\usepackage[english]{babel}

\usepackage[most,many,breakable]{tcolorbox}
\usepackage{xcolor}

\definecolor{defBoxBorder}{HTML}{395144}
\newtcolorbox{defBox}{colback=white,colframe=defBoxBorder,arc=3pt, boxrule=0.5pt, drop fuzzy shadow, title=Definition}
\definecolor{thBoxBorder}{HTML}{AC8441}
\newtcolorbox{thBox}{colback=white,colframe=thBoxBorder,arc=3pt, boxrule=0.5pt, drop fuzzy shadow, title=Theorem}
\definecolor{noteBoxBorder}{HTML}{4E6C50}
\newtcolorbox{noteBox}{colback=white,colframe=noteBoxBorder,arc=3pt, boxrule=0.5pt, drop fuzzy shadow, title=Note}
\definecolor{axBoxBorder}{HTML}{AA5656}
\newtcolorbox{axBox}{colback=white,colframe=axBoxBorder,arc=3pt, boxrule=0.5pt, drop fuzzy shadow, title=Axiom/Postulate}
\definecolor{corBoxBorder}{HTML}{8B7E74}
\newtcolorbox{corBox}{colback=white,colframe=corBoxBorder,arc=3pt, boxrule=0.5pt, drop fuzzy shadow, title=Corollary}
\definecolor{lemBoxBorder}{HTML}{B99B6B}
\newtcolorbox{lemBox}{colback=white,colframe=lemBoxBorder,arc=3pt, boxrule=0.5pt, drop fuzzy shadow, title=Lemma}
\definecolor{asBoxColor}{HTML}{FDFDF9}
\definecolor{asBoxBorder}{HTML}{DEB881}
\newtcolorbox{asBox}{coltext=black, colback=asBoxColor,colframe=asBoxBorder,arc=3pt, boxrule=0.5pt, drop fuzzy shadow, title=Aside}


\input{setup.tex}

\begin{document}
    \coverPage{ Mathematics }{ Linear Algebra I }{ Vector Spaces }{  }{ Alexander Mendoza }{\today}
    \tableofcontents

    \pagebreak
    \chapter{ Vector Spaces }

    \section{Vector Spaces}

    We now are ready to start looking at a direction that is closer to linear algebra by reviewing vector spaces. We define a vector space is a composite object consisting of a field, a set of vectors and two operations with certain special properties. The formal definition of a vector space goes as follows.

    \begin{defBox}
        \textit{\textbf{Vector Space.}} A vector space consists of:

        \begin{enumerate}
            \item A field $F$ of scalars;
            \item A set $V$ of objects, called vectors;
            \item A binary operation $\oplus : V \times V \rightarrow V$ called vector addition, which associates with each pair of vectors $\vec{v} , \vec{u} \in V$ to a vector $\vec{v} + \vec{u} \in V$, called the sum of $\vec{v}$ and $\vec{u}$, in such a way that $V$ together with $\oplus$, $(V, \oplus)$ form an abelian group:

            \begin{enumerate}
                \item Addition is associative, $\vec{v} + (\vec{u} + \vec{w}) = (\vec{v} + \vec{u}) + \vec{w}$;
                \item There is a unique vector $0$ in $V$, called the zero vector, such that $\vec{v} + 0 = \vec{v}$ for all $\vec{v}$ in $V$.
                \item For each vector $\vec{v}$ in $V$ there is a unique vector $-a$ in $V$ such that $\vec{v} + (-\vec{v}) = 0$.
                \item Addition is commutative, $\vec{v} + \vec{u} = \vec{u} + \vec{v}$;
            \end{enumerate}

            \item A function $\odot : F \times V \rightarrow V$ called scalar multiplication, which associates with each scalar $\alpha$ in $F$ and vector $\vec{v} \in V$ to a vector $\alpha\vec{v}$ in $V$, called the product of $\alpha$ and $\vec{v}$ in such a way that:

            \begin{enumerate}
                \item $1\vec{v} = \vec{v}$ for every $\vec{v}$ in $V$;
                \item $(\alpha, \beta)\vec{v} = \alpha(\beta\vec{v})$;
                \item $c(\vec{v} + \vec{u}) = \alpha\vec{v} + c\vec{u}$;
                \item $(\alpha + \beta)\vec{v} = \alpha\vec{v} + \beta\vec{v}$;
            \end{enumerate}
        \end{enumerate}

        We then say that $(V, \oplus, \odot)$ is a vector space over the field $(F, +, \cdot)$.
    \end{defBox}

    Notice how the scalar multiplication behaves like an operation, but is not an operation, this is because it includes elements from the field whichi is external to $V$. Also some characteristics of the vector addition is that the addition is performed component by component. Take for example the abelian group $\mathbb{R}^2$ with the vector addition $\oplus$, the latter can be defined as $(a, b) \oplus (c, d) = (a+c, b+d)$ for all $a,b,c,d \in \mathbb{R}$.
\end{document}
