\documentclass{report}
\usepackage[spanish]{babel}

\usepackage[most,many,breakable]{tcolorbox}
\usepackage{xcolor}

\definecolor{defBoxBorder}{HTML}{395144}
\newtcolorbox{defBox}{colback=white,colframe=defBoxBorder,arc=3pt, boxrule=0.5pt, drop fuzzy shadow, title=Definition}
\definecolor{thBoxBorder}{HTML}{AC8441}
\newtcolorbox{thBox}{colback=white,colframe=thBoxBorder,arc=3pt, boxrule=0.5pt, drop fuzzy shadow, title=Theorem}
\definecolor{noteBoxBorder}{HTML}{4E6C50}
\newtcolorbox{noteBox}{colback=white,colframe=noteBoxBorder,arc=3pt, boxrule=0.5pt, drop fuzzy shadow, title=Note}
\definecolor{axBoxBorder}{HTML}{AA5656}
\newtcolorbox{axBox}{colback=white,colframe=axBoxBorder,arc=3pt, boxrule=0.5pt, drop fuzzy shadow, title=Axiom/Postulate}
\definecolor{corBoxBorder}{HTML}{8B7E74}
\newtcolorbox{corBox}{colback=white,colframe=corBoxBorder,arc=3pt, boxrule=0.5pt, drop fuzzy shadow, title=Corollary}
\definecolor{lemBoxBorder}{HTML}{B99B6B}
\newtcolorbox{lemBox}{colback=white,colframe=lemBoxBorder,arc=3pt, boxrule=0.5pt, drop fuzzy shadow, title=Lemma}
\definecolor{asBoxColor}{HTML}{FDFDF9}
\definecolor{asBoxBorder}{HTML}{DEB881}
\newtcolorbox{asBox}{coltext=black, colback=asBoxColor,colframe=asBoxBorder,arc=3pt, boxrule=0.5pt, drop fuzzy shadow, title=Aside}


\input{setup.tex}

\begin{document}
    \coverPage{ Matemáticas }{ Álgebra Lineal II }{ Diagonalización }{  }{ Alexander Mendoza }{\today}
    \tableofcontents
    \pagebreak
    \chapter{ Diagonalización }
    
    \section{Algunas cosas para recordar}
    
    \begin{defBox}
        Sean $\mathcal{T}$ el conjunto de todas las transformaciones lineales de $V$ a $W$, y $L(V,W) = \{T \mid T \in \mathcal{T}\}$, entonces $L(V,W)$ es un espacio vectorial donde sus operaciones se definen como:
        
        \begin{itemize}
            \item \textit{\textbf{Suma de vectores}}. Sean $T_1, T_2 \in L(V,W)$ entonces para cada $v_1 \in V$
            $$(T_1+T_2)(v_1) = T_1(v_1) + T_2(v_1)$$
            \item \textit{\textbf{Multiplicación por un escalar}}. Sea $T \in L(V,W)$ y sea $\alpha \in F$, entonces para todo $v_1 \in V$
            $$(\alpha T)(v_1) = \alpha T(v_1)$$
        \end{itemize}
    \end{defBox}
    
    \begin{defBox}
        \textit{\textbf{Multiplicación de matrices por la izquierda}}. Sea $A$ una matriz de $m \times n$ con entradas de un campo $F$. Denotamos por $\mathrm{L}_A$ la función $\mathrm{L}_A: \mathrm{F}^n \rightarrow \mathrm{F}^m$ definida por $\mathrm{L}_A(x)=A x$ (el producto matriz de $A$ y $x$) para cada vector columna $x \in \mathrm{F}^n$. Llamamos $\mathrm{L}_A$ una transformación de multiplicación por la izquierda.
    \end{defBox}
    
    \begin{defBox}
        \textit{\textbf{Matriz diagonal}}. Una matriz se llama diagonal si todas sus entradas $A_{ij}$ son cero excepto cuando $i = j$.
    \end{defBox}
    
    \section{Valores propios y vectores propios}
    
    \begin{defBox}
        \textit{\textbf{Vector propio y valores propios}}. Definiciones. Sea $\mathrm{T}$ un operador lineal en un espacio vectorial V. Un vector no nulo $v \in \mathrm{V}$ se llama un vector propio de $\mathrm{T}$ si existe un escalar $\lambda$ tal que $\mathrm{T}(v)=\lambda v$. El escalar $\lambda$ se llama un valor propio correspondiente al vector propio $v$.\\
        
        Sea $A$ en $\mathrm{M}_{n \times n}(F)$. Un vector de longitud no nulo $v \in \mathrm{F}^n$ se llama un vector propio de $A$ si $v$ es un vector propio de $\mathrm{L}_A$; es decir, si $A v=\lambda v$ para algún escalar $\lambda$. El escalar $\lambda$ se llama un valor propio de $A$ correspondiente al vector propio $v$.
    \end{defBox}
    
    \begin{thBox}
        Sea $T$ un operador lineal sobre un espacio vectorial de dimension finita $V$. Y sea $\lambda$ un escalar, entonces las siguientes afirmaciones son equivalentes
    
        \begin{enumerate}
            \item $\lambda$ es un valor propio de $T$
            \item El operador $(T - \lambda I)$ es singular (no tiene inversa)
            \item $\det(T-\lambda I) = 0$
        \end{enumerate}
    \end{thBox}
    
    \begin{defBox}
        Si $A$ es una matriz sobre un campo $F$, un valor propio de $A$ en $F$ es un escalar en $F$, tal que $(A - \lambda I)$ es singular.
    \end{defBox}
    
    \begin{defBox}
        Sea $A$ una matriz cuadrada. La función $P_A: \mathbb{R} \to \mathbb{R}$ donde $P_A(\lambda) = \det(A - \lambda I)$ se llama el polinomio característico de $A$.
    \end{defBox}
    
    Con estos teoremas y definiciones, podemos obtener los valores propios de una matriz encontrando las raíces del polinomio característico y obtener el vector propio $v$ definido por el valor propio $\lambda$ resolviendo el sistema $(A-\lambda I)v = 0$
    
    \begin{Example}
        Sea
        $$A = \begin{bmatrix}
            -1 & -3 & 9\\
            0 & 5 & 18\\
            0 & -2 & -7
        \end{bmatrix}$$
        Para encontrar los valores propios, vectores propios y el polinomio característico, comenzamos con
        $$f(\lambda) = \det\left(A - \lambda I\right)$$
        Luego, \begin{align*}
            \det\left(A - \lambda I\right) &= \det\left(
                \begin{bmatrix}
                    -1 & -3 & 9\\
                    0 & 5 & 18\\
                    0 & -2 & -7
                \end{bmatrix}
            - \lambda
                \begin{bmatrix}
                    1 & 0 & 0\\
                    0 & 1 & 0\\
                    0 & 0 & 1
                \end{bmatrix}
            \right)\\\\
            &= \det\left(
                \begin{bmatrix}
                    -1-\lambda & -3 & 9\\
                    0 & 5-\lambda & 18\\
                    0 & -2 & -7-\lambda
                \end{bmatrix}
            \right)\\\\
            &= -\lambda^3 -3\lambda^2 -3\lambda -1
        \end{align*}
        Ahora, para encontrar los valores propios, necesitamos encontrar las raíces del polinomio. Se puede observar que $-\lambda^3 -3\lambda^2 -3\lambda -1$ es de la forma expandida de un binomio al cubo, por lo tanto, el polinomio puede ser factorizado de la siguiente manera:
        $$-(\lambda + 1)^3$$
        Así, al establecer $-(\lambda + 1)^3 = 0$, podemos concluir que $\lambda_1 = -1, \lambda_2 = -1, \lambda_3 = -1$. Ahora, vamos a encontrar los vectores propios, para esto, resolveremos el siguiente sistema homogéneo: $Ax = 0$, donde $x = (x_1, x_2, x_3)$ y $x_1, x_2, x_3 \in F$. Entonces, el sistema homogéneo para $\lambda = -1$ sería el siguiente:
        \begin{align*}
            \begin{bmatrix}
                0 & -3 & 9\\
                0 & 6 & 18\\
                0 & -2 & -6
            \end{bmatrix}
            \begin{bmatrix}
                x_1\\ x_2\\ x_3
            \end{bmatrix} = \begin{bmatrix}
                0\\0\\0
            \end{bmatrix}
        \end{align*}
        Al resolver el sistema, encontramos que el vector
        $$\begin{bmatrix}
            1\\0\\0
        \end{bmatrix}$$
        Es una solución para el sistema y, por lo tanto, es un vector propio para $A$.
    \end{Example}
    
    \begin{defBox}
        Sea $\mathrm{T}$ un operador lineal sobre un espacio vectorial $\mathrm{V}$, y sea $\lambda$ un valor propio de $\mathrm{T}$. Definimos $\mathrm{E}_\lambda=\{x \in \mathrm{V}: \mathrm{T}(x)=\lambda x\}=\mathrm{N}\left(\mathrm{T}-\lambda \mathrm{I}_{\mathrm{V}}\right)$. Al conjunto $\mathrm{E}_\lambda$ se le llama espacio propio $\mathrm{T}$ correspondiente al valor propio $\lambda$. Análogamente, definimos el espacio propio de una matriz cuadrada $A$ correspondiente al valor propio $\lambda$ como el espacio propio de $\mathrm{L}_A$ correspondiente a $\lambda$.
    \end{defBox}
    
    Continuando con el ejemplo anterior, su espacio propio $E_\lambda$ sería
    
    $$ E_\lambda = \left\{ \begin{bmatrix}1\\0\\0\end{bmatrix} \right\}$$
    
    \section{Trabajando con el polinomio característico}
    Vale la pena recordar algunas herramientas para encontrar los determinantes de una matriz y las raíces de un polinomio. Recordemos primordialmente la definición de determinante.
    
    Primero estableceremos un teorema que nos ayudará a encontrar algunas partes del polinomio característico.
    
    \begin{thBox}
        Denotemos los $n$ valores propios de una matriz $A \in M_n(c)$, por $\lambda_1, \lambda_2, \dots , \lambda_n$ y su polinomio característico por
        
        $$P_a(\lambda) = (-1)^n\lambda^n + C_{n-1}\lambda^{n-1} + \cdots + C_1\lambda + C_0$$
        
        Entonces $$C_0 = \det(A) = \lambda_1 \lambda_2 \cdots \lambda_n$$ y $$(-1)^{n-1}C_{n-1} = tr(A) = \lambda_1 + \lambda_2 + \cdots + \lambda_n$$
    \end{thBox}
    
    \begin{thBox}
        \textit{\textbf{Teorema de las raíces}}. Todo polinomio de grado $n \geq 1$ tiene exactamente $n$ raíces, siempre y cuando una raíz de multiplicidad $k$ se cuente $k$ veces.
    \end{thBox}
    
    \begin{thBox}
        \textit{\textbf{Teorema de las raíces racionales}}. Si el polinomio $P(x)=a_n x^n+a_{n-1} x^{n-1}+\cdots+a_1 x+a_0$ tiene coeficientes enteros, entonces todos los ceros racionales de $P$ tienen la forma
        
        $$\frac{p}{q}$$
        
        donde $\quad p$ es un factor del coeficiente constante $a_0$
        y $\quad q$ es un factor del coeficiente principal $a_n$.
    \end{thBox}
    
    \begin{noteBox}
        \textit{\textbf{Encontrando los ceros racionales de un polinomio}}.
        \begin{enumerate}
            \item Listar los ceros posibles. Enumera todos los ceros racionales posibles utilizando el teorema de los ceros racionales.
            \item Dividir. Usa la división sintética para evaluar el polinomio en cada uno de los candidatos a ceros racionales que se encontraron en el paso 1. Cuando el residuo es $0$, anota el cociente que has obtenido.
            \item Repetir. Repite los pasos 1 y 2 para el cociente. Detente cuando alcances un cociente que es cuadrático o se puede factorizar fácilmente, y usa la fórmula cuadrática o el factor para encontrar los ceros restantes.
        \end{enumerate}
    \end{noteBox}
    
    \section{Diagonalización de una matriz}
    
    \subsection*{Multiplicidad}
    
    \begin{defBox}
        \textit{\textbf{Multiplicidad geométrica de un valor propio}}. Sea $A$ una matriz cuadrada con valor propio $\lambda$. La multiplicidad geométrica de $\lambda$ es la dimensión de $E_\lambda$ (el espacio generado por $\lambda$).
    \end{defBox}
    
    Por ejemplo 1.2.5, la multiplicidad geométrica de $\lambda = -1$ es $1$ ya que
    $$E_\lambda = \left\{ \begin{bmatrix}1\\0\\0\end{bmatrix} \right\}$$
    
    \begin{defBox}
        Deje que A sea una matriz cuadrada con valor propio $\lambda$. La multiplicidad algebraica de $\lambda$ es el entero positivo más grande $k$ para el cual $(t-\lambda)^k$ es un factor de $f(t)$.
    \end{defBox}

    \begin{Example}
        Sea

        $$A =\begin{pmatrix}
            3 & 1 & 0\\
            0 & 3 & 4\\
            0 & 0 & 4
        \end{pmatrix}$$

        que tiene polinomio característico $f(t) = -(t-3)^2(t-4)$. Por lo tanto $\lambda = 3$ es un valor propio de $A$ con multiplicidad de 2, y $\lambda = 4$ es un valor propio de $A$ con una multiplicidad de 1.
    \end{Example}
    
    \begin{thBox}
        Deje que A sea una matriz cuadrada con valor propio $\lambda$. La multiplicidad algebraica de $\lambda$ es el entero positivo más grande $k$ para el cual $(t-\lambda)^k$ es un factor de $f(t)$.
    \end{thBox}
    
    \begin{thBox}
        La multiplicidad de una eigenvalor no excede a su multiplicidad algebraica.
    \end{thBox}
    
    \subsection*{Matrices similares}
    
    \begin{defBox}
        \textit{\textbf{Matrices similares}}. Dos matrices cuadradas $A$ y $B$ se dice que son similares si existe una matriz no singular $P$ tal que

        $$B = P^{-1}AP$$
    \end{defBox}
    
    \begin{thBox}
        Las matrices similares tienen los mismos valores propios contados con multiplicidad.
    \end{thBox}
    
    \subsection*{Diagonalización de una matriz}
    
    \begin{defBox}
        \textit{\textbf{Diagonalización de una matriz}}. Un operador lineal $T$ en un espacio vectorial de dimensional finita $V$ se llama diagonalizable si existe una base ordenada $B$ para $V$ tal que $[\mathrm{T}]_\beta$ es una matriz diagonal.

        Una matriz cuadrada $A$ se llama diagonalizable si $\mathrm{L}_A$ es diagonalizable. En otras palabras decimos que una matriz cuadrada $A$ es diagonalizable si y solo si existe una matriz invertible $P$ tal que

        $$D = P^{-1}AP$$ o
        $$A = PDP^{-1}$$
    \end{defBox}
    
    \begin{thBox}
        Si $A$ es similar a una matriz diagonal $D$, entonces los elementos de $D$ son los valores propios de $A$; y los valores son independientes de la forma en la cual se representa una transformación lineal.
    \end{thBox}
    
    \begin{thBox}
        Una matriz $nxn$ es diagonalizable si y solo si tiene vectores propios linealmente independientes.
    \end{thBox}
    
    \begin{thBox}
        Lo siguiente son equivalentes

        \begin{enumerate}
            \item $T: V \to V$ es diagonalizable
            \item Para cada valor propio $\lambda$ de $T$, la multiplicidad geométrica de $\lambda$ coincide con la multiplicidad algebraica.
        \end{enumerate}
    \end{thBox}
    
    \subsection*{Cambio de base}
    
    \begin{defBox}
        \textit{\textbf{Base ordenada}}. Sea $V$ un espacio vectorial de dimensional finita. Una base ordenada para tal espacio está provista de un orden específico, es decir, una base ordenada para $V$ es una secuencia de vectores linealmente independientes en $V$ que generan $V$.
    \end{defBox}
    
    \begin{Example}
        Sea $V = F^3$.

        $$\beta = \left\{ e_1, e_2, e_3 \right\}$$
        $$\alpha = \left\{ e_2, e_1, e_3 \right\}$$

        Entonces $\beta \not = \alpha$, y $\beta, \alpha$ son bases ordenadas.
    \end{Example}
    
    \begin{defBox}
        Sea $\beta = \left\{ u_1, u_2, \dots , u_n \right\}$ una base ordenada para $V$. Para cada $v \in V$, dejamos que $\alpha_1, \alpha_2, \dots , \alpha_n$ sean los escalares únicos tales que

        $$v = \sum_{i=1}^{n}\alpha_iu_i$$

        Definimos el vector de coordenadas de $v$ relativo a $\beta$, denotado por

        $$[v]_\beta = \begin{pmatrix}
            \alpha_1\\ \alpha_2 \\ \vdots \\ \alpha_n
        \end{pmatrix}$$
    \end{defBox}
    
    \begin{Example}
        Sea $P_2(\mathbb{R})$ y sea $\beta = \left\{ 1, x, x^2 \right\}$ y $\alpha = \left\{ x^2, x, 1 \right\}$ si $v = f(x) = 4 + 6x -7x^2$.

        $$[v]_\beta = \begin{pmatrix}
            4\\6\\-7
        \end{pmatrix}$$

        $$[v]_\alpha = \begin{pmatrix}
            -7\\6\\4
        \end{pmatrix}$$
    \end{Example}
    
    \begin{thBox}
        Sean $f_\beta: V \to \mathbb{R}^n$ y $g_\beta:\mathbb{R}^n \to V$ las funciones siguientes:
        
        $$f_\beta(v) = [v]_beta$$
        
        $$g_\beta([v]_\beta) = v$$
        
        Entonces $f_\beta$ y $g_\beta$ son linealments y $g_\beta \circ f_\beta$ es la identidad en $V$ y $f_\beta \circ g_\beta$ es la identidad en $\mathbb{R}^n$
    \end{thBox}
    
    \begin{Example}
        Sean las siguientes fórmulas
        
        $$f(\alpha) = A\alpha$$
        
        $$P = [I]_\beta^\gamma$$
        
        Donde
        
        $$A = \begin{bmatrix}
            3 & 1 & 2\\
            0 & 2 & 0\\
            0 & 0 & 1
        \end{bmatrix}$$
        
        $$\alpha = \begin{bmatrix}
            2\\1\\3
        \end{bmatrix}$$
        
        $$[I]_\beta^\gamma = \begin{bmatrix}
            1 & -2 & -1\\
            0 & 3 & -1
        \end{bmatrix}$$
        
        Entonces si calculamos $f(\alpha)$ obtenemos
        
        $$f(\alpha) = A\alpha = \begin{bmatrix}
            3 & 1 & 2\\
            0 & 2 & 0\\
            0 & 0 & 1
        \end{bmatrix}\begin{bmatrix}
            2\\1\\3
        \end{bmatrix} = \begin{bmatrix}
            12\\2\\3
        \end{bmatrix}$$
        
        y si calculamos $Pf(\alpha)$ tenemos
        
        $$Pf(\alpha) = [I]_\beta^\gamma f(\alpha) = \begin{bmatrix}
            1 & -2 & -1\\
            0 & 3 & -1
        \end{bmatrix}\begin{bmatrix}
            12\\2\\3
        \end{bmatrix} = \begin{bmatrix}
            -10\\ 6
        \end{bmatrix}$$
    \end{Example}
    
    \section{Suma de subespacios}
    
    \begin{defBox}
        \textit{\textbf{Suma de conjuntos}}. Supongamos que $U_1, \dots U_m$ son subconjuntos de un espacio vectorial $V$. Entonces la suma de $U_1, \dots , U_m$ denotada por $U_1 + \cdots + U_m $ es el conjunto de todas las sumas posibles de los elementos de $U_1, \dots , U_m$, más precisamente
        
        $$U_1 + \cdots U_m = \left\{ u_1 + \cdots u_m \mid u_1 \in U_1 , \dots , u_m \in U_m \right\}$$
    \end{defBox}
    
    \begin{Example}
        $$U = \left\{ (x, 0, 0) \in F^3 \mid x \in F \right\}$$
        $$W = \left\{ (0, y, 0) \in F^3 \mid x \in F \right\}$$

        Entonces

        $$U + W = \left\{ (x, y, 0) \in F^3 \mid x, y \in F \right\}$$
    \end{Example}

    \begin{Example}
        $$W_1 = \left\{ m(1,0) \mid m \in \mathbb{R} \right\}$$
        $$W_2 = \left\{ l(0,1) \mid l \in \mathbb{R} \right\}$$

        Entonces

        $$W_1 + W_2 = \left\{ m(1,0) + l(1,0) \mid m, l \in \mathbb{R} \right\}$$
    \end{Example}
    
    \begin{thBox}
        \textit{\textbf{La suma de subespacios también es un subespacio}}. Sea $V_1, \dots , V_n$ subespacios de un espacio vectorial $V$, entonces
        
        $$ V_1 + \cdots V_n $$
    \end{thBox}
    
    % \begin{defBox}
    %     \textit{\textbf{Suma directa}}. Supongamos que $U_1, \dots , U_n$ son subespacios de $V$.

    %     \begin{enumerate}
    %         \item La suma $U_1 + \cdots + U_n$ se llama suma directa si cada elemento de $U_1 + \cdots + U_n$ se puede escribir en una forma única como una suma $U_1 + \cdots + U_m$
    %     \end{enumerate}
    % \end{defBox}
\end{document}