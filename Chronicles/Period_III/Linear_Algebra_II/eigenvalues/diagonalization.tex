\documentclass{report}
\usepackage[english]{babel}

\usepackage[most,many,breakable]{tcolorbox}
\usepackage{xcolor}

\definecolor{defBoxBorder}{HTML}{395144}
\newtcolorbox{defBox}{colback=white,colframe=defBoxBorder,arc=3pt, boxrule=0.5pt, drop fuzzy shadow, title=Definition}
\definecolor{thBoxBorder}{HTML}{AC8441}
\newtcolorbox{thBox}{colback=white,colframe=thBoxBorder,arc=3pt, boxrule=0.5pt, drop fuzzy shadow, title=Theorem}
\definecolor{noteBoxBorder}{HTML}{4E6C50}
\newtcolorbox{noteBox}{colback=white,colframe=noteBoxBorder,arc=3pt, boxrule=0.5pt, drop fuzzy shadow, title=Note}
\definecolor{axBoxBorder}{HTML}{AA5656}
\newtcolorbox{axBox}{colback=white,colframe=axBoxBorder,arc=3pt, boxrule=0.5pt, drop fuzzy shadow, title=Axiom/Postulate}
\definecolor{corBoxBorder}{HTML}{8B7E74}
\newtcolorbox{corBox}{colback=white,colframe=corBoxBorder,arc=3pt, boxrule=0.5pt, drop fuzzy shadow, title=Corollary}
\definecolor{lemBoxBorder}{HTML}{B99B6B}
\newtcolorbox{lemBox}{colback=white,colframe=lemBoxBorder,arc=3pt, boxrule=0.5pt, drop fuzzy shadow, title=Lemma}
\definecolor{asBoxColor}{HTML}{FDFDF9}
\definecolor{asBoxBorder}{HTML}{DEB881}
\newtcolorbox{asBox}{coltext=black, colback=asBoxColor,colframe=asBoxBorder,arc=3pt, boxrule=0.5pt, drop fuzzy shadow, title=Aside}


\input{setup.tex}

\begin{document}
    \coverPage{ Mathematics }{ Linear Algebra II }{ Diagonalization }{  }{ Alexander Mendoza }{\today}
    \tableofcontents
    \pagebreak
    \chapter{ Diagonalization }

    \section{Some things to remember}

    \begin{defBox}
        Let $\mathcal{T}$ be the set of all linear transformations from $V$ into $W$, and let $L(V,W) = \{T \mid T \in \mathcal{T}\}$, then $L(V,W)$ is a vector space where its operations are defined as:

        \begin{itemize}
            \item \textit{\textbf{Vector addition}}. Let $T_1, T_2 \in L(V,W)$ then for every $v_1 \in V$
            $$(T_1+T_2)(v_1) = T_1(v_1) + T_2(v_1)$$
            \item \textit{\textbf{Scalar multiplication}}. Let $T \in L(V,W)$ and let $\alpha \in F$, then for all $v_1 \in V$
            $$(\alpha T)(v_1) = \alpha T(v_1)$$
        \end{itemize}
    \end{defBox}

    \begin{defBox}
        \textit{\textbf{Matrix left multiplication}}. Let $A$ be an $m \times n$ matrix with entries from a field $F$. We denote by $\mathrm{L}_A$ the mapping $\mathrm{L}_A: \mathrm{F}^n \rightarrow \mathrm{F}^m$ defined by $\mathrm{L}_A(x)=A x$ (the matrix product of $A$ and $x$ ) for each column vector $x \in \mathrm{F}^n$. We call $\mathrm{L}_A$ a left-multiplication transformation.
    \end{defBox}

    \begin{defBox}
        \textit{\textbf{Diagonal matrix}}. A matrix is called diagonal if all its entries $A_{ij}$ are zero except when $i = j$.
    \end{defBox}

    \section{Eigenvalues and Eigenvectors}

    \begin{defBox}
        \textit{\textbf{Eigenvector and Eigenvalues}}. Definitions. Let $\mathrm{T}$ be a linear operator on a vector space V. A nonzero vector $v \in \mathrm{V}$ is called an eigenvector of $\mathrm{T}$ if there exists a scalar $\lambda$ such that $\mathrm{T}(v)=\lambda v$. The scalar $\lambda$ is called the eigenvalue corresponding to the eigenvector $v$.\\

        Let $A$ be in $\mathrm{M}_{n \times n}(F)$. A nonzero vector $v \in \mathrm{F}^n$ is called an eigenvector of $A$ if $v$ is an eigenvector of $\mathrm{L}_A$; that is, if $A v=\lambda v$ for some scalar $\lambda$. The scalar $\lambda$ is called the eigenvalue of $A$ corresponding to the eigenvector $v$.
    \end{defBox}

    \begin{thBox}
        Let $T$ be a linear operator over a finite dimensional vector space $V$. And let $\lambda$ be a scalar, then the following are equivalent

        \begin{enumerate}
            \item $\lambda$ is an eigenvalue of $T$
            \item The operator $(T - \lambda I)$ is singular (it doesn't have inverse)
            \item $\det(T-\lambda I) = 0$
        \end{enumerate}
    \end{thBox}

    \begin{defBox}
        If $A$ is a matrix over a field $F$, an eigenvalue of $A$ in $F$ is a scalar in $F$, such that $(A - \lambda I)$ is singular.
    \end{defBox}

    \begin{defBox}
        Let $A$ be an squared matrix. The function $P_A: \mathbb{R} \to \mathbb{R}$ where $P_A(\lambda) = \det(A - \lambda I)$ is called the characteristic polynomial of $A$.
    \end{defBox}

    With these theorems and definitions, we can obtain eigenvalues of a matrix by finding the roots of the characteristic polynomial and obtain the eigenvector $v$ corresponding to an eigenvalue $\lambda$ by solving the system $(A-\lambda I)v = 0$

    \begin{Example}
        Let
        $$A = \begin{bmatrix}
            -1 & -3 & 9\\
            0 & 5 & 18\\
            0 & -2 & -7
        \end{bmatrix}$$
        To find the eigenvalues, eigenvectors, and the characteristic polynomial, we start with

        $$f(\lambda) = \det\left(A - \lambda I\right)$$

        Then, \begin{align*}
            \det\left(A - \lambda I\right) &= \det\left(
                \begin{bmatrix}
                    -1 & -3 & 9\\
                    0 & 5 & 18\\
                    0 & -2 & -7
                \end{bmatrix}
            - \lambda
                \begin{bmatrix}
                    1 & 0 & 0\\
                    0 & 1 & 0\\
                    0 & 0 & 1
                \end{bmatrix}
            \right)\\\\
            &= \det\left(
                \begin{bmatrix}
                    -1-\lambda & -3 & 9\\
                    0 & 5-\lambda & 18\\
                    0 & -2 & -7-\lambda
                \end{bmatrix}
            \right)\\\\
            &= -\lambda^3 -3\lambda^2 -3\lambda -1
        \end{align*}

        Now, to find the eigenvalues, we need to find the roots of the polynomial. It can be observed that $-\lambda^3 -3\lambda^2 -3\lambda -1$ is in the expanded form of a cube of a binomial, therefore, the polynomial can be factorized as follows:

        $$-(\lambda + 1)^3$$

        Thus, by setting $-(\lambda + 1)^3 = 0$, we can conclude that $\lambda_1 = -1, \lambda_2 = -1, \lambda_3 = -1$. Now let's find the eigenvectors, for this, we will solve the following homogeneous system: $Ax = 0$, where $x = (x_1, x_2, x_3)$ and $x_1, x_2, x_3 \in F$. So, the homogeneous system for $\lambda = -1$ would be as follows:
        \begin{align*}
            \begin{bmatrix}
                0 & -3 & 9\\
                0 & 6 & 18\\
                0 & -2 & -6
            \end{bmatrix}
            \begin{bmatrix}
                x_1\\ x_2\\ x_3
            \end{bmatrix} = \begin{bmatrix}
                0\\0\\0
            \end{bmatrix}
        \end{align*}

        Solving the system, we find that the vector

        $$\begin{bmatrix}
            1\\0\\0
        \end{bmatrix}$$

        is a solution to the system and therefore is an eigenvector for $A$.
    \end{Example}

    \begin{defBox}
        Let $\mathrm{T}$ be a linear operator on a vector space $\mathrm{V}$, and let $\lambda$ be an eigenvalue of $\mathrm{T}$. Define $\mathrm{E}_\lambda=\{x \in \mathrm{V}: \mathrm{T}(x)=\lambda x\}=\mathrm{N}\left(\mathrm{T}-\lambda \mathrm{I}_{\mathrm{V}}\right)$. The set $\mathrm{E}_\lambda$ is called the eigenspace of $\mathrm{T}$ corresponding to the eigenvalue $\lambda$. Analogously, we define the eigenspace of a square matrix $A$ corresponding to the eigenvalue $\lambda$ to be the eigenspace of $\mathrm{L}_A$ corresponding to $\lambda$.
    \end{defBox}

    Continuing with the previous example, its eigenspace $E_\lambda$ would be

    $$ E_\lambda = \left\{ \begin{bmatrix}1\\0\\0\end{bmatrix} \right\}$$

    \section{Working with the characteristic polynomial}
    It's worth to remember some tools to find the determinants of a matrix and the roots of a polynomial. Let's first remember the definition of determinant.

    We'll first state a theorem that will help us to find some parts of the characteristic polynomial.

    \begin{thBox}
        Denote the $n$ eigenvalues of a $A \in M_n(c)$, by $\lambda_1, \lambda_2, \dots , \lambda_n$ and its characteristic polynomial by

        $$P_a(\lambda) = (-1)^n\lambda^n + C_{n-1}\lambda^{n-1} + \cdots + C_1\lambda + C_0$$

        Then $$C_0 = \det(A) = \lambda_1 \lambda_2 \cdots \lambda_n$$ and $$(-1)^{n-1}C_{n-1} = tr(A) = \lambda_1 + \lambda_2 + \cdots + \lambda_n$$
    \end{thBox}

    \subsection*{Determinants}
    We extend the definition of the determinant to $n \times n$ matrices for $n \geq 3$. For this definition, it is convenient to introduce the following notation: Given $A \in \mathrm{M}_{n \times n}(F)$, for $n \geq 2$, denote the $(n-1) \times(n-1)$ matrix obtained from $A$ by deleting row $i$ and column $j$ by $\bar{A}_{i j}$. Thus for
    $$
    A=\left(\begin{array}{lll}
    1 & 2 & 3 \\
    4 & 5 & 6 \\
    7 & 8 & 9
    \end{array}\right) \in \mathrm{M}_{3 \times 3}(R),
    $$
    we have
    $$
    \tilde{A}_{11}=\left(\begin{array}{ll}
    5 & 6 \\
    8 & 9
    \end{array}\right), \quad \tilde{A}_{13}=\left(\begin{array}{ll}
    4 & 5 \\
    7 & 8
    \end{array}\right), \quad \text { and } \quad \tilde{A}_{32}=\left(\begin{array}{ll}
    1 & 3 \\
    4 & 6
    \end{array}\right),
    $$
    and for
    $$
    B=\left(\begin{array}{rrrr}
    1 & -1 & 2 & -1 \\
    -3 & 4 & 1 & -1 \\
    2 & -5 & -3 & 8 \\
    -2 & 6 & -4 & 1
    \end{array}\right) \in \mathrm{M}_{4 \times 4}(R),
    $$
    we have
    $$
    \tilde{B}_{23}=\left(\begin{array}{rrr}
    1 & -1 & -1 \\
    2 & -5 & 8 \\
    -2 & 6 & 1
    \end{array}\right) \quad \text { and } \quad \tilde{B}_{42}=\left(\begin{array}{rrr}
    1 & 2 & -1 \\
    -3 & 1 & -1 \\
    2 & -3 & 8
    \end{array}\right) .
    $$

    \begin{defBox}
        Let $A \in \mathrm{M}_{n \times n}(F)$. If $n=1$, so that $A=\left(A_{11}\right)$, we define $\operatorname{det}(A)=A_{11}$. For $n \geq 2$, we $\operatorname{define} \operatorname{det}(A)$ recursively as
        $$
        \operatorname{det}(A)=\sum_{j=1}^n(-1)^{1+j} A_{1 j} \cdot \operatorname{det}\left(\tilde{A}_{1 j}\right) .
        $$
    \end{defBox}

    The scalar $\operatorname{det}(A)$ is called the determinant of $A$ and is also denoted by $|A|$. The scalar
    $$
    (-1)^{i+j} \operatorname{det}\left(\tilde{A}_{i j}\right)
    $$
    is called the cofactor of the entry of $A$ in row $i$, column $j$.

    The determinant of a $2x2$ matrix $A$ would be $det(A) = A_{11}A_{22} - A_{21}A_{12}$. Also, remember that the determinant of a $3x3$ matrix can be obtained from the following algorithm
    \begin{center}
        \includegraphics[width=1\textwidth]{images/3x3det.png}
    \end{center}

    \begin{Example}
        Let
        $$
        A=\left(\begin{array}{rrr}
        1 & 3 & -3 \\
        -3 & -5 & 2 \\
        -4 & 4 & -6
        \end{array}\right) \in \mathrm{M}_{3 \times 3}(R) .
        $$

        Using cofactor expansion along the first row of $A$, we obtain
        $$
        \begin{aligned}
        \operatorname{det}(A)= & (-1)^{1+1} A_{11} \cdot \operatorname{det}\left(\tilde{A}_{11}\right)+(-1)^{1+2} A_{12} \cdot \operatorname{det}\left(\tilde{A}_{12}\right) \\
        & +(-1)^{1+3} A_{13} \cdot \operatorname{det}\left(\tilde{A}_{13}\right) \\
        = & (-1)^2(1) \cdot \operatorname{det}\left(\begin{array}{rr}
        -5 & 2 \\
        4 & -6
        \end{array}\right)+(-1)^3(3) \cdot \operatorname{det}\left(\begin{array}{rr}
        -3 & 2 \\
        -4 & -6
        \end{array}\right) \\
        & \quad+(-1)^4(-3) \cdot \operatorname{det}\left(\begin{array}{rr}
        -3 & -5 \\
        -4 & 4
        \end{array}\right) \\
        = & 1[-5(-6)-2(4)]-3[-3(-6)-2(-4)]-3[-3(4)-(-5)(-4)] \\
        = & 1(22)-3(26)-3(-32) \\
        = & 40 .
        \end{aligned}
        $$
    \end{Example}

    \subsubsection*{Properties of determinants}

    \begin{enumerate}
        \item If $B$ is a matrix obtained by interchanging any two rows or interchanging any two columns of an $n \times n$ matrix $A$, then $\operatorname{det}(B)=-\operatorname{det}(A)$.

        \item If $B$ is a matrix obtained by multiplying each entry of some row or column of an $n \times n$ matrix $A$ by a scalar $k$, then $\operatorname{det}(B)=k \cdot \operatorname{det}(A)$.

        \item If $B$ is a matrix obtained from an $n \times n$ matrix $A$ by adding a multiple of row $i$ to row $j$ or a multiple of column $i$ to column $j$ for $i \neq j$, then $\operatorname{det}(B)=\operatorname{det}(A)$.

        As an example of the use of these three properties in evaluating determinants, let us compute the determinant of the $4 \times 4$ matrix $A$ considered previously. Our procedure is to introduce zeros into the second column of $A$ by employing property 3, and then to expand along that column. (The elementary row operations used here consist of adding multiples of row 1 to rows 2 and 4.) This procedure yields
        $$
        \begin{aligned}
        \operatorname{det}(A) & =\operatorname{det}\left(\begin{array}{rrrr}
        2 & 1 & 1 & 5 \\
        1 & 1 & -4 & -1 \\
        2 & 0 & -3 & 1 \\
        3 & 6 & 1 & 2
        \end{array}\right)=\operatorname{det}\left(\begin{array}{rrrr}
        2 & 1 & 1 & 5 \\
        -1 & 0 & -5 & -6 \\
        2 & 0 & -3 & 1 \\
        -9 & 0 & -5 & -28
        \end{array}\right) \\
        & =1(-1)^{1+2} \operatorname{det}\left(\begin{array}{rrr}
        -1 & -5 & -6 \\
        2 & -3 & 1 \\
        -9 & -5 & -28
        \end{array}\right) .
        \end{aligned}
        $$

        The resulting determinant of a $3 \times 3$ matrix can be evaluated in the same manner: Use type 3 elementary row operations to introduce two zeros into the first column, and then expand along that column. This results in the value $-102$. Therefore
        $$
        \operatorname{det}(A)=1(-1)^{1+2}(-102)=102
        $$

        \item The determinant of an upper triangular matrix is the product of its diagonal entries. In particular, $\operatorname{det}(I)=1$.

        \item If two rows (or columns) of a matrix are identical, then the determinant of the matrix is zero.

        As an illustration of property 4, notice that
        $$
        \operatorname{det}\left(\begin{array}{rrr}
        -3 & 1 & 2 \\
        0 & 4 & 5 \\
        0 & 0 & -6
        \end{array}\right)=(-3)(4)(-6)=72.
        $$

        \item For any $n \times n$ matrices $A$ and $B$, $\operatorname{det}(A B)=\operatorname{det}(A) \cdot \operatorname{det}(B)$.

        \item An $n \times n$ matrix $A$ is invertible if and only if $\operatorname{det}(A) \neq 0$. Furthermore, if $A$ is invertible, then $\operatorname{det}\left(A^{-1}\right)=\frac{1}{\operatorname{det}(A)}$.

        \item For any $n \times n$ matrix $A$, the determinants of $A$ and $A^{t}$ are equal.

        For example, property 7 guarantees that the matrix $A$ on page 233 is invertible because $\operatorname{det}(A)=102$.

        The final property, stated as Exercise 15 of Section 4.3, is used in Chapter 5. It is a simple consequence of properties 6 and 7.

        \item If $A$ and $B$ are similar matrices, then $\operatorname{det}(A)=\operatorname{det}(B)$.
    \end{enumerate}


    \subsection*{Roots of a polynomial}
    The process of finding the roots of a polynomial can be either very straight forward or very complicated. Over the time we've developed multiple tools that help us in this task, we'll list a few of them to help us work with the characteristic polynomial. The most useful method for finding roots is to factorize the polynomials into factors of the form $(x - c)$ where $c$ is a real number, or find quadratic factors to then apply the quadratic formula. To find these factors we can use the synthetic division algorithm.

    \begin{thBox}
        \textit{\textbf{Fundamental Theorem Of Algebra}}. Every polynomial with complex coefficients has at least one complex zero.
    \end{thBox}

    \begin{thBox}
        \textit{\textbf{Roots theorem}}. Every polynomial of degree $n \geq 1$ has exactly $n$ zeros, provided that a zero of multiplicity $k$ is counted $k$ times.
    \end{thBox}

    \begin{thBox}
        \textit{\textbf{Rational Zeros Theorem}}. If the polynomial $P(x)=a_n x^n+a_{n-1} x^{n-1}+\cdots+a_1 x+a_0$ has integer coefficients, then every rational zero of $P$ is of the form
        $$
        \frac{p}{q}
        $$
        where $\quad p$ is a factor of the constant coefficient $a_0$
        and $\quad q$ is a factor of the leading coefficient $a_n$.
    \end{thBox}

    \begin{noteBox}
        \textit{\textbf{Finding the Rational Zeros of a Polynomial}}.
        \begin{enumerate}
            \item List Possible Zeros. List all possible rational zeros using the Rational Zeros Theorem.
            \item Divide. Use synthetic division to evaluate the polynomial at each of the candidates for rational zeros that you found in Step 1. When the remainder is 0 , note the quotient you have obtained.
            \item Repeat. Repeat Steps 1 and 2 for the quotient. Stop when you reach a quotient that is quadratic or factors easily, and use the quadratic formula or factor to find the remaining zeros.
        \end{enumerate}
    \end{noteBox}

    \section{Diagonalization of a Matrix}

    \subsection*{Multiplicity}

    \begin{defBox}
        \textit{\textbf{Geometrical multiplicity of an eigenvalue}}. Let $A$ be a squared matrix with the eigenvalue $\lambda$. The geometrical multiplicity of $\lambda$ is the dimension of $E_\lambda$ (The space generated by $\lambda$).
    \end{defBox}

    For example 1.2.5, the geometrical multiplicity of $\lambda = -1$ is $1$ since
    $$E_\lambda = \left\{ \begin{bmatrix}1\\0\\0\end{bmatrix} \right\}$$

    \begin{defBox}
        Let $A$ be a squared matrix with the eigenvalue $\lambda$. The algebraic multiplicity of $\lambda$ is the largest positive integer $k$ for which $(t-\lambda)^k$ is a factor of $f(t)$.
    \end{defBox}

    \begin{Example}
        Let

        $$A =\begin{pmatrix}
            3 & 1 & 0\\
            0 & 3 & 4\\
            0 & 0 & 4
        \end{pmatrix}$$

        which has characteristic polynomial $f(t) = -(t-3)^2(t-4)$. Hence $\lambda = 3$ is an eigenvalue of $A$ with multiplicity of 2, and $\lambda = 4$ is an eigenvalue of $A$ with multiplicity 1.
    \end{Example}

    \begin{thBox}
        Let $T$ be a linear operation on a finite-dimensional vector space $V$. And let $\lambda$ be an eigenvalue of $T$. Having algebraic multiplicity of $m$. Then $1 \leq dim(E_\lambda) \leq m$.
    \end{thBox}

    \subsection*{Similar matrices}

    \begin{defBox}
        \textit{\textbf{Similar matrices}}. Two square matrices $A$ and $B$ are said to be similar if there exists a non-singular matrix $P$ such that

        $$B = P^{-1}AP$$
    \end{defBox}

    \begin{thBox}
        Let $A \in \mathrm{M}_{n \times n}(F)$, and let $\gamma$ be an ordered basis for $\mathrm{F}^n$. Then $\left[\mathrm{L}_A\right]_\gamma=Q^{-1} A Q$, where $Q$ is the $n \times n$ matrix whose $j$ th column is the $j$ th vector of $\gamma$.
    \end{thBox}

    \begin{thBox}
        Similar matrices have the same eigenvalues counting multiplicity.
    \end{thBox}

    \subsection*{Diagonalization of a Matrix}

    \begin{defBox}
        \textit{\textbf{Diagonalization of a Matrix}}. A linear operator $\mathrm{T}$ on a finite-dimensional vector space $\mathrm{V}$ is called diagonalizable if there is an ordered basis $\beta$ for $\mathrm{V}$ such that $[\mathrm{T}]_\beta$ is a diagonal matrix.

        A square matrix $A$ is called diagonalizable if $\mathrm{L}_A$ is diagonalizable. In other words we say that an squared matrix $A$ is diagonalizable if and only if there exists an invertible matrix $P$ such that

        $$D = P^{-1}AP$$ or
        $$A = PDP^{-1}$$
    \end{defBox}

    \begin{thBox}
        If $A$ is similar to a diagonal matrix $D$, then the elements of $D$ are the eigenvalues of $A$; And the values are independent of the form in which a linear transformation is represented.
    \end{thBox}

    \begin{thBox}
        A $nxn$ matrix is diagonalizable if and only if admits a linearly independent eigenvectors.
    \end{thBox}

    \begin{thBox}
        The following are equivalent

        \begin{enumerate}
            \item $T: V \to V$ is diagonalizable
            \item For every eigenvalue $\lambda$ of $T$, the geometric multiplicity of $\lambda$ coincides with the algebra multiplicity.
        \end{enumerate}
    \end{thBox}

    \subsection*{Change of Basis}

    \begin{defBox}
        \textit{\textbf{Ordered basis}}. Let $V$ be a finite-dimensional vector space. An ordered basis for such space is endowed with a specific order, that is, an ordered basis for $V$ is a sequence of linearly independent vectors in $V$ that generate $V$.
    \end{defBox}

    \begin{Example}
        Let $V = F^3$.

        $$\beta = \left\{ e_1, e_2, e_3 \right\}$$
        $$\alpha = \left\{ e_2, e_1, e_3 \right\}$$

        Then $\beta \not = \alpha$. And $\beta, \alpha$ are ordered basis.
    \end{Example}

    \begin{defBox}
        Let $\beta = \left\{ u_1, u_2, \dots , u_n \right\}$ be an ordered basis for $V$. For each $v \in V$, let $\alpha_1, \alpha_2, \dots , \alpha_n$ be the unique scalars such that

        $$v = \sum_{i=1}^{n}\alpha_iu_i$$

        We define the coordinate vector of $v$ relative to $\beta$, denoted by

        $$[v]_\beta = \begin{pmatrix}
            \alpha_1\\ \alpha_2 \\ \vdots \\ \alpha_n
        \end{pmatrix}$$
    \end{defBox}

    \begin{Example}
        Let $P_2(\mathbb{R})$ and let $\beta = \left\{ 1, x, x^2 \right\}$ and $\alpha = \left\{ x^2, x, 1 \right\}$ if $v = f(x) = 4 + 6x -7x^2$.

        $$[v]_\beta = \begin{pmatrix}
            4\\6\\-7
        \end{pmatrix}$$

        $$[v]_\alpha = \begin{pmatrix}
            -7\\6\\4
        \end{pmatrix}$$
    \end{Example}

    \begin{thBox}
        Let $\beta = \left\{ u_1, u_2, \dots , u_n \right\}$ an ordered basis for $V$, then a coordinate map

        $$[]_\beta: V \to \mathbb{R}^n$$
        $$V \to [v]_\beta$$

        Is a linear transformation from $V \to \mathbb{R}^n$
    \end{thBox}

    \begin{defBox}
        We call $mxn$ matrix $A$ defined by $A_{ij} = a_{ij}$. The matrix representation of $T$ in the ordered basis $\beta$ and $\beta'$ and write $A = [T]_\beta$.\\

        If $V = W$ and $\beta = \beta'$, then $A = [T]_\beta$.
    \end{defBox}

    \begin{Example}
        Let $\beta = \{u_1, u_2, \dots , u_n\}$ be an ordered basis for $V$, and let $T$ be a coordinate mapping such that $T(v) = [v]_\beta$. Then $T$ is a linear transformation.

        \textit{\textbf{Proof}}. First, let's show that $T(v + w) = T(v) + T(w)$ for all $v, w \in V$.

        Let $v, w \in V$. We have:

        $$T(v) = [v]_\beta = \begin{bmatrix}
            \alpha_1\\ \alpha_2\\ \vdots \\ \alpha_n
        \end{bmatrix}$$

        where $\alpha_1,\dots, \alpha_n$ are scalars such that

        $$v = \sum_{i=1}^{n} \alpha_i u_i$$

        and similarly,

        $$T(w) = [w]_\beta = \begin{bmatrix}
            \gamma_1\\ \gamma_2\\ \vdots \\ \gamma_n
        \end{bmatrix}$$

        where $\gamma_1,\dots, \gamma_n$ are scalars such that

        $$w = \sum_{i=1}^{n} \gamma_i u_i$$

        Therefore,

        \begin{align*}
        T(v + w) &= [v + w]_\beta \\
        &= \left[\sum_{i=1}^{n} (\alpha_i + \gamma_i) u_i\right]_\beta \\
        &= \begin{bmatrix}
            \alpha_1 + \gamma_1\\ \alpha_2 + \gamma_2\\ \vdots \\ \alpha_n + \gamma_n
        \end{bmatrix} \\
        &= \begin{bmatrix}
            \alpha_1\\ \alpha_2\\ \vdots \\ \alpha_n
        \end{bmatrix} + \begin{bmatrix}
            \gamma_1\\ \gamma_2\\ \vdots \\ \gamma_n
        \end{bmatrix} \\
        &= T(v) + T(w)
        \end{align*}

        Next, let's show that $T(cv) = cT(v)$ for all $c \in \mathbb{R}$ and all $v \in V$.

        Let $v \in V$ and $c \in \mathbb{R}$. Then we have:

        $$T(v) = [v]_\beta = \begin{bmatrix}
            \alpha_1\\ \alpha_2\\ \vdots \\ \alpha_n
        \end{bmatrix}$$

        where $\alpha_1,\dots, \alpha_n$ are scalars such that

        $$v = \sum_{i=1}^{n} \alpha_i u_i$$

        So,

        \begin{align*}
        T(cv) &= [cv]_\beta \\
        &= \left[\sum_{i=1}^{n} (c\alpha_i) u_i\right]_\beta \\
        &= \begin{bmatrix}
            c\alpha_1\\ c\alpha_2\\ \vdots \\ c\alpha_n
        \end{bmatrix} \\
        &= c\begin{bmatrix}
            \alpha_1\\ \alpha_2\\ \vdots \\ \alpha_n
        \end{bmatrix} \\
        &= cT(v)
        \end{align*}

        Thus, $T$ is a linear transformation.

    \end{Example}

    \begin{Example}
        Sea $T: \mathcal{M}_{2x2}(\mathbb{R}) \to P_2(\mathbb{R})$ tal que

        $$T\left(\begin{bmatrix}
            a & b \\
            c & d
        \end{bmatrix}\right)$$

        y sean $\beta_1$ y $\beta_2$ bases de $\mathcal{M}_{2x2}(\mathbb{R})$ y $P_2(\mathbb{R})$ tal que

        $$\beta_1 = \left\{
            \begin{bmatrix}
                1 & 0 \\
                0 & 0
            \end{bmatrix},
            \begin{bmatrix}
                0 & 1 \\
                0 & 0
            \end{bmatrix},
            \begin{bmatrix}
                0 & 0 \\
                1 & 0
            \end{bmatrix},
            \begin{bmatrix}
                0 & 0 \\
                0 & 1
            \end{bmatrix}
        \right\}$$

        $$\beta_2 = \left\{1, x, x^2\right\}$$

        Encuentre $[T]_{\beta_1}^{\beta_2}$.

        Para encontrar $[T]_{\beta_1}^{\beta_2}$ debemos aplicar la transformación a cada vector de $\beta_1$ y luego obtener la matriz de la transformación. En este orden de ideas, para $v_j \in \beta_1$ tenemos lo siguiente:

        $$T(v_1) = T \left(\begin{bmatrix}
            1 & 0 \\
            0 & 0
        \end{bmatrix}\right) = (1 + 0) + 2\cdot 0x + 0x^2 = 1$$

        $$T(v_2) = T \left(\begin{bmatrix}
            0 & 1 \\
            0 & 0
        \end{bmatrix}\right) = (0 + 1) + 2\cdot 0x + 1x^2 = 1 + x^2$$

        $$T(v_3) = T \left(\begin{bmatrix}
            0 & 0 \\
            1 & 0
        \end{bmatrix}\right) = (0 + 0) + 2\cdot 0x + 0x^2 = 0$$

        $$T(v_4) = T \left(\begin{bmatrix}
            0 & 0 \\
            0 & 1
        \end{bmatrix}\right) = (0 + 0) + 2\cdot 1x + 0x^2 = 2x$$

        Luego

        $$[T]_{\beta_1}^{\beta_2} = \begin{bmatrix}
            1 & 1 & 0 & 0\\
            0 & 0 & 0 & 2\\
            0 & 1 & 0 & 0
        \end{bmatrix}$$
    \end{Example}

    \begin{Example}
            Let $T: \mathcal{M}_{2 \times 2}(\mathbb{R}) \to P_2(\mathbb{R})$ be such that

        $$T\left(\begin{bmatrix}
            a & b \\
            c & d
        \end{bmatrix}\right)$$

        and let $\beta_1$ and $\beta_2$ be bases of $\mathcal{M}_{2 \times 2}(\mathbb{R})$ and $P_2(\mathbb{R})$ respectively, given by

        $$\beta_1 = \left\{
            \begin{bmatrix}
                1 & 0 \\
                0 & 0
            \end{bmatrix},
            \begin{bmatrix}
                0 & 1 \\
                0 & 0
            \end{bmatrix},
            \begin{bmatrix}
                0 & 0 \\
                1 & 0
            \end{bmatrix},
            \begin{bmatrix}
                0 & 0 \\
                0 & 1
            \end{bmatrix}
        \right\}$$

        $$\beta_2 = \left\{1, x, x^2\right\}$$

        Find $[T]_{\beta_1}^{\beta_2}$.

        To find $[T]_{\beta_1}^{\beta_2}$, we must apply the transformation to each vector in $\beta_1$ and then obtain the matrix of the transformation. In this vein, for $v_j \in \beta_1$ we have the following:

        $$T(v_1) = T \left(\begin{bmatrix}
            1 & 0 \\
            0 & 0
        \end{bmatrix}\right) = (1 + 0) + 2 \cdot 0x + 0x^2 = 1$$

        $$T(v_2) = T \left(\begin{bmatrix}
            0 & 1 \\
            0 & 0
        \end{bmatrix}\right) = (0 + 1) + 2 \cdot 0x + 1x^2 = 1 + x^2$$

        $$T(v_3) = T \left(\begin{bmatrix}
            0 & 0 \\
            1 & 0
        \end{bmatrix}\right) = (0 + 0) + 2 \cdot 0x + 0x^2 = 0$$

        $$T(v_4) = T \left(\begin{bmatrix}
            0 & 0 \\
            0 & 1
        \end{bmatrix}\right) = (0 + 0) + 2 \cdot 1x + 0x^2 = 2x$$

        Then

        $$[T]_{\beta_1}^{\beta_2} = \begin{bmatrix}
            1 & 1 & 0 & 0 \\
            0 & 0 & 0 & 2 \\
            0 & 1 & 0 & 0
        \end{bmatrix}$$

        Let $F: \mathbb{R}^3 \to \mathbb{R}^2$ be defined as

        $$F(x, y, z) = (3x + 2y - 4z, x - 5y + 3z)$$

        \begin{enumerate}
            \item Find the matrix of $F$ for the following bases of $\mathbb{R}^3$ and $\mathbb{R}^2$

            $$\beta_1 = \{(1,1,1), (1,1,0), (1,0,0)\}$$
            $$\beta_2 = \{(1,3), (2,5)\}$$

            \item Given $V \in \mathbb{R}^3$, verify
            $$[F]_{\beta_1}^{\beta_2}[v]_{\beta_1} = [F(v)]_{\beta_2}$$
        \end{enumerate}

        \begin{enumerate}
            \item By definition of $F$, 
            $$F((1,1,1)) = (3(1) + 2(1) - 4(1), 1 - 5(1) + 3(1)) = (1, -1)$$
            $$F((1,1,0)) = (3(1) + 2(1) - 4(0), 1 - 5(1) + 3(0)) = (5, -4)$$
            and
            $$F((1,0,0)) = (3(1) + 2(0) - 4(0), 1 - 5(0) + 3(0)) = (3, 1)$$

            Then
            $$\begin{bmatrix}
                1 \\ -1
            \end{bmatrix}_{\beta_2} = \begin{bmatrix}
                -7 \\ 1
            \end{bmatrix}$$
            $$\begin{bmatrix}
                5 \\ -4
            \end{bmatrix}_{\beta_2} = \begin{bmatrix}
                -33 \\ 19
            \end{bmatrix}$$
            $$\begin{bmatrix}
                3 \\ -1
            \end{bmatrix}_{\beta_2} = \begin{bmatrix}
                -13 \\ 8
            \end{bmatrix}$$

            Finally,

            $$[F]_{\beta_1}^{\beta_2} = \begin{bmatrix}
                -7 & -33 & -13 \\
                1 & 19 & 8
            \end{bmatrix}$$

            \item Let $v = (a, b, c)$. We have

            $$[F(v)]_{\beta_2} = \begin{bmatrix}
                a \\ b
            \end{bmatrix}_{\beta_2} = \begin{bmatrix}
                \begin{bmatrix}
                    1 & 2 & 3a + 2b - 4c \\
                    3 & 5 & a - 5b + 3c
                \end{bmatrix} = \begin{bmatrix}
                    -13a - 20b + 26c \\
                    8a + 11b - 15c
                \end{bmatrix}
            \end{bmatrix}$$

            Similarly, we have

            $$[v]_{\beta_1} = \begin{bmatrix}
                c \\ b - c \\ a - b
            \end{bmatrix}$$

            Finally

            \begin{align*}
                [F]_{\beta_1}^{\beta_2}[v]_{\beta_1} &=
                \begin{bmatrix}
                    -7 & -33 & -13 \\
                    1 & 19 & 8
                \end{bmatrix} \begin{bmatrix}
                    c \\
                    b - c \\
                    a - b
                \end{bmatrix} \\
                &= \begin{bmatrix}
                    -13a - 20b + 26c \\
                    8a + 11b - 15c
                \end{bmatrix} \\
                &= [F(v)]_{\beta_2}
            \end{align*}
        \end{enumerate}
    \end{Example}

    \begin{thBox}
        If $T: V \to V$ is a linear operator and $\beta = \left\{ \alpha_1, \dots , \alpha_n \right\}$ and $\beta' = \left\{ \alpha_1', \dots , \alpha_n' \right\}$ two ordered bases for $V$, there exists a matrix $P \in M_{nxn}$ invertible such that

        $$[T]_{\beta'} = P^{-1}[T]P$$
    \end{thBox}

    \begin{thBox}
        Let $V$ and $W$ be finite dimensional vector space $V$ with the ordered bases $\beta$ and $\gamma$, respectively and let $T: V \to W$ be a linear transformation, for every $v \in V$, we have

        $$[T(v)]_\gamma = [T]_\beta^\gamma[v]_\beta$$
    \end{thBox}

    \begin{thBox}
        If $T: V \to W$ is a linar transformation, such that $T \in L(V, W)$, and $\beta = \left\{ \alpha_1, \dots \alpha_n \right\}$ a base for $V$, and $\beta' = \left\{ \alpha_1', \dots \alpha_n' \right\}$ be a base for $W$, then there exists a unique matrix $A \in M_{nxn}$ such that for each $x \in V$

        $$[T(x)]_\beta = A[x]_\beta$$
    \end{thBox}

    \section{Sum of subspaces}

    \begin{defBox}
        \textit{\textbf{Sum of sets}}. Suppose that $U_1, \dots U_m$ are subsets of a vector space $V$. Then the sum of $U_1, \dots , U_m$ denoted by $U_1 + \cdots + U_m $ is the set of all possible sums of the elements of $U_1, \dots , U_m$, more precisely

        $$U_1 + \cdots U_m = \left\{ u_1 + \cdots u_m \mid u_1 \in U_1 , \dots , u_m \in U_m \right\}$$
    \end{defBox}

    \begin{Example}
        $$U = \left\{ (x, 0, 0) \in F^3 \mid x \in F \right\}$$
        $$W = \left\{ (0, y, 0) \in F^3 \mid x \in F \right\}$$

        Then

        $$U + W = \left\{ (x, y, 0) \in F^3 \mid x, y \in F \right\}$$
    \end{Example}

    \begin{Example}
        $$W_1 = \left\{ m(1,0) \mid m \in \mathbb{R} \right\}$$
        $$W_2 = \left\{ l(0,1) \mid l \in \mathbb{R} \right\}$$

        Then

        $$W_1 + W_2 = \left\{ m(1,0) + l(1,0) \mid m, l \in \mathbb{R} \right\}$$
    \end{Example}

    \begin{thBox}
        \textit{\textbf{The sum of subspaces is also a subspace}}. Let $V_1, \dots , V_n$ subspaces of a vector space $V$, then

        $$ V_1 + \cdots V_n $$
    \end{thBox}

    % \begin{defBox}
    %     \textit{\textbf{Direct Sum}}. Suppose $U_1, \dots , U_n$ are subspaces of $V$.

    %     \begin{enumerate}
    %         \item The sum $U_1 + \cdots + U_n$ then we call it a direct sum if each element of $U_1 + \cdots + U_n$ can be written in a unique form like a sum $U_1 + \cdots + U_m$
    %     \end{enumerate}
    % \end{defBox}
\end{document}
