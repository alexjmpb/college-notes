\documentclass{report}
\usepackage[english]{babel}

\usepackage[most,many,breakable]{tcolorbox}
\usepackage{xcolor}

\definecolor{defBoxBorder}{HTML}{395144}
\newtcolorbox{defBox}{colback=white,colframe=defBoxBorder,arc=3pt, boxrule=0.5pt, drop fuzzy shadow, title=Definition}
\definecolor{thBoxBorder}{HTML}{AC8441}
\newtcolorbox{thBox}{colback=white,colframe=thBoxBorder,arc=3pt, boxrule=0.5pt, drop fuzzy shadow, title=Theorem}
\definecolor{noteBoxBorder}{HTML}{4E6C50}
\newtcolorbox{noteBox}{colback=white,colframe=noteBoxBorder,arc=3pt, boxrule=0.5pt, drop fuzzy shadow, title=Note}
\definecolor{axBoxBorder}{HTML}{AA5656}
\newtcolorbox{axBox}{colback=white,colframe=axBoxBorder,arc=3pt, boxrule=0.5pt, drop fuzzy shadow, title=Axiom/Postulate}
\definecolor{corBoxBorder}{HTML}{8B7E74}
\newtcolorbox{corBox}{colback=white,colframe=corBoxBorder,arc=3pt, boxrule=0.5pt, drop fuzzy shadow, title=Corollary}
\definecolor{lemBoxBorder}{HTML}{B99B6B}
\newtcolorbox{lemBox}{colback=white,colframe=lemBoxBorder,arc=3pt, boxrule=0.5pt, drop fuzzy shadow, title=Lemma}
\definecolor{asBoxColor}{HTML}{FDFDF9}
\definecolor{asBoxBorder}{HTML}{DEB881}
\newtcolorbox{asBox}{coltext=black, colback=asBoxColor,colframe=asBoxBorder,arc=3pt, boxrule=0.5pt, drop fuzzy shadow, title=Aside}


\input{setup.tex}

\begin{document}
    \coverPage{ Mathematics }{ Linear Algebra II }{ Eigenvalues and Eigenvectors }{  }{ Alexander Mendoza }{\today}
    \tableofcontents
    \pagebreak
    \chapter{ Eigenvalues and Eigenvectors }

    \section{The vector space of linear combinations}

    Let $\mathcal{T}$ be the set of all linear transformations from $V$ into $W$, and let $L(V,W) = \{T| T \in \mathcal{T}\}$, then $L(V,W)$ is a vector space where its operations are defined as:

    \begin{itemize}
        \item \textit{\textbf{Vector addition}}. Let $T_1, T_2 \in L(V,W)$ then for every $v_1 \in V$
        $$(T_1+T_2)(v_1) = T_1(v_1) + T_2(v_1)$$
        \item \textit{\textbf{Scalar multiplication}}. Let $T \in L(V,W)$ and let $\alpha \in F$, then for all $v_1 \in V$
        $$(\alpha T)(v_1) = \alpha T(v_1)$$
    \end{itemize}

    \section{Eigenvalues and Eigenvectors}

    \begin{defBox}
        \textit{\textbf{Definition}}. Let $T$ be a linear operation on vector space $V$. A nonzero vector $v$ is called an eigenvector of $T$ if there exists an scalar $\lambda \in F$ such that $T(v) = \lambda v$. The scalar $\lambda$ is called the eigenvalue corresponding to the eigenvector $v$.
    \end{defBox}

    \begin{thBox}
        Let $T$ be a linear operator over a finite dimensional vector space $V$. And let $\lambda$ be a scalar, then the following are equivalent

        \begin{enumerate}
            \item $\lambda$ is an eigenvalue of $T$
            \item The operator $(T - \lambda I)$ is singular (it doesn't have inverse)
            \item $\det(T-\lambda I) = 0$
        \end{enumerate}
    \end{thBox}

    \begin{defBox}
        If $A$ is a matrix over a field $F$, an eigenvalue of $A$ in $F$ is a scalar in $F$, such that $(A - \lambda I)$ is singular.
    \end{defBox}
\end{document}
