\documentclass{report}
\usepackage[english]{babel}

\usepackage[most,many,breakable]{tcolorbox}
\usepackage{xcolor}

\definecolor{defBoxBorder}{HTML}{395144}
\newtcolorbox{defBox}{colback=white,colframe=defBoxBorder,arc=3pt, boxrule=0.5pt, drop fuzzy shadow, title=Definition}
\definecolor{thBoxBorder}{HTML}{AC8441}
\newtcolorbox{thBox}{colback=white,colframe=thBoxBorder,arc=3pt, boxrule=0.5pt, drop fuzzy shadow, title=Theorem}
\definecolor{noteBoxBorder}{HTML}{4E6C50}
\newtcolorbox{noteBox}{colback=white,colframe=noteBoxBorder,arc=3pt, boxrule=0.5pt, drop fuzzy shadow, title=Note}
\definecolor{axBoxBorder}{HTML}{AA5656}
\newtcolorbox{axBox}{colback=white,colframe=axBoxBorder,arc=3pt, boxrule=0.5pt, drop fuzzy shadow, title=Axiom/Postulate}
\definecolor{corBoxBorder}{HTML}{8B7E74}
\newtcolorbox{corBox}{colback=white,colframe=corBoxBorder,arc=3pt, boxrule=0.5pt, drop fuzzy shadow, title=Corollary}
\definecolor{lemBoxBorder}{HTML}{B99B6B}
\newtcolorbox{lemBox}{colback=white,colframe=lemBoxBorder,arc=3pt, boxrule=0.5pt, drop fuzzy shadow, title=Lemma}
\definecolor{asBoxColor}{HTML}{FDFDF9}
\definecolor{asBoxBorder}{HTML}{DEB881}
\newtcolorbox{asBox}{coltext=black, colback=asBoxColor,colframe=asBoxBorder,arc=3pt, boxrule=0.5pt, drop fuzzy shadow, title=Aside}


\input{setup.tex}

\begin{document}
    \coverPage{ Mathematics }{ Integral Calculus and Series }{ Integrability }{  }{ Alexander Mendoza }{\today}
    \tableofcontents

    \chapter*{Prologue}
    After visiting the introductory to infinitesimal Calculus, differential Calculus, we'll now continue with the course that will unleash the full potential of the derivative, Integral Calculus. In this first part of the course we'll begin by giving an intuitive idea of what the integral is, stating the problems that originated it. We will then formally define the integral of a function and state its properties to easily find the integral of a function. And we will finally provide the theorem that unifies differential and integral calculus, thus making a relation between the integral and the derivative of a function.

    \pagebreak
    \chapter{ Integrability }
    

    % Add an introduction to the relation between integral and derivatives
    To define the integral we will first need some definitions.
    \section*{Inferior and Superior Sums}
    We will use both the superior and inferior sums to define the concept of an integral.

    \begin{defBox}
        \textit{\textbf{Partition of an integral.}} Let $a, b$ be real numbers such that $a<b$ and let $P$ be the following set

        $$P = \{x_0, x_1, \dots , x_n\}$$ with $a=x_0<x_1<\cdots<x_n=b$, then we say that $P$ is a partition of $[a,b]$
    \end{defBox}

    % Insert graphic
    Figure 1, represents graphically the concept of partition.

    \begin{defBox}
        Let $f$ be a function bounded on $[a,b]$ and let $P = \{x_0, x_1, \dots x_n\}$ be a partition of $[a,b]$. then

        The minimum of a function in $[a,b]$ is defined as
        $$m_i := inf\{f(x) | x_{i-1} \leq x \leq x_i\}$$
        And the maximum of a function in $[a,b]$ is defined as
        $$M_i := sup\{f(x) | x_{i-1} \leq x \leq x_i\}$$
    \end{defBox}

    % Add examples

    With this we can now define the upper and lower sums.

    \begin{defBox}
        Suppose $f$ is bounded on $[a,b]$, let $P = \{t_0,\dots,t_n\}$ be a partition of $[a,b]$ and let

        $$m_i = inf\{f(x) | x_{i-1} \leq x \leq x_i\}$$
        $$M_i = sup\{f(x) | x_{i-1} \leq x \leq x_i\}$$

        Then we define the lower sum of $f$ for $P$, denoted by $L(f, P)$, as

        $$L(f, P) = \sum_{i=1}^{n}m_i(t_{i} - t_{i-1})$$

        and the upper sum of $f$ for $P$, denoted by $U(f, P)$, is defined as

        $$U(f, P) = \sum_{i=1}^{n}M_i(t_{i} - t_{i-1})$$
    \end{defBox}

    % add examples

    Let's now see which relation the lower and upper sum have. Our intuition may tell us that given $f$ bounded over $[a,b]$ and $P$ a partition of $[a,b]$, $L(f, P) \leq U(f, P)$, in fact this is true for any pair of distinct partitions. But, let's review first a lemma to demonstrate this fact.

    \begin{lemBox}
        Let $f$ be a bounded function and $P, Q$ be partitions of $[a,b]$. If $Q \subseteq P$, then

        $$L(f, Q) \leq L(f, P) $$
        $$U(f, P) \leq U(f, Q) $$
    \end{lemBox}

    \begin{ideaBox}
        The idea for this proof is to first demonstrate that the lemma is true for a partition that contains only one more element, and then construct a succession of partitions, where each new partition will have one more element than the partition before it, and proof the general case.
        % Expand more on this idea
    \end{ideaBox}

    \textit{\textbf{Proof.}} Let $f$ be a bounded function and $P, Q$ be partitions of $[a,b]$ where $Q \subseteq P$ and $P$ contains only one more element than $Q$, namely $u$. We can represent $P$ and $Q$ as

    $$P = \{x_0,x_1,\dots,x_{k-1},x_k,\dots,x_n\}$$
    $$Q = \{x_0,x_1,\dots,x_{k-1},u,x_k,\dots,x_n\}$$

    Then by definition we have

    \begin{align*}
        L(f,P) &= \sum_{i=1}^{n} m_i(x_i-x_{i-1})\\
        &= \sum_{i=1}^{k-1}m_i(x_i-x_{i-1}) + m_k(x_k-x_{k-1}) + \sum_{i=k+1}^{n}m_i(x_i-x_{i-1})
    \end{align*}

    Similarly

    \begin{align*}
        L(f,Q) &= \sum_{i=1}^{n} m_i(x_i-x_{i-1})\\
        &= \sum_{i=1}^{k-1}m_i(x_i-x_{i-1}) + m'(u-x_{k-1}) + m''(x_k-u) + \sum_{i=k+1}^{n}m_i(x_i-x_{i-1})
    \end{align*}

    % COMPLETE THE PROOF

    With this we can state a theorem that will allow us to then define the integral of a function. Before providing the theorem, a useful observation is that, given any partition $P$
    $$L(f, P) \leq U(f, P)$$
    We know this because
    $$L(f, P) = \sum_{i=1}^{n}m_i(t_{i} - t_{i-1})$$
    $$U(f, P) = \sum_{i=1}^{n}M_i(t_{i} - t_{i-1})$$
    and by definition, for any $i$:
    \begin{align*}
        m_i &\leq M_i\\
        m_i(t_{i} - t_{i-1}) &\leq M_i(t_{i} - t_{i-1})
    \end{align*}

    \begin{thBox}
        Let $P_1, P_2$ be partitions of an interval $[a,b]$ and let $f$ be bounded over $[a,b]$, then it's true that

        $$L(f, P_1) \leq U(f, P_2)$$
    \end{thBox}

    \textit{\textbf{Proof.}} Let $P = P_1 \cup P_2$, then by definition of union, $P_1 \subseteq P$ and $P_2 \subseteq P$, and by the previous lemma:

    $$L(f, P_1) \leq L(f, P)$$
    $$U(f, P) \leq U(f, P_2)$$
    By the afore mentioned observation, we have
    $$L(f, P) \leq U(f, P)$$
    And all together
    $$L(f, P_1) \leq L(f, P) \leq U(f, P) \leq U(f, P_2)$$
    Finally, by transitivity, we therefore have
    $$L(f, P_1) \leq U(f, P_2)$$
    Concluding our proof.

    % EXPAND ON WHY THIS IS IMPORTANT

    \subsection*{The integral of a function}

    % Description
    \begin{defBox}
        Let $f$ be a function bounded on $[a,b]$, $f$ is said to be integrable on $[a,b]$ if


        \begin{align*}
            sup\{L(f,P) | P \text{ is a partition of } [a,b]\}\\
            = inf\{U(f,P) | P \text{ is a partition of } [a,b]\}
        \end{align*}

        That common number is what we call the integral of $f$ over $[a,b]$ and is denoted by

        $$\int_{b}^{a}f$$

        And is read "the integral of $f$ over $[a,b]$"
    \end{defBox}

    % INSERT EXAMPLES
\end{document}
