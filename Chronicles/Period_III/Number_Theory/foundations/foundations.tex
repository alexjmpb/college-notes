\documentclass{report}
\usepackage[english]{babel}

\usepackage[most,many,breakable]{tcolorbox}
\usepackage{xcolor}

\definecolor{defBoxBorder}{HTML}{395144}
\newtcolorbox{defBox}{colback=white,colframe=defBoxBorder,arc=3pt, boxrule=0.5pt, drop fuzzy shadow, title=Definition}
\definecolor{thBoxBorder}{HTML}{AC8441}
\newtcolorbox{thBox}{colback=white,colframe=thBoxBorder,arc=3pt, boxrule=0.5pt, drop fuzzy shadow, title=Theorem}
\definecolor{noteBoxBorder}{HTML}{4E6C50}
\newtcolorbox{noteBox}{colback=white,colframe=noteBoxBorder,arc=3pt, boxrule=0.5pt, drop fuzzy shadow, title=Note}
\definecolor{axBoxBorder}{HTML}{AA5656}
\newtcolorbox{axBox}{colback=white,colframe=axBoxBorder,arc=3pt, boxrule=0.5pt, drop fuzzy shadow, title=Axiom/Postulate}
\definecolor{corBoxBorder}{HTML}{8B7E74}
\newtcolorbox{corBox}{colback=white,colframe=corBoxBorder,arc=3pt, boxrule=0.5pt, drop fuzzy shadow, title=Corollary}
\definecolor{lemBoxBorder}{HTML}{B99B6B}
\newtcolorbox{lemBox}{colback=white,colframe=lemBoxBorder,arc=3pt, boxrule=0.5pt, drop fuzzy shadow, title=Lemma}
\definecolor{asBoxColor}{HTML}{FDFDF9}
\definecolor{asBoxBorder}{HTML}{DEB881}
\newtcolorbox{asBox}{coltext=black, colback=asBoxColor,colframe=asBoxBorder,arc=3pt, boxrule=0.5pt, drop fuzzy shadow, title=Aside}


\input{setup.tex}

\begin{document}
    \coverPage{ Mathematics }{ Number Theory }{ Foundations }{ Subtitle example. }{ Alexander Mendoza }{\today}
    \tableofcontents

    \pagebreak
    \chapter{ Foundations }

    \section{Definition of the natural numbers and its operations}
    Let's begin by defining our workbench, the natural numbers. We'll define the natural numbers in the most standard way, that is using Peano's axioms.

    \subsection*{Peano's Axioms}
    There exists the set of natural numbers $\mathbb{N}$
    % Correct enumerate with axioms enumerate
    \begin{enumerate}
        \item $0 \in \mathbb{N}$.
        \item If $x$ is a natural number, then $s(x)\in \mathbb{N}$.
        \item Does not exists $x \in \mathbb{N}$ such that $s(x) = 0$.
        \item Let $x,y \in \mathbb{N}$, if $s(x) = s(y)$, then $x=y$.
        \item Let $B \subseteq \mathbb{N}$ such that:
            \begin{itemize}
                \item $0 \in B$
                \item If $n\in B$, then $s(n)=B$
            \end{itemize}
        \begin{noteBox}
            % IMPROVE GRAMMAR
            $B$ is said to be an inductive set. The original Axioms stated $B$ as any set instead of a subset of $\mathbb{N}$, and then, by a theorem, $\mathbb{N}$ would be the smallest of all inductive sets. To avoid all this work, we state that $B$ is a subset of $\mathbb{N}$ and from this we can conclude that $B = \mathbb{N}$
        \end{noteBox}
    \end{enumerate}

    The last axiom is known as the Principle of Mathematical Induction (PMI), we'll elaborate further in this topic since it's going to be used everywhere from now on.

    \subsection*{Addition}

    Addition is a recursive function $+$ with the following properties:

    $$+: \mathbb{N}\times \mathbb{N} \to \mathbb{N}$$ where given $m,n \in \mathbb{N}$:

    $$+(m,0) = m$$
    $$+(m,s(n)) = s(+(m,n))$$

    The output of the function is called the sum of $m$ and $n$ and is denoted as $m+n$ for convenience, with this notation our definition can be viewed as:

    $$m+0 = m$$
    $$m + s(n) = s(m, n)$$

    Let's give a couple of examples to verify that this function is indeed the sum we are familiar with.

    \begin{Example}
        Let $m=2$ and $n=3$, then by definition

        \begin{align*}
            2+3 &= s(2+2)\\
            &= s(s(2+1))\\
            &= s(s(s(2+0)))\\
            &= s(s(s(2)))\\
            &= s(s(3))\\
            &= s(4)\\
            &= 5
        \end{align*}
    \end{Example}
\end{document}
