\documentclass[11pt]{article}
\usepackage[english]{babel}
\usepackage[utf8x]{inputenc}
\usepackage{amsmath}
\usepackage{graphicx}
\usepackage[colorinlistoftodos]{todonotes}
\usepackage{enumitem}
\usepackage{listings}
\usepackage{verbatim}
\usepackage{eurosym}
\usepackage[export]{adjustbox}
\usepackage{amssymb}
\usepackage{bussproofs}
\usepackage{amsmath}
\usepackage{tikz}

%----------------------------------------------------------------------------------------
%	COVER START
%----------------------------------------------------------------------------------------
\begin{document}

    \begin{titlepage}

        \newcommand{\HRule}{\rule{\linewidth}{0.5mm}}
        \newcommand{\department}{Math Department}
        \newcommand{\course}{Fundamentals of Mathematics}
        \newcommand{\titleValue}{Chronicals}
        \newcommand{\subtitleValue}{Set Theory}
        \newcommand{\authorName}{Alexander Mendoza}

        \center

        %----------------------------------------------------------------------------------------
        %	HEADER
        %----------------------------------------------------------------------------------------

        \vspace{0.5cm}
        \textsc{\Large \department}\\[0.5cm]
        \textsc{\Large \course}\\[0.5cm]
        \vfill

        %----------------------------------------------------------------------------------------
        %	TITLE
        %----------------------------------------------------------------------------------------

        \HRule\\
        \Huge
        \textbf{\titleValue}\\[0.5cm]
        \Large
        \textbf{\subtitleValue}\\
        \HRule\\[0.5cm]

        %----------------------------------------------------------------------------------------
        %	AUTHOR AND DATE
        %----------------------------------------------------------------------------------------

        \vfill
        \Large
        \textit{\authorName}\\
        {\large \today}\\[2cm]

    \end{titlepage}
%----------------------------------------------------------------------------------------
%	COVER END
%----------------------------------------------------------------------------------------
\section{Sets}
    The intuitively idea of a set is that a set is a collection of objects, not necessarily mathematical ones. Almost all mathematical contexts are constructed from this idea of a set. We can take the set of all people in Colombia. The objects contained in the set are called the elements of the set. If $A$ is a set and $a$ is an element of $A$, we write $a \in A$. If $a$ is not in the set $A$, we write $a \notin A$.

    Logically, either $a \in A$ or $a \notin A$ but not both.\\

    The two most common way of presenting a set is:\\
    \begin{itemize}
        \item To list its elements $A = \{a, b, c\}$.
        \item To provide a statement that all the elements of the set must satisfy in order to be part of it. $ A = \{x\ |\ x > 0\} $.
    \end{itemize}
    \subsection{Subsets}
        A subset has the following definition. $ A \subseteq B \Leftrightarrow \forall x(x \in A \rightarrow x \in B) $. That is, a set $A$ is a subset of a set $B$ if and only if for all $x$ that belongs to $A$, $x$ also belongs to $B$.

        The following is the most common way to prove that a set A is a subset of a set B. First let $x \in A$ ... (argumentation) ... then $a \in B$. Hence $A \subseteq B$.

    \subsection{Properties of sets}
        \begin{enumerate}
            \item $A \subseteq B$. \textit{Proof}. Let $a \in A$, $A$ being the left side of $A \subseteq A$. It then follows that $a \in A$, being this $A$ the right side of $A \subseteq A$. Hence, $A \subseteq A$ by definition of subsets.
            \item $\emptyset \subseteq A$. \textit{Proof}. The following is a proof by contradiction. Suppose $\emptyset \not \subseteq A$. Then by definition of subsets it follows that there exists an element $x \in \emptyset$ and $x \not \in A$. By definition of empty set this statement cannot be true. Hence, $\emptyset \subseteq A$.
            \item If $A \subseteq B$ and $B \subseteq C$, then $A \subseteq C$. \textit{Proof}. Let $a \in A$. Because $A \subseteq B$, it follows that $a \in B$. Then because $B \subseteq C$, it follows that $a \in C$. Thus, we see that $a \in A$ implies $a \in C$, $A \subseteq C$.
            \item Let $A$ and $B$ be sets, $A = B$ if $A \subseteq B$ and $B \subseteq A$.
            \item Let $A$ and $B$ be sets, $A$ is a proper subset of $B$ if $A \subseteq B$ and $A \not = B$.
            \item Let $A$ be a set. The power set of $A$, denoted $\mathcal{P}(A)$ is the set defined by $\mathcal{P}(A) = \{X | X \subseteq A \}$
        \end{enumerate}

    \subsection{Set Operations}
        We can get new sets out of old ones. This can be achieved by performing certain operations in these sets, the most common operations with sets correspond to the \textit{or} and \textit{and} operators in logic.

        \begin{enumerate}
            \item $A \cup B$. Let $A$ and $B$ be sets. The union of $A$ and $B$, written as $A \cup B$, is the set with elements $x$ such that $x \in A$ or $x \in B$, that is $ \{ x | x \in A \vee x \in B \} $
            \item $A \cap B$. Let $A$ and $B$ be sets. The intersection of $A$ and $B$, written as $A \cap B$, is the set with elements $x$ such that $x \in A$ and $x \in B$, that is $ \{ x | x \in A \wedge x \in B \}$
        \end{enumerate}
        We can utilize Venn diagrams to visualize sets.

        \subsubsection{Properties of set operations}
            \begin{enumerate}
                \item $A \cap B \subseteq A$ and $A \cap B \subseteq B$. If $X$ is a set such that $X \subseteq A$ and $X \subseteq B$, then $X \subseteq A \cap B$.
                \item $A \subseteq A \cup B$ and $B \subseteq A \cup B$. If $X$ is a set such that $X \subseteq A$ and $X \subseteq B$, then $X \subseteq A \cap B$.
                \item $A \cup B = B \cup A$ and $A \cap B = B \cap A$. (\textit{Commutative Laws}).
            \end{enumerate}
\end{document}