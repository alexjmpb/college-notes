\documentclass{report}

\usepackage[most,many,breakable]{tcolorbox}
\usepackage{xcolor}

\definecolor{defBoxBorder}{HTML}{395144}
\newtcolorbox{defBox}{colback=white,colframe=defBoxBorder,arc=3pt, boxrule=0.5pt, drop fuzzy shadow, title=Definition}
\definecolor{thBoxBorder}{HTML}{AC8441}
\newtcolorbox{thBox}{colback=white,colframe=thBoxBorder,arc=3pt, boxrule=0.5pt, drop fuzzy shadow, title=Theorem}
\definecolor{noteBoxBorder}{HTML}{4E6C50}
\newtcolorbox{noteBox}{colback=white,colframe=noteBoxBorder,arc=3pt, boxrule=0.5pt, drop fuzzy shadow, title=Note}
\definecolor{axBoxBorder}{HTML}{AA5656}
\newtcolorbox{axBox}{colback=white,colframe=axBoxBorder,arc=3pt, boxrule=0.5pt, drop fuzzy shadow, title=Axiom/Postulate}
\definecolor{corBoxBorder}{HTML}{8B7E74}
\newtcolorbox{corBox}{colback=white,colframe=corBoxBorder,arc=3pt, boxrule=0.5pt, drop fuzzy shadow, title=Corollary}
\definecolor{lemBoxBorder}{HTML}{B99B6B}
\newtcolorbox{lemBox}{colback=white,colframe=lemBoxBorder,arc=3pt, boxrule=0.5pt, drop fuzzy shadow, title=Lemma}
\definecolor{asBoxColor}{HTML}{FDFDF9}
\definecolor{asBoxBorder}{HTML}{DEB881}
\newtcolorbox{asBox}{coltext=black, colback=asBoxColor,colframe=asBoxBorder,arc=3pt, boxrule=0.5pt, drop fuzzy shadow, title=Aside}

\input{setup.tex}

\begin{document}
    \coverPage{ Mathematics }{ Introduction to Calculus }{ Polynomials and Complex Numbers }{   }{ Alexander Mendoza }{\today}
    \tableofcontents

    \pagebreak
    \chapter{ Polynomials and Complex Numbers}
        \section{Real zeroes of polynomials}
            \begin{thBox}
                \textit{Rational Zeros Theorem}. If the polynomial $P(x) = a_nX^n + a_{n-1}X^{n-1} + \cdots + a_1X + a_0$ has integer coefficients, then every rational zero of $p$ is of the form $\frac{p}{q}$ where $p$ is a factor of the constant coefficient $a_0$ and $q$ is a factor of the leading coefficient $a_n$.
            \end{thBox}
            \begin{thBox}
                \textit{Remainder Theorem}. If the polynomial $P(x)$ is divided by $x - c$, then the reminder is the value $P(c)$.
            \end{thBox}
            \begin{thBox}
                \textit{Factor Theorem}. $c$ is a zero of $P$ if and only if $x-c$ is a factor of $P(x)$.
            \end{thBox}
\end{document}
