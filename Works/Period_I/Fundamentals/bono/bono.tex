\documentclass{report}

\usepackage[most,many,breakable]{tcolorbox}
\usepackage{xcolor}

\definecolor{defBoxBorder}{HTML}{395144}
\newtcolorbox{defBox}{colback=white,colframe=defBoxBorder,arc=3pt, boxrule=0.5pt, drop fuzzy shadow, title=Definition}
\definecolor{thBoxBorder}{HTML}{AC8441}
\newtcolorbox{thBox}{colback=white,colframe=thBoxBorder,arc=3pt, boxrule=0.5pt, drop fuzzy shadow, title=Theorem}
\definecolor{noteBoxBorder}{HTML}{4E6C50}
\newtcolorbox{noteBox}{colback=white,colframe=noteBoxBorder,arc=3pt, boxrule=0.5pt, drop fuzzy shadow, title=Note}
\definecolor{axBoxBorder}{HTML}{AA5656}
\newtcolorbox{axBox}{colback=white,colframe=axBoxBorder,arc=3pt, boxrule=0.5pt, drop fuzzy shadow, title=Axiom/Postulate}
\definecolor{corBoxBorder}{HTML}{8B7E74}
\newtcolorbox{corBox}{colback=white,colframe=corBoxBorder,arc=3pt, boxrule=0.5pt, drop fuzzy shadow, title=Corollary}
\definecolor{lemBoxBorder}{HTML}{B99B6B}
\newtcolorbox{lemBox}{colback=white,colframe=lemBoxBorder,arc=3pt, boxrule=0.5pt, drop fuzzy shadow, title=Lemma}
\definecolor{asBoxColor}{HTML}{FDFDF9}
\definecolor{asBoxBorder}{HTML}{DEB881}
\newtcolorbox{asBox}{coltext=black, colback=asBoxColor,colframe=asBoxBorder,arc=3pt, boxrule=0.5pt, drop fuzzy shadow, title=Aside}

\input{setup.tex}

\begin{document}
    \coverPage{ Matemáticas }{ Fundamentos Matemáticos }{ Bono 2 }{   }{ Alexander Mendoza }{\today}
    \tableofcontents
    \setlength{\parindent}{0pt}
    \pagebreak

    1. Una sucesión en un conjunto $A$ es una función $f: \mathbb{N} arrow A$, usualmente la imagen de $n$ a través de $f$ es denotada como $a_{n}$ y la sucesión como $(a_{n})_{n \in \mathbb{N}}$.

    Una sucesión $(a_{n})_{n \in \mathbb{N}}$ en un conjunto ordenado $(A, \preceq)$ se dice creciente si $n<m$ entonces $a_{n} \preceq a_{m}$, para cualquier par de números naturales $n \mathrm{y}$ $m$.

    Una sucesión $(a_{n})_{n \in \mathbb{N}}$ en un conjunto ordenado $(A, \preceq)$ se dice decreciente si $n<m$ entonces $a_{m} \preceq a_{n}$, para cualquier par de números naturales $n$ y $m$.

    Sea $\mathcal{A}$ una familia de conjuntos y $(A_{n})_{n \in \mathbb{N}}$ una sucesión de elementos en $\mathcal{A}$. Se definen el límite superior limite sup $A_n$ y limite inf $A_n$ como sigue:

    $$
    \lim \sup A_{n}=\bigcap_{n \in \mathbb{N}}(\bigcup_{m=n}^{\infty} A_{m}) \quad \lim \inf A_{n}=\bigcup_{n \in \mathbb{N}}(\bigcap_{m=n}^{\infty} A_{m}) .
    $$

    a. Demuestre que la sucesión $(B_{n})_{n \in \mathbb{N}}$ dada por $B_{n}=\bigcup_{m=n}^{\infty} A_{m}$ es decreciente respecto a la inclusión de conjuntos.

    b. Demuestre que la sucesión $(C_{n})_{n \in \mathbb{N}}$ dada por $C_{n}=\bigcap_{m=n}^{\infty} A_{m}$ es creciente respecto a la inclusión de conjuntos.

    c. Demuestre que lím inf $A_{n} \subseteq \lim \sup A_{n}$.\\\\
a. Para demostrar que la sucesión $(B_n){n \in \mathbb{N}}$ es decreciente, debemos mostrar que $B_{n+1} \subseteq B_n$ para todo $n \in \mathbb{N}$. Observemos que:
\begin{align*}
B_{n+1} &= \bigcup_{m=n+1}^{\infty} A_m \
&= \bigcup_{m=n}^{\infty} A_m \cup A_{n+1} \
&= B_n \cup A_{n+1}.
\end{align*}
Por lo tanto, $B_{n+1} \subseteq B_n$ ya que la unión de dos conjuntos contiene a cada uno de ellos. Concluimos que $(B_n)_{n \in \mathbb{N}}$ es decreciente respecto a la inclusión de conjuntos.

b. Para demostrar que la sucesión $(C_n){n \in \mathbb{N}}$ es creciente, debemos mostrar que $C_n \subseteq C{n+1}$ para todo $n \in \mathbb{N}$. Observemos que:
    \begin{align*}
    C_n &= \bigcap_{m=n}^{\infty} A_m \
    C_{n+1} &= \bigcap_{m=n+1}^{\infty} A_m \
    &= \bigcap_{m=n}^{\infty} A_m \cap A_{n+1} \
    &= C_n \cap A_{n+1}.
    \end{align*}
    Por lo tanto, $C_n \subseteq C_{n+1}$ ya que la intersección de dos conjuntos es un subconjunto de cada uno de ellos. Concluimos que $(C_n)_{n \in \mathbb{N}}$ es creciente respecto a la inclusión de conjuntos.

    c. $\lim \inf A_n \subseteq \lim \sup A_n$. Observemos que:
    \begin{align*}
    \limsup A_n &= \bigcap_{n \in \mathbb{N}}(\bigcup_{m=n}^{\infty} A_m) \
    &= \bigcap_{n \in \mathbb{N}} B_n,
    \end{align*}
    donde $B_n=\bigcup_{m=n}^{\infty} A_m$. Por lo tanto, para todo $n \in \mathbb{N}$ se tiene que $\limsup A_n \subseteq B_n$. En particular, $\limsup A_n \subseteq B_1$.

    Por otro lado, observemos que:
    \begin{align*}
    \liminf A_n &= \bigcup_{n \in \mathbb{N}}(\bigcap_{m=n}^{\infty} A_m) \
    &= \bigcup_{n \in \mathbb{N}} C_n,
    \end{align*}
    donde $C_n=\bigcap_{m=n}^{\infty} A_m$. Por lo tanto, existe $n_0 \in \mathbb{N}$ tal que $\liminf A_n \subseteq C_{n_0}$. Como $(C_n){n \in \mathbb{N}}$ es creciente, se tiene que $C{n_0} \subseteq C_n$ para todo $n$
    Por la definición de $\limsup A_n$, tenemos que para todo $n \in \mathbb{N}$:
    $$\bigcup_{m=n}^{\infty} A_m \supseteq \limsup A_n$$
    y por la definición de $\liminf A_n$, tenemos que para todo $n \in \mathbb{N}$:
    $$\bigcap_{m=n}^{\infty} A_m \subseteq \liminf A_n$$

    Tomando la intersección sobre todos los $n \in \mathbb{N}$ en la primera desigualdad y la unión sobre todos los $n \in \mathbb{N}$ en la segunda, obtenemos:
    \begin{align*}
    \bigcap_{n \in \mathbb{N}}(\bigcup_{m=n}^{\infty} A_m) &\subseteq \limsup A_n \\
    \liminf A_n &\subseteq \bigcup_{n \in \mathbb{N}}(\bigcap_{m=n}^{\infty} A_m)
    \end{align*}

    Por lo tanto, $\liminf A_n \subseteq \limsup A_n$, como queríamos demostrar.
\end{document}
