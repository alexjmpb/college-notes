\documentclass[11pt]{article}
\usepackage[english]{babel}
\usepackage[utf8x]{inputenc}
\usepackage{amsmath}
\usepackage{graphicx}
\usepackage[colorinlistoftodos]{todonotes}
\usepackage{enumitem}
\usepackage{listings}
\usepackage{verbatim}
\usepackage{eurosym}
\usepackage[export]{adjustbox}
\usepackage{amssymb}
\usepackage{bussproofs}
\usepackage{amsmath}
\usepackage{tikz}

%----------------------------------------------------------------------------------------
%	COVER START
%----------------------------------------------------------------------------------------
\begin{document}

    \begin{titlepage}

        \newcommand{\HRule}{\rule{\linewidth}{0.5mm}}
        \newcommand{\department}{Escuela de Matemáticas}
        \newcommand{\course}{Geometría Euclidiana}
        \newcommand{\titleValue}{Demostración del Teorema de Pasch}
        \newcommand{\subtitleValue}{}
        \newcommand{\authorName}{Alexander Mendoza}

        \center

        %----------------------------------------------------------------------------------------
        %	HEADER
        %----------------------------------------------------------------------------------------

        \includegraphics{images/logo_usa.png}
        \vspace{0.5cm}
        \textsc{\Large \department}\\[0.5cm]
        \textsc{\Large \course}\\[0.5cm]
        \vfill

        %----------------------------------------------------------------------------------------
        %	TITLE
        %----------------------------------------------------------------------------------------

        \HRule\\
        \Huge
        \textbf{\titleValue}\\[0.5cm]
        \Large
        \textbf{\subtitleValue}\\
        \HRule\\[0.5cm]

        %----------------------------------------------------------------------------------------
        %	AUTHOR AND DATE
        %----------------------------------------------------------------------------------------

        \vfill
        \Large
        \textit{\authorName}\\
        {\large \today}\\[2cm]

    \end{titlepage}
%----------------------------------------------------------------------------------------
%	COVER END
%----------------------------------------------------------------------------------------
\section*{Teorema de Pasch}
    Sean $A, B, C$ tres puntos no colineales, $D$ un punto de modo que $A-D-B$ y l una recta contenida en el plano determinado por $A, B, C$; de modo que $D \in l$ entonces, $\ell$ debe intersecar a $\overline{BC}$ o a $\overline{AC}$.\\

    \noindent Antes de pasar a la demostración definiremos algunos teoremas.\\

    \noindent \textit{Teorema 1}. Si $P$ y $Q$ están en lados opuestos de $\ell$ y $Q$ y $R$ en lados opuestos de $\ell$, entonces, $P$ y $R$ están del mismo lado de $\ell$.\\

    \noindent \textit{Teorema 2}. Si $P$ y $Q$ están en lados opuestos de $\ell$ y $Q$ y $R$ están del mismo lado de $\ell$, entonces, $P$ y $R$ están en lados opuestos de $\ell$.\\

    \noindent \textit{Demostración.} Sean $A, B, C$ tres puntos no colineales, $D$ un punto de modo que $A-D-B$ y $\ell$ una recta contenida en el plano $\alpha$ determinado por $A, B, C$; de modo que $D \in \ell$. Luego $D \in \overline{AB}$ esto por definición de interestancia, así $\ell \cap \overline{AB} \not = \emptyset$. Luego $A \in \mathcal{S}_{\ell, \neg B}$ por el postulado de separación del plano, así tenemos dos opciones para $C$, o $C \in \mathcal{S}_{\ell, A}$ o $C \in \mathcal{S}_{\ell, \neg A}$.\\

    \noindent Si $C \in \mathcal{S}_{\ell, A}$ entonces $C \in \mathcal{S}_{\ell, \neg B}$, esto por \textit{Teorema 2}. Luego $\ell \cap \overline{BC} \not = \emptyset$, esto lo sabemos por el postulado de separación del plano.\\

    \noindent De manera similar, si $C \in \mathcal{S}_{\ell, \neg A}$ entonces $C \in \mathcal{S}_{\ell, B}$, esto por \textit{Teorema 1}. Luego $\ell \cap \overline{AC} \not = \emptyset$ por el postulado de separación del plano.\\

    \noindent Nótese que en caso de que $C \in \ell$ o de que $\ell = \overleftrightarrow{AB}$ se sobreentiende que $\ell \cap \overline{BC} \not = \emptyset$ y $\ell \cap \overline{AC} \not = \emptyset$. De esta manera demostramos que en cualquier caso $\ell$ interseca a $\overline{BC}$ o $\overline{AC}$.
\end{document}