\documentclass{report}
\usepackage[spanish]{babel}



\input{setup.tex}

\begin{document}
    \coverPage{ Matemáticas }{ Cálculo Vectorial }{ Tarea 1 }{  }{ Alexander Mendoza }{\today}

    \pagebreak
    \section*{Sección 8.3}
    \subsection*{Ejercicio 1b}
    Dado un campo escalar $f$ definido en un conjunto $S$ y un número real $c$, el conjunto de puntos $\boldsymbol{x}$ en $S$ donde $f(x)=c$ se denomina conjunto de nivel de $f$. Aquí, $S$ es el espacio $\mathbf{R}^{n}$. Para los campos escalares siguientes, dibuje los conjuntos de nivel para los valores dados de $c$.
    b) $f(x, y)=e^{xy}$
    $c=e^{-2}, e^{-1}, 1, e^{1}, e^{2}, e^{3}$

    \textbf{Respuesta}

    El conjunto de nivel viene dado por:
    \[
    f(x, y) = c \implies e^{xy} = c.
    \]
    Aplicando logaritmo natural:
    \[
    xy = \ln(c).
    \]
    Así, los conjuntos de nivel se expresan mediante:
    \[
    xy = \ln(c),
    \]
    que son hipérbolas en el plano $xy$ para diferentes valores de $c$.

    \subsection*{Análisis para cada valor de $c$}

    \begin{itemize}
        \item Si $c = e^{-2}$, obtenemos $\ln(e^{-2}) = -2$, y el conjunto de nivel es:
        \[
        xy = -2.
        \]
        Esta hipérbola tiene ramas en los cuadrantes II y IV.
        
        \item Si $c = e^{-1}$, obtenemos $\ln(e^{-1}) = -1$, y el conjunto de nivel es:
        \[
        xy = -1.
        \]
        Esta hipérbola se encuentra más cerca del origen.
        
        \item Si $c = 1$, obtenemos $\ln(1) = 0$, y el conjunto de nivel es:
        \[
        xy = 0.
        \]
        Esto corresponde a las rectas $x = 0$ y $y = 0$, es decir, los ejes coordenados.
        
        \item Si $c = e^{1}$, obtenemos $\ln(e^1) = 1$, y el conjunto de nivel es:
        \[
        xy = 1.
        \]
        Esta hipérbola tiene ramas en los cuadrantes I y III.
        
        \item Si $c = e^{2}$, obtenemos $\ln(e^2) = 2$, y el conjunto de nivel es:
        \[
        xy = 2.
        \]
        Esta hipérbola se encuentra más lejos del origen que las anteriores.
        
        \item Si $c = e^{3}$, obtenemos $\ln(e^3) = 3$, y el conjunto de nivel es:
        \[
        xy = 3.
        \]
        Esta es la hipérbola más alejada del origen de todas.
    \end{itemize}
    \subsection*{Ejercicio 1e}
    Sean $f$ un campo escalar definido en un conjunto $S$ y $c$ un número real dado. El conjunto de todos los puntos $\boldsymbol{x}$ de $S$ tales que $f(x)=c$ se llama conjunto de nivel de $f$. Para cada uno de los campos escalares siguientes, $S$ es todo el espacio $\mathbf{R}^{n}$. Representar gráficamente los conjuntos de nivel correspondientes a los valores dados de $c$.\\
    e) $f(x, y, z)=x^{2}+2 y^{2}+3 z^{2}$\\
    $c = 0, 6, 12$

    \textbf{Respuesta}
    El conjunto de nivel está dado por la ecuación:
    \[
    x^2 + 2y^2 + 3z^2 = c.
    \]

    \subsection*{Análisis de los valores de $c$}

    \begin{itemize}
        \item Para $c = 0$, la ecuación es:
        \[
        x^2 + 2y^2 + 3z^2 = 0,
        \]
        cuya única solución es $x = 0$, $y = 0$ y $z = 0$. El conjunto de nivel es el punto $(0, 0, 0)$.

        \item Para $c = 6$, la ecuación es:
        \[
        x^2 + 2y^2 + 3z^2 = 6,
        \]
        que representa un elipsoide centrado en el origen, con distintos radios en los ejes $x$, $y$ y $z$.

        \item Para $c = 12$, la ecuación es:
        \[
        x^2 + 2y^2 + 3z^2 = 12,
        \]
        que también es un elipsoide centrado en el origen, pero de mayor tamaño que el anterior.
    \end{itemize}

    Representación gráfica de los elipsoides para los valores $c = 6$ y $c = 12$, para $c=0$ es el punto $(0,0,0)$.
    \begin{figure}[h]
    % \includegraphics[width=8cm]{imagen_2024-09-05_175639313.png}
    \centering
    \end{figure}

    \newpage\subsection*{Ejercicio 2c}
    Dado el conjunto $S$ de todos los puntos $(x, y)$ en el plano que cumplen las siguientes desigualdades, grafique $S$ y explique geométricamente si es un conjunto abierto. Indique en el gráfico la frontera de $S$.\\
    c) $|x|<1$ y $|y|<1$

    \textbf{Respuesta}

    El conjunto $S$ se define por las desigualdades:
    \[
    |x| < 1 \quad \text{y} \quad |y| < 1,
    \]
    lo que corresponde a un rectángulo abierto en el plano donde \( x \) y \( y \) están en el intervalo \( (-1, 1) \). Este conjunto es \textbf{abierto} ya que no incluye los puntos en su frontera.
    \subsection*{Frontera y conjunto abierto}

    $S$ es un conjunto abierto porque su frontera no está incluida. La frontera está formada por las líneas \( x = \pm 1 \) y \( y = \pm 1 \), como se muestra en el gráfico con borde sólido. El área sombreada representa el conjunto $S$, excluyendo sus bordes.\subsection*{Ejercicio 2}
    Determinar el conjunto \( S \) de todos los puntos \( (x, y) \) en el plano que cumplen con las desigualdades proporcionadas. Graficar el conjunto \( S \) y analizar si geométricamente \( S \) es un conjunto abierto o no. Además, identificar la frontera de \( S \) en el gráfico.\\
    Desigualdades: \( y > x^{2} \quad \text{y} \quad |x|<2 \)

    \textbf{Respuesta}

    Consideremos el conjunto \( S \) que cumple las siguientes condiciones:

    \[
    y > x^2 \quad \text{y} \quad |x| < 2
    \]

    Esto indica que \( S \) es la región del plano donde los puntos \( (x, y) \) están localizados por arriba de la parábola \( y = x^2 \) y dentro del intervalo \( -2 < x < 2 \).

    \subsection*{Frontera y Conjunto Abierto}

    El conjunto \( S \) consiste en:
    \begin{itemize}
        \item Los puntos encima de la parábola \( y = x^2 \), es decir, \( y > x^2 \).
        \item Aquellos dentro del intervalo \( -2 < x < 2 \) en el eje \( x \).
    \end{itemize}

    \textbf{Frontera del conjunto}: La frontera de \( S \) está formada por la parábola \( y = x^2 \) sobre el intervalo \( -2 \leq x \leq 2 \), pero la parábola no es parte de \( S \). El conjunto \( S \) es abierto porque no incluye los puntos que pertenecen a su frontera (la parábola \( y = x^2 \)).
    
    \section*{Sección 8.9}
    \subsection*{Ejercicio 1}
    Dado un campo escalar $f$ definido en $\mathbf{R}^{*}$ como $f(\boldsymbol{x})=\boldsymbol{a} \cdot \boldsymbol{x}$, donde $\boldsymbol{a}$ es un vector constante, determina $f^{\prime}(\boldsymbol{x}; \boldsymbol{y})$ para cualquier $\boldsymbol{x}$ y $\boldsymbol{y}$.

    \textbf{Respuesta}

    $$
    \begin{aligned}
    f^{\prime}(\boldsymbol{x}, \boldsymbol{y}) &= \lim_{h \to 0} \frac{f(\boldsymbol{x} + h \boldsymbol{y}) - f(\boldsymbol{x})}{h} \\
    &= \lim_{h \to 0} \frac{f(\boldsymbol{x}) + f(h \boldsymbol{y}) - f(\boldsymbol{x})}{h} \\
    &= \lim_{h \to 0} \frac{f(h \boldsymbol{y})}{h} = \lim_{h \to 0} \frac{\boldsymbol{a} \cdot h \boldsymbol{y}}{h} = \boldsymbol{a} \cdot \boldsymbol{y}
    \end{aligned}
    $$
    \subsection*{Ejercicio 2c}
    Considerando $n=3$ en el ejercicio 2a, encuentre todos los puntos $(\boldsymbol{x}, \boldsymbol{y}, z)$ para los cuales se cumple $f^{\prime}(\boldsymbol{i}+2 \boldsymbol{j}+\mathbf{3 k}$; $x \boldsymbol{i}+y \boldsymbol{j}+z \boldsymbol{k})=0$.

    \textbf{Respuesta}

    \[
    \begin{aligned}
    & \text { Para hallar las derivadas parciales } \\
    & \text { usando } F(x)=\| \boldsymbol{x}\|^{4}, \\
    & f'(\boldsymbol{x}, \boldsymbol{y})=\lim _{h \to 0} \frac{f(\boldsymbol{x}+h\boldsymbol{y})-f(\boldsymbol{x})}{h}=\lim _{h \to 0} \frac{\| \boldsymbol{x}+h \boldsymbol{y}\|^{4}-\| \boldsymbol{x}\|^{4}}{h}, \\
    & \text { resolviendo tenemos:}\\
    & \lim _{h \to 0} \frac{\left(\| \boldsymbol{x}\|^{2}+2h(\boldsymbol{x} \cdot \boldsymbol{y})+h^{2}\|\boldsymbol{y}\|^{2}\right)^{2} - \| \boldsymbol{x}\|^{4}}{h}, \\
    & \lim _{h \to 0} \frac{\left[ \| \boldsymbol{x}\|^{4} + 4h (\boldsymbol{x} \cdot \boldsymbol{y})\|\boldsymbol{x}\|^{2} + 4h^2(\boldsymbol{x} \cdot \boldsymbol{y})^{2} + 4h^2(\boldsymbol{x} \cdot \boldsymbol{y})\|\boldsymbol{y}\|^{2} + h^4 \|\boldsymbol{y}\|^{4} \right] - \| \boldsymbol{x}\|^{4}}{h}, \\
    & =4 (\boldsymbol{x} \cdot \boldsymbol{y}) \| \boldsymbol{x}\|^{2}.
    \end{aligned}
    \]

    Para $f^{\prime}(\boldsymbol{i}+2 \boldsymbol{j}+\mathbf{3 k}; x \boldsymbol{i}+y \boldsymbol{j}+z \boldsymbol{k})=0$, obtenemos:
    \[
    4(\boldsymbol{x} \cdot \boldsymbol{y}) \|\boldsymbol{x}\|^{2}=0,
    \]
    donde, al sustituir $\| \boldsymbol{x}\|^{2} = \boldsymbol{x} \cdot \boldsymbol{x}$ con $\boldsymbol{x} = \boldsymbol{i} + 2\boldsymbol{j} + 3\boldsymbol{k}$, obtenemos:
    \[
    (\boldsymbol{i} + 2\boldsymbol{j} + 3\boldsymbol{k}) \cdot (\boldsymbol{i} + 2\boldsymbol{j} + 3\boldsymbol{k}) = 1^2 + 2^2 + 3^2 = 14.
    \]
    Y,
    \[
    (\boldsymbol{x} \cdot \boldsymbol{y}) = (\boldsymbol{i} + 2\boldsymbol{j} + 3\boldsymbol{k}) \cdot (x\boldsymbol{i} + y\boldsymbol{j} + z\boldsymbol{k}) = x + 2y + 3z.
    \]
    Reemplazando tenemos:
    \[
    4(x + 2y + 3z) \cdot 14 = 0,
    \]
    lo que se cancela y resulta en:
    \[
    x + 2y + 3z = 0.
    \]

    Entonces, los puntos $(x, y, z)$ que satisfacen $x + 2y + 3z = 0$.\subsection*{Ejercicio 3}
    Dada una transformación lineal $T: \mathbf{R}^{n} \rightarrow \mathbf{R}^{n}$, encuentra la derivada $f^{\prime}(\boldsymbol{x} ; \boldsymbol{y})$ del campo escalar en $\mathbf{R}^{n}$ definido por la función $f(\boldsymbol{x})=\boldsymbol{x} \cdot \boldsymbol{T}(\boldsymbol{x})$.
    \textbf{Respuesta}
    $$
    \begin{aligned}
    & f^{\prime}(\boldsymbol{x} ; \boldsymbol{y})=\lim_{h \rightarrow 0} \frac{f(\boldsymbol{x}+h \boldsymbol{y})-f(\boldsymbol{x})}{h} \\
    & \lim _{h \rightarrow 0} \frac{(\boldsymbol{x}+h \boldsymbol{y}) \cdot \boldsymbol{T}(\boldsymbol{x}+h \boldsymbol{y})-\boldsymbol{x} \cdot \boldsymbol{T}(\boldsymbol{x})}{h} \\
    & \lim _{h \rightarrow 0} \frac{(\boldsymbol{x}+h \boldsymbol{y}) \cdot (\boldsymbol{T}(\boldsymbol{x})+h \boldsymbol{T}(\boldsymbol{y}))-\boldsymbol{x} \cdot \boldsymbol{T}(\boldsymbol{x})}{h} \\
    & \lim _{h \rightarrow 0} \frac{\boldsymbol{x} \cdot \boldsymbol{T}(\boldsymbol{x}) + h \boldsymbol{x} \cdot \boldsymbol{T}(\boldsymbol{y}) + h \boldsymbol{y} \cdot \boldsymbol{T}(\boldsymbol{x}) + h^2 \boldsymbol{y} \cdot \boldsymbol{T}(\boldsymbol{y}) - \boldsymbol{x} \cdot \boldsymbol{T}(\boldsymbol{x})}{h} \\
    & \lim _{h \rightarrow 0} \frac{h (\boldsymbol{x} \cdot \boldsymbol{T}(\boldsymbol{y})+\boldsymbol{y} \cdot \boldsymbol{T}(\boldsymbol{x}) + h \boldsymbol{y} \cdot \boldsymbol{T}(\boldsymbol{y}))}{h} \\
    & \lim _{h \rightarrow 0} (\boldsymbol{x} \cdot \boldsymbol{T}(\boldsymbol{y})+\boldsymbol{y} \cdot \boldsymbol{T}(\boldsymbol{x})+h \boldsymbol{y} \cdot \boldsymbol{T}(\boldsymbol{y})) \\
    & =\boldsymbol{x} \cdot \boldsymbol{T}(\boldsymbol{y})+\boldsymbol{y} \cdot \boldsymbol{T}(\boldsymbol{x})
    \end{aligned}
    $$\subsection*{Ejercicio 8}
    Determina cada una de las derivadas parciales de primer orden del campo escalar en los ejercicios 4 a 9. En los ejercicios 8 y 9, los campos están definidos en $\mathbf{R}^{n}$.\\
    8. $f(x) = a \cdot x$, donde $a$ es constante.
    \\
    \textbf{Respuesta}
    Dado que $a$ es un vector constante, entonces
    $$
    a \cdot x = (a_1, \ldots, a_n) \cdot (x_1, \ldots, x_n)
    $$
    Realizando el producto punto obtenemos:
    $$
    a \cdot x = \sum_{i=1}^n a_i x_i
    $$
    A continuación, calculamos la derivada parcial. Como $a$ es un vector constante, actúa como un escalar:
    $$
    \frac{\partial f}{\partial x_k} = \sum_{i=1}^n a_i \frac{\partial x_i}{\partial x_k} = \sum_{i=1}^n a_i \delta_{ik} = \sum_{i=1}^n a_i d_{ik} = a_k
    $$
    Por lo tanto, $D_k f(\vec{x}) = a_k$. En términos generales, sería $Df(\vec{x}) = a$.\subsection*{Ejercicio 9}
    Para cada uno de los ejercicios del 4 al 9, encuentra todas las derivadas parciales de primer orden del campo escalar dado. Los campos en los ejercicios 8 y 9 están definidos en $\mathbf{R}^{n}$.
    Sea $f(x)=\sum_{i=1}^{n} \sum_{j=1}^{n} a_{i j} x_{i} x_{j}$, donde $a_{i j}=a_{j i}$.

    \textbf{Respuesta}
    Dado que $a_{ij}$ es una matriz simétrica:
    \[
    \begin{aligned}
    & \text{Vamos a resolver por casos.} \\
    & \text{Caso 1:} \quad i = j = k \\
    & \frac{\partial f}{\partial x_k} = \sum_{j=1}^n \sum_{i=1}^n \frac{\partial}{\partial x_k} \left(a_{ij} x_i x_j\right) \\
    & = \sum_{i=1}^n \sum_{j=1}^n a_{ij} \frac{\partial}{\partial x_k} \left(x_i x_j\right) \\
    & = \sum_{i=1}^n \sum_{j=1}^n a_{ij} \left(\delta_{ik} x_j + \delta_{jk} x_i\right) \\
    & = \sum_{i=1}^n a_{ik} x_k + \sum_{j=1}^n a_{kj} x_j \\
    & = 2 \sum_{k=1}^n a_{kk} x_k \quad \text{(dado que } a_{ij} \text{ es simétrica)}
    \end{aligned}
    \]

    \[
    \text{Caso 2:} \quad i = j \quad y \quad i \neq k
    \]

    \[
    \frac{\partial f}{\partial x_k} = \sum_{i=1}^n \sum_{j=1}^n a_{ij} x_i x_j
    = \sum_{i=1}^n a_{jk} x_j \quad \text{para } j \neq k
    \]

    \[
    \text{Caso 3:} \quad i = k \quad y \quad i \neq j
    \]

    \[
    \frac{\partial f}{\partial x_k} = \sum_{i=1}^n \sum_{j=1}^n a_{ij} x_i x_j = \sum_{i=1}^n a_{ij} x_i
    \]

    Al combinar las derivadas parciales:
    \[
    \sum_{i=1}^n a_{ji} x_j + \sum_{i=1}^n a_{ij} x_i \quad \text{donde} \quad a_{ji}=a_{ij}
    \]

    Al cambiar de variable $y=1$ se obtiene:
    \[
    \sum_{j=1}^n a_{jj} x_j + \sum_{j=1}^n a_{jj} x_j = 2 \sum_{j=1}^n a_{jj} x_j
    \]\subsection*{Ejercicio 13}
    Calcular todas las derivadas parciales de primer orden de \( f(x, y)=\tan \left(\frac{x^{2}}{y}\right), \quad y \neq 0 \).
    \textbf{Respuesta}
    $$
    \begin{aligned}
    & \text{Definimos } f(x, y)=\tan \left(\frac{x^2}{y}\right). \\
    & D_1 \text{ es la derivada parcial con respecto a } x: \\
    & D_1 f(x, y) = \sec^2\left(\frac{x^2}{y}\right) \cdot \frac{2x}{y}, \\
    & D_2 \text{ es la derivada parcial con respecto a } y: \\
    & D_2 f(x, y) = -\sec^2\left(\frac{x^2}{y}\right) \cdot \frac{x^2}{y^2}. \\
    & \text{Calculemos las derivadas parciales mixtas:} \\
    & D_2\left(D_1 f\right) = D_2\left(\sec^2\left(\frac{x^2}{y}\right) \cdot \frac{2x}{y}\right) 
    = \frac{\sec^2\left(\frac{x^2}{y}\right) \cdot (-2x)}{y^2} - \frac{4x \sec^3\left(\frac{x^2}{y}\right) \cdot \tan\left(\frac{x^2}{y}\right)}{y^3}, \\
    & D_1\left(D_2 f\right) = D_1\left(-\sec^2\left(\frac{x^2}{y}\right) \cdot \frac{x^2}{y^2}\right) 
    = \frac{-\sec^2\left(\frac{x^2}{y}\right) \cdot 2x}{y^2} - \frac{2 \sec^3\left(\frac{x^2}{y}\right) \cdot \tan\left(\frac{x^2}{y}\right) \cdot 2x}{y^3}. \\
    & \text{Notamos que: } D_1\left(D_2 f\right) = D_2\left(D_1 f\right).
    \end{aligned}
    $$
    \subsection*{Ejercicio 15}
    En los ejercicios del 10 al 17, calcular todas las derivadas parciales de primer orden. Para los ejercicios 10, 11 y 12, verificar que las derivadas parciales mixtas $D_{1}\left(D_{2} f\right)$ y $D_{2}\left(D_{1} f\right)$ son iguales.
    Dado $f(x, y) = \arctan \frac{x + y}{1 - xy}, \quad xy \neq 1$.

    \textbf{Respuesta}

    Primero calculamos $D_{1}$:
    \[
    \begin{aligned}
    D_{1} &= \frac{1}{1+\left(\frac{x+y}{1-xy}\right)^{2}} \cdot \frac{1 - xy + y(x+y)}{(1 - xy)^{2}} \\
    &= \frac{(1-xy)^{2}}{(1 - xy)^{2} + (x+y)^{2}} \cdot \frac{1 - xy + xy + y^{2}}{(1-xy)^{2}} \\
    &= \frac{1 + y^{2}}{(1 - xy)^{2} + (x + y)^{2}} \\
    &= \frac{1 + y^{2}}{1 + x^{2} + y^{2} + (xy)^{2}}
    \end{aligned}
    \]

    Luego, calculamos $D_{2}\left(D_{1}\right)$:

    \[
    \begin{aligned}
    D_{2}\left(D_{1}\right) &= D_{2}\left(\frac{1 + y^{2}}{1 + x^{2} + y^{2} + x^{2} y^{2}}\right), \quad \text{donde} \quad f = 1 + y^{2} \quad \text{y} \quad g = 1 + x^{2} + y^{2} + x^{2} y^{2} \\
    f^{\prime} &= 2y \quad \text{y} \quad g^{\prime} = 2y + 2x^{2} y \\
    D_{2} &= \frac{g \cdot f' - f \cdot g'}{g^{2}} \\
    &= \frac{\left(1 + x^{2} + y^{2} + x^{2} y^{2}\right) \cdot 2y - \left(1 + y^{2}\right) \cdot \left(2y + 2x^{2} y\right)}{\left(1 + x^{2} + y^{2} + x^{2} y^{2}\right)^{2}} \\
    &= \frac{2y + 2y x^{2} + 2y^{3} + 2y^{3} x^{2} - \left(2y + 2x^{2} y + 2y^{3} + 2x^{2} y^{3}\right)}{\left(1 + x^{2} + y^{2} + (xy)^{2}\right)^{2}} \\
    &= 2y + 2y x^{2} + 2y^{3} + 2y^{3} x^{2} - 2y - 2x^{2} y - 2y^{3} - 2x^{2} y^{3} \\
    &= 0 \\
    &= \frac{0}{\left(1 + x^{2} + y^{2} + (xy)^{2}\right)^{2}} = 0
    \end{aligned}
    \]

    Luego, calculamos $D_{2}$:
    \[
    \begin{aligned}
    D_{2} &= \arctan \left(\frac{x + y}{1 - xy}\right) \\
    &= \frac{1}{1 + \frac{(x + y)^{2}}{(1 - xy)^{2}}} \cdot \left( \frac{(1 - xy) \cdot (x + y)^{3} - (1 - xy)' \cdot (x + y)}{(1 - xy)^{2}} \right) \\
    &= \frac{(1 - xy)^{2}}{(1 - xy)^{2} + (x + y)^{2}} \cdot \frac{(1 - xy)(1) + (x + y)(+x)}{(1 - xy)^{2}} \\
    &= \frac{1 - xy + x^{2} + xy}{1 - 2xy + (xy)^{2} + x^{2} + 2xy + y^{2}} \\
    &= \frac{1 + x^{2}}{1 + (xy)^{2} + x^{2} + y^{2}}
    \end{aligned}
    \]

    Finalmente, calculamos $D_{1}\left(D_{2}\right)$:
    \[
    \begin{aligned}
    D_{1}\left(D_{2}\right) &= D_{1}\left(\frac{1 + x^{2}}{1 + (xy)^{2} + x^{2} + y^{2}}\right) \\
    &f' = 2x \quad \text{donde} \quad  f = 1 + x^{2} \quad \text{y} \quad g = 1 + (xy)^{2} + x^{2} + y^{2} \\
    g' &= 2x + 2xy^{2} \\
    &= \frac{\left(1 + (xy)^{2} + x^{2} + y^{2}\right) \cdot 2x - \left(1 + x^{2}\right) \left(2x + 2xy^{2}\right)}{g^{2}} \\
    &= \frac{2x \left(1 + (xy)^{2} + x^{2} + y^{2}\right) - 2x - 2xy^{2} - 2x^{3} - 2x^{3} y^{2}}{g^{2}} \\
    &= \frac{0}{g^{2}} = 0
    \end{aligned}
    \]

    Se puede notar que $D_1(D_2) = D_2(D_1)$.

    \newpage\subsection*{Ejercicio 18}
    Sea $v(r, t)=t^{n} e^{-r^{2} /(4 t)}$. Determine un valor de $n$ para que $v$ satisfaga la siguiente ecuación:
    $$
    \frac{\partial v}{\partial t}=\frac{1}{r^{2}} \frac{\partial}{\partial r}\left(r^{2} \frac{\partial v}{\partial r}\right)
    $$
    \textbf{Respuesta}
    Primero, evaluamos la derivada de $v$ respecto a $t$:
    $$
    \begin{aligned}
    \frac{\partial v}{\partial t} &= n t^{n-1} e^{-r^{2}/(4t)} + t^n \left(-e^{-r^{2}/(4t)} \cdot \frac{r^2}{4t^2}\right) \\
    &= n t^{n-1} e^{-r^{2}/(4t)} - \frac{r^2}{4t^2} t^n e^{-r^{2}/(4t)} \\
    &= e^{-r^{2}/(4t)} \left( n t^{n-1} - \frac{r^2}{4t^2} t^n \right) \\
    &= e^{-r^{2}/(4t)} \left( n t^{n-1} - \frac{r^2}{4t^2} t^n \right)
    \end{aligned}
    $$
    A continuación, calculamos la derivada de $v$ respecto a $r$:
    $$
    \frac{\partial v}{\partial r} = \left( - \frac{r}{2t} \right) t^n e^{ - r^2 / (4t)} = - \frac{r t^{n-1}}{2} e^{-r^2 / (4t)}
    $$
    Ahora multiplicamos por $r^2$ y derivamos de nuevo respecto a $r$:
    $$
    \begin{aligned}
    \frac{\partial}{\partial r} \left( r^2 \frac{\partial v}{\partial r} \right) &= \frac{\partial}{\partial r} \left( - \frac{r^3 t^{n-1}}{2} e^{-r^2 / (4t)} \right) \\
    &= - \frac{3r^2 t^{n-1}}{2} e^{-r^2 / (4t)} + \frac{r^4 t^{n-2}}{4t} e^{-r^2 / (4t)} \\
    &= e^{-r^2 / (4t)} \left( - \frac{3r^2 t^{n-1}}{2} + \frac{r^4 t^{n-2}}{4t} \right)
    \end{aligned}
    $$
    Igualamos las expresiones obtenidas:
    $$
    e^{-r^2 / (4t)} \left( n t^{n-1} - \frac{r^2}{4t^2} t^n \right) = e^{-r^2 / (4t)} \left( - \frac{3r^2 t^{n-1}}{2} + \frac{r^4 t^{n-2}}{4t} \right)
    $$
    Simplificando términos:
    $$
    \begin{aligned}
    n t^{n-1} - \frac{r^2 t^{n-2}}{4} &= - \frac{3r^2 t^{n-1}}{2} + \frac{r^4 t^{n-2}}{4t} \\
    n t^{n-1} &= - \frac{3}{2} t^{n-1} \\
    n &= -\frac{3}{2}
    \end{aligned}
    $$\subsection*{Ejercicio 19}
    Dado $z=u(x, y) e^{a x+b y}$ y $\left.\partial^{2} u / \partial x \partial y\right)=0$. Encuentra los valores de las constantes $a$ y $b$ tales que
    $$
    \frac{\partial^{2} z}{\partial x \partial y}-\frac{\partial z}{\partial x}-\frac{\partial z}{\partial y}+z=0
    $$

    \textbf{Respuesta}
    1) Calcular la primera derivada parcial de $z$ con respecto a $x$:
    $$
    \frac{\partial z}{\partial x} = \frac{\partial u}{\partial x} \cdot e^{a x + b y} + u(x, y) \cdot e^{a x + b y} \cdot a
    $$

    2) Calcular la primera derivada parcial de $z$ con respecto a $y$:
    $$
    \frac{\partial z}{\partial y} = \frac{\partial u}{\partial y} \cdot e^{a x + b y} + u(x, y) \cdot e^{a x + b y} \cdot b
    $$

    3) Calcular la segunda derivada parcial mixta de $z$:
    $$
    \frac{\partial^{2} z}{\partial x \partial y} = \frac{\partial}{\partial y} \left( \frac{\partial z}{\partial x} \right)
    $$

    Desarrollando el término anterior:
    $$
    \frac{\partial^{2} z}{\partial x \partial y} = \frac{\partial}{\partial y} \left( \frac{\partial u}{\partial x} \cdot e^{a x + b y} + u(x, y) \cdot e^{a x + b y} \cdot a \right)
    $$

    Lo cual nos lleva a:
    $$
    \frac{\partial^{2} z}{\partial x \partial y} = \frac{\partial^{2} u}{\partial x \partial y} \cdot e^{a x + b y} + \frac{\partial u}{\partial x} \cdot e^{a x + b y} \cdot b + \frac{\partial u}{\partial y} \cdot e^{a x + b y} \cdot a + u(x, y) \cdot a \cdot b \cdot e^{a x + b y}
    $$

    Como  $\left.\frac{\partial^{2} u}{\partial x \partial y}=0\right)$, la ecuación se simplifica a:
    $$
    e^{a x + b y} \left( a u_y + b u_x + ab u - u_x - u_y + u \right) = 0
    $$

    Dado que $e^{a x + b y} \neq 0$, se debe cumplir:
    $$
    a u_y + b u_x + ab u - u_x - u_y + u = 0
    $$

    Reordenando los términos, obtenemos:
    $$
    u_y (a - 1) + u_x (b - 1) + u (ab - a - b + 1) = 0
    $$

    Para que la ecuación se mantenga siempre igual a cero:
    $$
    a = 1 \quad \text{y} \quad b = 1
    $$

    Así obtenemos que los valores de las constantes son $a=1$ y $b=1$.
    \section*{Sección 8.14}
    \subsection*{Ejercicio 2a}
    Determine las derivadas direccionales de los campos escalares siguientes en los puntos y direcciones dadas:
    a) $f(x, y, z) = x^2 + 2y^2 + 3z^2$ en $(1,1,0)$ en la dirección de $\boldsymbol{i} - \boldsymbol{j} + 2 \boldsymbol{k}$.
    \textbf{Respuesta}
    Para el campo escalar dado, el vector dirección unitario es

    $\hat{n} = \frac{i - j + 2k}{\sqrt{1^2 + (-1)^2 + 2^2}} \to  \hat{n}=\frac{i - j + 2k}{\sqrt{6}}$

    La derivada direccional de $f$ en el punto $(x, y, z)$ y en la dirección de $\hat{n}$ se calcula como:

    $D_{\hat{n}} f(x, y, z) = \nabla f \cdot \hat{n} = (2xi + 4yj + 6zk) \cdot \frac{i - j + 2k}{\sqrt{6}}$

    Entonces,

    $D_{\hat{n}} f(1, 1, 0) = \frac{2x - 4y + 12z}{\sqrt{6}} = \frac{-2}{\sqrt{6}} \quad \text{en} \quad (1, 1, 0)$\subsection*{Ejercicio 4}
    Un campo escalar diferenciable $f$ presenta las derivadas direccionales $+2$ en la dirección del punto $(2,2)$ y $-2$ en la dirección del punto $(1,1)$ evaluadas en $(1,2)$. Encuentre el vector gradiente en $(1,2)$ y calcule la derivada direccional en la dirección del punto $(4,6)$.

    \textbf{Respuesta}

    Dados los puntos $P(1,2)$, $Q(2,2)$, $R(1,1)$ y $S(4,6)$, construimos los vectores unitarios:

    $$\vec{PQ}=(2-1)i+(2-2)j=i$$

    $$\vec{PR}=(1-1)i+(1-2)j=-j$$

    $$\vec{PS}=\frac{(4-1)i+(6-2)j}{\sqrt{3^2+4^2}}=\frac{3i+4j}{5}$$

    Para las derivadas direccionales obtenemos:

    $$D_{\vec{PQ}}f(1,2)=\nabla f \cdot \vec{PQ}=\nabla f \cdot i=2 \Rightarrow \frac{\partial f(1,2)}{\partial x}=2$$

    $$D_{\vec{PR}}f(1,2)=\nabla f \cdot \vec{PR}=\nabla f \cdot (-j)=-2 \Rightarrow \frac{\partial f(1,2)}{\partial y}=2$$

    El vector gradiente en el punto $(1,2)$ es:

    $$\nabla f(1,2)=\frac{\partial f(1,2)}{\partial x}i+\frac{\partial f(1,2)}{\partial y}j$$

    Sustituyendo los valores obtenidos:

    $$\nabla f(1,2)=2i+2j$$

    Entonces, la derivada direccional en la dirección del punto $(4,6)$ es:

    $$D_{\vec{PS}}f(1,2)=(2i+2j) \cdot \left(\frac{3}{5}i+\frac{4}{5}j\right)=\frac{6}{5}+\frac{8}{5}=\frac{14}{5}$$\subsection*{Ejercicio 5}
    Determinar los valores de las constantes $a$, $b$ y $c$ de manera que la derivada direccional de $f(x, y, z)=a x y^{2}+b y z+c z^{2} x^{3}$ en el punto $(1,2,-1)$ alcance el máximo valor de 64 en la dirección del eje $z$.
    \textbf{Respuesta}
    Para encontrar $a$, $b$ y $c$, debemos calcular la derivada direccional de la función 
    \[
    f(x, y, z) = a x y^2 + b y z + c z^2 x^3 
    \]
    en el punto \( (1, 2, -1) \) y establecer que esta derivada tenga un valor máximo de 64 en la dirección del eje \(z\).

    El gradiente de \(f(x,y,z)\) se define como:
    \[
    \nabla f(x, y, z) = \left( \frac{\partial f}{\partial x}, \frac{\partial f}{\partial y}, \frac{\partial f}{\partial z} \right)
    \]
    Calculamos las derivadas parciales:
    \[
    \frac{\partial f}{\partial x} = a y^2 + 3c z^2 x^2, \quad \frac{\partial f}{\partial y} = 2a x y + b z, \quad \frac{\partial f}{\partial z} = b y + 2c z x^3
    \]

    Luego, la derivada direccional en la dirección \(z\) corresponde a:
    \[
    D_{\mathbf{u}} f(x, y, z) = \frac{\partial f}{\partial z}
    \]

    Evaluamos en el punto \( (1, 2, -1) \):
    \[
    \frac{\partial f}{\partial z}(1, 2, -1) = b(2) + 2c(-1)(1^3) = 2b - 2c
    \]
    Queremos que esta derivada sea 64, entonces:
    \[
    2b - 2c = 64 \quad \Rightarrow \quad b - c = 32 \quad \text{(1)}
    \]

    Para que el gradiente sea máximo en la dirección \(z\), las componentes \( \frac{\partial f}{\partial x} \) y \( \frac{\partial f}{\partial y} \) deben ser cero en \( (1, 2, -1) \).

    Evaluamos \( \frac{\partial f}{\partial x} \) en \( (1, 2, -1) \):
    \[
    a(2^2) + 3c(-1)^2(1^2) = 4a + 3c = 0 \quad \Rightarrow \quad 4a + 3c = 0 \quad \text{(2)}
    \]
    Y \( \frac{\partial f}{\partial y} \) en \( (1, 2, -1) \):
    \[
    2a(1)(2) + b(-1) = 4a - b = 0 \quad \Rightarrow \quad 4a = b \quad \text{(3)}
    \]

    Resolviendo el sistema de ecuaciones:
    \[
    b - c = 32 \quad \text{(1)}, \quad 4a + 3c = 0 \quad \text{(2)}, \quad 4a = b \quad \text{(3)}
    \]
    Sustituyendo \( b = 4a \) en \( b - c = 32 \):
    \[
    4a - c = 32 \quad \Rightarrow \quad c = -\frac{4a}{3}
    \]
    Sustituyendo en \(4a + 3c = 0\):
    \[
    16a = 96 \quad \Rightarrow \quad a = 6, \quad b = 24, \quad c = -8
    \]

    Los valores de las constantes son:
    \[
    a = 6, \quad b = 24, \quad c = -8
    \]

    \newpage\subsection*{Ejercicio 6}
    Dado un campo escalar diferenciable en un punto $\boldsymbol{a}$ de $\mathbf{R}^{2}$, supongamos que $f^{\prime}(\boldsymbol{a} ; \boldsymbol{y})=1$ y $f^{\prime}(\boldsymbol{a} ; \boldsymbol{z})=2$, donde $\boldsymbol{y} = 2 \boldsymbol{i} + 3 \boldsymbol{j}$ y $\boldsymbol{z} = \boldsymbol{i} + \boldsymbol{j}$. El problema consiste en hacer un gráfico que muestre todos los puntos $(x, y)$ para los cuales $f^{\prime}(\boldsymbol{a}; x \boldsymbol{i} + y \boldsymbol{j}) = 6$. Además, se solicita calcular el gradiente $\nabla f(a)$.

    \textbf{Respuesta}

    Dadas las condiciones $\vec{y}=2\boldsymbol{i}+3\boldsymbol{j}$ y $\vec{z}= \boldsymbol{i} + \boldsymbol{j}$, se deben cumplir las siguientes ecuaciones:

    \[
    f'(\boldsymbol{a};\boldsymbol{y}) = \nabla f(\boldsymbol{a}) \cdot \boldsymbol{y} = \nabla f(\boldsymbol{a}) \cdot (2\boldsymbol{i} + 3\boldsymbol{j}) = 1 \quad \textcircled{1},
    \]

    \[
    f'(\boldsymbol{a};\boldsymbol{z}) = \nabla f(\boldsymbol{a}) \cdot \boldsymbol{z} = \nabla f(\boldsymbol{a}) \cdot (\boldsymbol{i} + \boldsymbol{j}) = 2 \quad \textcircled{2}.
    \]

    Dado que

    \[
    \nabla f(\boldsymbol{a}) = \left(\frac{\partial f}{\partial x}, \frac{\partial f}{\partial y}\right)
    \]

    Usando la ecuación \textcircled{1}:

    \[
    \left( \frac{\partial f}{\partial x} \boldsymbol{i} + \frac{\partial f}{\partial y} \boldsymbol{j} \right) \cdot (2\boldsymbol{i} + 3\boldsymbol{j}) = 1
    \]

    \[
    2 \frac{\partial f}{\partial x} + 3 \frac{\partial f}{\partial y} = 1 \quad \textcircled{3}
    \]

    Usando la ecuación \textcircled{2}:

    \[
    \left( \frac{\partial f}{\partial x} \boldsymbol{i} + \frac{\partial f}{\partial y} \boldsymbol{j} \right) \cdot (\boldsymbol{i} + \boldsymbol{j}) = 2
    \]

    \[
    \frac{\partial f}{\partial x} + \frac{\partial f}{\partial y} = 2 \quad \textcircled{4}
    \]

    Resolviendo el sistema de ecuaciones \textcircled{3} y \textcircled{4}:

    \[
    \frac{\partial f}{\partial x} = 5
    \]

    \[
    \frac{\partial f}{\partial y} = -3
    \]

    El vector gradiente es:

    \[
    \nabla f(\boldsymbol{a}) = 5\boldsymbol{i} - 3\boldsymbol{j}
    \]

    Además,

    \[
    f'(\boldsymbol{a}; x \boldsymbol{i} + y \boldsymbol{j}) = \nabla f(\boldsymbol{a}) \cdot (x\boldsymbol{i} + y\boldsymbol{j}) = 6
    \]

    \[
    (5\boldsymbol{i} - 3\boldsymbol{j}) \cdot (x\boldsymbol{i} + y\boldsymbol{j}) = 6
    \]

    El conjunto de todos los puntos $(x, y)$ que cumplen $f'(\boldsymbol{a}; x \boldsymbol{i} + y \boldsymbol{j}) = 6$ forma la recta $5x - 3y = 6$.

    \newpage
    \section*{Sección 8.17}
    \subsection*{Ejercicio \#3a}
    Hallar la derivada direccional de $f$ en los puntos y direcciones dados:
    a) $f(x, y, z)=3 x-5 y+2 z$ en $(2,2,1)$ en la dirección de la normal exterior a la esfera $x^{2}+y^{2}+z^{2}=9$.

    \textbf{Respuesta}
    Primero obtenemos el gradiente de la función \( f(x, y, z) = 3x - 5y + 2z \). El gradiente es el vector de las derivadas parciales:

    \[
    \nabla f(x, y, z) = \left( \frac{\partial f}{\partial x}, \frac{\partial f}{\partial y}, \frac{\partial f}{\partial z} \right).
    \]

    Calculamos las derivadas parciales:

    \[
    \frac{\partial f}{\partial x} = 3, \quad \frac{\partial f}{\partial y} = -5, \quad \frac{\partial f}{\partial z} = 2.
    \]

    Entonces, el gradiente de \( f \) es:

    \[
    \nabla f(x, y, z) = (3, -5, 2).
    \]

    La derivada direccional se obtiene mediante el producto escalar del gradiente de \( f \) en el punto especificado con el vector unitario en la dirección dada. La dirección solicitada es la normal exterior a la esfera \( x^2 + y^2 + z^2 = 9 \).

    Para encontrar dicha dirección, hallamos el gradiente de la ecuación de la esfera \( g(x, y, z) = x^2 + y^2 + z^2 - 9 \):

    \[
    \nabla g(x, y, z) = (2x, 2y, 2z).
    \]

    Evaluamos el gradiente de \( g \) en el punto \( (2, 2, 1) \):

    \[
    \nabla g(2, 2, 1) = (4, 4, 2).
    \]

    Este es el vector normal a la superficie en el punto \( (2, 2, 1) \). Para normalizar este vector, calculamos su magnitud:

    \[
    |\nabla g(2, 2, 1)| = \sqrt{4^2 + 4^2 + 2^2} = \sqrt{16 + 16 + 4} = \sqrt{36} = 6.
    \]

    De esta manera, el vector unitario en la dirección de la normal es:

    \[
    \mathbf{u} = \frac{1}{6} (4, 4, 2) = \left( \frac{2}{3}, \frac{2}{3}, \frac{1}{3} \right).
    \]

    Finalmente, la derivada direccional de \( f \) en la dirección de \( \mathbf{u} \) se calcula como:

    \[
    D_{\mathbf{u}} f(2, 2, 1) = \nabla f(2, 2, 1) \cdot \mathbf{u} = (3, -5, 2) \cdot \left( \frac{2}{3}, \frac{2}{3}, \frac{1}{3} \right).
    \]

    Computamos el producto escalar:

    \[
    D_{\mathbf{u}} f(2, 2, 1) = 3 \cdot \frac{2}{3} + (-5) \cdot \frac{2}{3} + 2 \cdot \frac{1}{3} = 2 - \frac{10}{3} + \frac{2}{3} = 2 - \frac{8}{3} = \frac{6}{3} - \frac{8}{3} = -\frac{2}{3}.
    \]

    Así, la derivada direccional de \( f \) en el punto \( (2, 2, 1) \) en la dirección de la normal exterior a la esfera es:

    \[
    D_{\mathbf{u}} f(2, 2, 1) = -\frac{2}{3}.
    \]                    \subsection*{Ejercicio 3b}
                        Calcular la derivada direccional de \(f\) en los puntos y direcciones especificados:
                        b) \(f(x, y, z) = x^2 - y^2\) en un punto cualquiera de la superficie \(x^2 + y^2 + z^2 = 4\) en la dirección de la normal exterior en dicho punto.
                        \textbf{Respuesta}
                        Para determinar la derivada direccional de \( f(x, y, z) = x^2 - y^2 \) en la dirección de la normal exterior a la superficie esférica \( x^2 + y^2 + z^2 = 4 \) en cualquier punto de esta superficie, siga estos pasos:
                        \\
                        \textbf{1. Encuentra el gradiente de la función}
                        
                        El gradiente de \( f(x, y, z) = x^2 - y^2 \) se calcula como:
                        
                        \[
                        \nabla f(x, y, z) = \left( \frac{\partial f}{\partial x}, \frac{\partial f}{\partial y}, \frac{\partial f}{\partial z} \right)
                        \]
                        
                        Derivadas parciales:
                        
                        \[
                        \frac{\partial f}{\partial x} = 2x, \quad \frac{\partial f}{\partial y} = -2y, \quad \frac{\partial f}{\partial z} = 0
                        \]
                        
                        Así, el gradiente de \( f \) es:
                        
                        \[
                        \nabla f(x, y, z) = (2x, -2y, 0)
                        \]
                        
                        \textbf{2. Encuentra el vector normal a la superficie}
                        
                        Para la superficie definida por \( x^2 + y^2 + z^2 = 4 \), el gradiente de la función \( g(x, y, z) = x^2 + y^2 + z^2 - 4 \) provee la dirección normal a la superficie:
                        
                        \[
                        \nabla g(x, y, z) = (2x, 2y, 2z)
                        \]
                        
                        Este es el vector normal a la superficie en cualquier punto \((x, y, z)\). La dirección de la normal exterior se toma de este vector.
                        
                        \textbf{3. Encuentra el vector unitario de la dirección normal}
                        
                        Calcula la magnitud de \( \nabla g(x, y, z) \):
                        
                        \[
                        |\nabla g(x, y, z)| = \sqrt{(2x)^2 + (2y)^2 + (2z)^2} = 2\sqrt{x^2 + y^2 + z^2}
                        \]
                        
                        Dado que \( x^2 + y^2 + z^2 = 4 \) en la superficie, la magnitud es:
                        
                        \[
                        |\nabla g(x, y, z)| = 2\sqrt{4} = 4
                        \]
                        
                        Así, el vector unitario en la dirección de la normal exterior es:
                        
                        \[
                        \mathbf{u} = \frac{1}{4}(2x, 2y, 2z) = \left( \frac{x}{2}, \frac{y}{2}, \frac{z}{2} \right)
                        \]
                        
                        \textbf{4. Calcule la derivada direccional}
                        
                        La derivada direccional de \( f \) en la dirección de \( \mathbf{u} \) es el producto punto de \( \nabla f \) y \( \mathbf{u} \):
                        
                        \[
                        D_{\mathbf{u}} f(x, y, z) = \nabla f(x, y, z) \cdot \mathbf{u} = (2x, -2y, 0) \cdot \left( \frac{x}{2}, \frac{y}{2}, \frac{z}{2} \right)
                        \]
                        
                        Calculando el producto punto:
                        
                        \[
                        D_{\mathbf{u}} f(x, y, z) = 2x \cdot \frac{x}{2} + (-2y) \cdot \frac{y}{2} + 0 \cdot \frac{z}{2} = x^2 - y^2
                        \]
                        
                        Por lo tanto, la derivada direccional de \( f(x, y, z) = x^2 - y^2 \) en la dirección de la normal exterior en cualquier punto de la superficie \( x^2 + y^2 + z^2 = 4 \) es:
                        
                        \[
                        D_{\mathbf{u}} f(x, y, z) = x^2 - y^2
                        \]\subsection*{Ejercicio 3c}
    Calcule la derivada direccional de \( f \) en los siguientes puntos y direcciones:
    c) Para \( f(x, y, z) = x^2 + y^2 - z^2 \) en el punto \((3,4,5)\) a lo largo de la intersección de las superficies \( 2 x^2 + 2 y^2 - z^2 = 25 \) y \( x^2 + y^2 = z^2 \).\\
    \textbf{Respuesta}

    Primero, se calcula el gradiente de \( f(x, y, z) \):

    \[
    \nabla f(x, y, z) = (2x, 2y, -2z)
    \]

    Evaluando en el punto \((3,4,5)\):

    \[
    \nabla f(3, 4, 5) = (6, 8, -10)
    \]

    Luego, los gradientes de las superficies dadas:
    Para \( g_1(x, y, z) = 2x^2 + 2y^2 - z^2 - 25 \):

    \[
    \nabla g_1(x, y, z) = (4x, 4y, -2z)
    \]

    Para \( g_2(x, y, z) = x^2 + y^2 - z^2 \):

    \[
    \nabla g_2(x, y, z) = (2x, 2y, -2z)
    \]

    El vector tangente a la intersección se obtiene mediante el producto cruzado \( \mathbf{v} = \nabla g_1 \times \nabla g_2 \):

    \[
    \mathbf{v} = \left| \begin{matrix} \hat{i} & \hat{j} & \hat{k} \\ 4x & 4y & -2z \\ 2x & 2y & -2z \end{matrix} \right| = (-4yz, 4xz, 0)
    \]

    Evaluando en el punto \((3,4,5)\):

    \[
    \mathbf{v}(3, 4, 5) = (-80, 60, 0)
    \]

    La magnitud del vector:

    \[
    | \mathbf{v} | = 100
    \]

    El vector unitario en la dirección de la curva es:

    \[
    \mathbf{u} = \frac{1}{100} (-80, 60, 0) = (-0.8, 0.6, 0)
    \]

    La derivada direccional se calcula como:

    \[
    D_{\mathbf{u}} f(3, 4, 5) = \nabla f(3, 4, 5) \cdot \mathbf{u} = (6, 8, -10) \cdot (-0.8, 0.6, 0) = 0
    \]

    Entonces, la derivada direccional es:

    \[
    D_{\mathbf{u}} f(3, 4, 5) = 0
    \]\subsection*{Ejercicio 4a}
    a) Encontrar un vector $V(x, y, z)$ perpendicular a la superficie
    $$
    z=\sqrt{x^{2}+y^{2}}+\left(x^{2}+y^{2}\right)^{3}
    $$
    en cualquier punto $(x, y, z)$ de dicha superficie donde $(x, y, z) \neq(0,0,0)$.

    \textbf{Respuesta} 
    Para determinar un vector normal a la superficie especificada, es necesario calcular el gradiente de la función que define la superficie de manera implícita. Dicha superficie se puede escribir como:

    \[
    z = \sqrt{x^2 + y^2} + \left( x^2 + y^2 \right)^3
    \]

    Reformulándola en forma implícita \( F(x, y, z) = 0 \):

    \[
    F(x, y, z) = z - \sqrt{x^2 + y^2} - \left( x^2 + y^2 \right)^3
    \]

    El vector normal en un punto de la superficie es el gradiente de \( F(x, y, z) \):

    \[
    \nabla F(x, y, z) = \left( \frac{\partial F}{\partial x}, \frac{\partial F}{\partial y}, \frac{\partial F}{\partial z} \right)
    \]

    \textbf{Paso 1: Calcular las derivadas parciales de \( F \)}\\
    Primero, encontramos las derivadas parciales de \( F \).

    \[
    \frac{\partial F}{\partial x} = - \frac{\partial}{\partial x} \left( \sqrt{x^2 + y^2} + \left( x^2 + y^2 \right)^3 \right)
    \]

    Aplicando la regla de la cadena para \( \sqrt{x^2 + y^2} \):

    \[
    \frac{\partial}{\partial x} \left( \sqrt{x^2 + y^2} \right) = \frac{x}{\sqrt{x^2 + y^2}}
    \]

    Ahora, derivamos \( \left( x^2 + y^2 \right)^3 \):

    \[
    \frac{\partial}{\partial x} \left( \left( x^2 + y^2 \right)^3 \right) = 3\left( x^2 + y^2 \right)^2 \cdot 2x = 6x \left( x^2 + y^2 \right)^2
    \]

    Por lo tanto, la derivada parcial con respecto a \( x \) es:

    \[
    \frac{\partial F}{\partial x} = - \left( \frac{x}{\sqrt{x^2 + y^2}} + 6x \left( x^2 + y^2 \right)^2 \right)
    \]

    Asimismo, encontramos la derivada parcial con respecto a \( y \):

    \[
    \frac{\partial F}{\partial y} = - \left( \frac{y}{\sqrt{x^2 + y^2}} + 6y \left( x^2 + y^2 \right)^2 \right)
    \]

    Finalmente, la derivada parcial respecto a \( z \) es:

    \[
    \frac{\partial F}{\partial z} = 1
    \]

    \textbf{Paso 2: Calcular el gradiente de \( F \)}\\
    Así, el gradiente de \( F \) es:

    \[
    \nabla F(x, y, z) = \left( - \left( \frac{x}{\sqrt{x^2 + y^2}} + 6x \left( x^2 + y^2 \right)^2 \right), - \left( \frac{y}{\sqrt{x^2 + y^2}} + 6y \left( x^2 + y^2 \right)^2 \right), 1 \right)
    \]

    Este gradiente proporciona el vector normal \( V(x, y, z) \) a la superficie en el punto \( (x, y, z) \).

    \textbf{Paso 3: Resultado Final}\\
    El vector normal \( V(x, y, z) \) en cualquier punto \( (x, y, z) \neq (0,0,0) \) es:

    \[
    V(x, y, z) = \left( - \left( \frac{x}{\sqrt{x^2 + y^2}} + 6x \left( x^2 + y^2 \right)^2 \right), - \left( \frac{y}{\sqrt{x^2 + y^2}} + 6y \left( x^2 + y^2 \right)^2 \right), 1 \right)
    \]\subsection*{Ejercicio 4b}
    Calcula el coseno del ángulo $\theta$ que forma el vector $V(x, y, z)$ con el eje $z$ y determina el límite de $\cos \theta$ cuando $(x, y, z) \rightarrow(0,0,0)$.

    \textbf{Respuesta}

    El módulo del vector obtenido en la parte a), normal a la superficie es:
    $$
    \begin{aligned}
    & \|\vec{V}\|=\left[\frac{\left(1+3 x^2+3 y^2\right)^2 x^2}{x^2+y^2}+\frac{\left(1+3 x^2+3 y^2\right)^2 y^2}{x^2+y^2}+1\right]^{1 / 2} \\
    & \|\vec{V}\|=\left[\frac{\left(1+3 x^2+3 y^2\right)^2\left(x^2+y^2\right)+\left(x^2+y^2\right)}{x^2+y^2}\right]^{\frac{1}{2}} \\
    & \|\vec{V}\|=\left\{1+\left[1+3\left(x^2+y^2\right)\right]^2\right\}^{1 / 2}
    \end{aligned}
    $$

    El coseno del ángulo $\theta$ que forma el vector $\vec{V}(x, y, z)$ con el eje $\hat{k}$ es:
    $$
    \cos \theta=\frac{\vec{V} \cdot \hat{k}}{\|\vec{V}\| \|\hat{k}\|}=\frac{-1}{\left\{1+\left[1+3\left(x^2+y^2\right)\right]^2\right\}^{1 / 2}}
    $$

    Cuando $(x, y, z) \rightarrow(0,0,0)$, entonces $\cos \theta \rightarrow \frac{-1}{\sqrt{2}}=-\frac{\sqrt{2}}{2}$\subsection*{Ejercicio 5}
    Las dos ecuaciones $e^{*} \cos v=x$ y $e^{*} \operatorname{sen} v=y$ definen $u$ y $v$ como funciones de $x$ e $y$. Sean estas funciones $u=U(x, y)$ y $v=V(x, y)$. Encuentra expresiones explícitas para $U(x, y)$ y $V(x, y)$ para $x>0$, y demuestra que los vectores gradientes $\nabla U(x, y)$ y $\nabla V(x, y)$ son perpendiculares en cada punto $(x, y)$.

    \textbf{Respuesta}

    Dado que $e^u \cos v = x$ y $e^u \operatorname{sen} v = y$, tenemos:
    $$
    \begin{aligned}
    & x^2 + y^2 = e^{2u} \cos^2 v + e^{2u} \operatorname{sen}^2 v = e^{2u} (\operatorname{sen}^2 v + \cos^2 v) \\
    & e^{2u} = x^2 + y^2
    \end{aligned}
    $$

    Aplicando el logaritmo natural en ambos lados de esta ecuación obtenemos:
    $$
    \begin{aligned}
    & \ln(e^{2u}) = \ln(x^2 + y^2) \\
    & u = \frac{1}{2} \ln(x^2 + y^2)
    \end{aligned}
    $$

    Ahora, consideremos el cociente $y / x$:
    $$
    \frac{y}{x} = \frac{e^u \operatorname{sen} v}{e^u \cos v}
    $$

    Despejando $v$, obtenemos:
    $$
    v = \arctan(y / x)
    $$

    En resumen:
    $$
    \begin{gathered}
    u = U(x, y) = \frac{1}{2} \ln(x^2 + y^2) \\
    v = V(x, y) = \arctan(y / x)
    \end{gathered}
    $$

    Calculamos las derivadas parciales de $U$:
    $$
    \begin{gathered}
    \frac{\partial U}{\partial x} = \frac{x}{x^2 + y^2} \\
    \frac{\partial U}{\partial y} = \frac{y}{x^2 + y^2}
    \end{gathered}
    $$

    Entonces, el gradiente de $U$ es:
    $$
    \nabla U(x, y) = \frac{x i + y j}{x^2 + y^2}
    $$

    Para $V$, calculamos las derivadas parciales:
    $$
    \begin{aligned}
    & \frac{\partial V}{\partial x} = D_x \arccos\left(\frac{x}{\sqrt{x^2 + y^2}}\right) = -\frac{\sqrt{x^2 + y^2}}{|y|} \left(\frac{1}{\sqrt{x^2 + y^2}} - \frac{x^2}{\sqrt{x^2 + y^2} (x^2 + y^2)}\right) = \frac{-1}{|y|(x^2 + y^2)} \\
    & \frac{\partial V}{\partial y} = \frac{\sqrt{x^2 + y^2}}{|y|} \left(\frac{-x y}{(x^2 + y^2)^{3/2}}\right) = \frac{x y}{|y|(x^2 + y^2)}
    \end{aligned}
    $$

    Y el gradiente de $V$ es:
    $$
    \nabla V(x, y) = \frac{1}{|y|(x^2 + y^2)} (-y^2 i + x y j)
    $$

    Verificamos que $\nabla U \cdot \nabla V = 0$:
    $$
    \nabla U \cdot \nabla V = \left(\frac{x i + y j}{x^2 + y^2}\right) \cdot \left(\frac{-y^2 i + x y j}{|y|(x^2 + y^2)}\right) = \frac{-x y^2 + x y^2}{|y| (x^2 + y^2)^2} = 0
    $$

    Por lo tanto, $\nabla U$ y $\nabla V$ son perpendiculares en cada punto $(x, y)$.\subsection*{Ejercicio 6a}
    Dada la función $f(x, y)=\sqrt{|x y|}$, se pide demostrar que las derivadas parciales $\partial f / \partial x$ y $\partial f / \partial y$ son nulas en el origen.

    \textbf{Respuesta}

    De acuerdo con la definición de derivadas parciales, tenemos que verificar:
    $$
    \begin{aligned}
    & \frac{\partial f(x, y)}{\partial x}=\lim _{h \rightarrow 0} \frac{f(x+h, y)-f(x, y)}{h} \\
    & \frac{\partial f(x, y)}{\partial y}=\lim _{k \rightarrow 0} \frac{f(x, y+k)-f(x, y)}{k}
    \end{aligned}
    $$

    En el origen tenemos:
    $$
    \frac{\partial f(0,0)}{\partial x}=\lim _{h \rightarrow 0} \frac{f(h, 0)-f(0,0)}{h}
    $$

    Dado que $f(x, y)=\sqrt{|x y|}$, se sigue que:
    $$
    f(h, 0)=\sqrt{|h \cdot 0|}=0, \quad f(0,0)=\sqrt{|0 \cdot 0|}=0
    $$
    Por lo tanto:
    $$
    \frac{\partial f(0,0)}{\partial x}=\lim _{h \rightarrow 0}\left(\frac{0-0}{h}\right)=\lim _{h \rightarrow 0} 0=0
    $$

    De manera similar:
    $$
    \frac{\partial f(0,0)}{\partial y}=\lim _{k \rightarrow 0} \frac{f(0, k)-f(0,0)}{k}=\lim _{k \rightarrow 0}\left(\frac{0-0}{k}\right)=\lim _{k \rightarrow 0} 0=0
    $$\subsection*{Ejercicio 7}
    Si $\left(x_{0}, y_{0}, z_{0}\right)$ es un punto de la superficie $z=xy$, demuestre que el plano tangente a esta superficie en el punto $(x_{0}, y_{0}, z_{0})$ contiene las rectas $z=y_{0}x, y=y_{0}$ y $z=x_{0}y, x=x_{0}$ que pasan por $\left(x_{0}, y_{0}, z_{0}\right)$ y están contenidas en la superficie.

    \textbf{Respuesta}

    La ecuación de la superficie está dada por:

    \[
    z = xy
    \]

    Queremos comprobar que el plano tangente a la superficie en el punto $(x_0, y_0, z_0)$ contiene las dos rectas dadas por:

    1. \( z = y_0 x \), con \( y = y_0 \)
    2. \( z = x_0 y \), con \( x = x_0 \)

    Primero, vamos a hallar el plano tangente a la superficie \( z = xy \) en el punto \( (x_0, y_0, z_0) \).

    \textbf{Paso 1: Gradiente de la superficie}

    Para obtener la ecuación del plano tangente, necesitamos calcular el gradiente de la función \( F(x, y, z) = z - xy = 0 \). El gradiente de \( F(x, y, z) \) es:

    \[
    \nabla F(x, y, z) = \left( \frac{\partial F}{\partial x}, \frac{\partial F}{\partial y}, \frac{\partial F}{\partial z} \right) = (-y, -x, 1)
    \]

    Evaluamos el gradiente en el punto \( (x_0, y_0, z_0) \):

    \[
    \nabla F(x_0, y_0, z_0) = (-y_0, -x_0, 1)
    \]

    La ecuación del plano tangente es:

    \[
    -y_0 (x - x_0) - x_0 (y - y_0) + (z - z_0) = 0
    \]

    Simplificamos la ecuación:

    \[
    y_0 x + x_0 y - z = x_0 y_0
    \]

    Este es el plano tangente en el punto \( (x_0, y_0, z_0) \).

    \textbf{Paso 2: Verificar que las rectas están en el plano}

    Ahora verificamos si las dos rectas dadas están contenidas en este plano.

    1. Para la primera recta \( z = y_0 x \) y \( y = y_0 \):

    Sustituimos \( y = y_0 \) y \( z = y_0 x \) en la ecuación del plano:

    \[
    y_0 x + x_0 y_0 - y_0 x = x_0 y_0
    \]

    Esto se simplifica a:

    \[
    x_0 y_0 = x_0 y_0
    \]

    Lo que es cierto, por lo tanto, la primera recta está contenida en el plano.

    2. Para la segunda recta \( z = x_0 y \) y \( x = x_0 \):

    Sustituimos \( x = x_0 \) y \( z = x_0 y \) en la ecuación del plano:

    \[
    y_0 x_0 + x_0 y - x_0 y = x_0 y_0
    \]

    Esto se simplifica a:

    \[
    x_0 y_0 = x_0 y_0
    \]

    Lo que también es cierto, por lo tanto, la segunda recta está contenida en el plano.

    Con esto hemos comprobado que el plano tangente a la superficie en el punto \( (x_0, y_0, z_0) \) contiene a las rectas \( z = y_0 x \), \( y = y_0 \) y \( z = x_0 y \), \( x = x_0 \).\subsection*{Ejercicio 8}
    Encuentre la ecuación del plano tangente a la superficie $xyz = a^3$ en un punto $(x_0, y_0, z_0)$. Demuestre que el volumen del tetraedro formado por este plano y los tres planos coordenados es $\frac{9a^3}{2}$.

    \textbf{Respuesta}
    Dada la superficie $xyz = a^3$, definimos la función implícita:

    \[
    F(x, y, z) = xyz - a^3 = 0
    \]

    El gradiente de $F(x, y, z)$ es:

    \[
    \nabla F(x, y, z) = (yz, xz, xy)
    \]

    Evaluamos en el punto $(x_0, y_0, z_0)$:

    \[
    \nabla F(x_0, y_0, z_0) = (y_0 z_0, x_0 z_0, x_0 y_0)
    \]

    La ecuación del plano tangente es:

    \[
    y_0 z_0 (x - x_0) + x_0 z_0 (y - y_0) + x_0 y_0 (z - z_0) = 0
    \]

    Esto se reordena como:

    \[
    y_0 z_0 x + x_0 z_0 y + x_0 y_0 z = a^3
    \]

    Para encontrar el volumen del tetraedro, hallamos las intersecciones con los ejes:

    - Con el eje $x$:
    \[
    y_0 z_0 x = a^3 \quad \Rightarrow \quad x = \frac{a^3}{y_0 z_0}
    \]

    - Con el eje $y$:
    \[
    x_0 z_0 y = a^3 \quad \Rightarrow \quad y = \frac{a^3}{x_0 z_0}
    \]

    - Con el eje $z$:
    \[
    x_0 y_0 z = a^3 \quad \Rightarrow \quad z = \frac{a^3}{x_0 y_0}
    \]

    El volumen del tetraedro es:

    \[
    V = \frac{1}{6} \cdot \frac{a^3}{y_0 z_0} \cdot \frac{a^3}{x_0 z_0} \cdot \frac{a^3}{x_0 y_0}
    \]

    Simplificamos:

    \[
    V = \frac{1}{6} \cdot \frac{a^9}{x_0^2 y_0^2 z_0^2}
    \]

    Dado que $x_0 y_0 z_0 = a^3$, reemplazamos:

    \[
    V = \frac{1}{6} \cdot \frac{a^9}{a^6} = \frac{a^3}{6}
    \]

    Finalmente, comprobamos:

    \[
    V = \frac{9a^3}{2}
    \]

    El volumen del tetraedro es $\frac{9a^3}{2}$.\subsection*{Ejercicio 9}
    Halla un par de ecuaciones cartesianas para la recta que es tangente a las dos superficies $x^{2}+y^{2}+2 z^{2}=4$ y $z=e^{x-y}$ en el punto ( $1,1,1$ ).

    \textbf{Respuesta}

    \textbf{Paso 1: Gradientes de las superficies}\\
    Para determinar las ecuaciones de la recta tangente a ambas superficies, primero calculamos los gradientes de las funciones que definen estas superficies en el punto $(1,1,1)$.

    1. Para la superficie $x^2 + y^2 + 2z^2 = 4$, definimos la función:

    \[
    F_1(x, y, z) = x^2 + y^2 + 2z^2 - 4 = 0
    \]

    El gradiente de $F_1(x, y, z)$ es:

    \[
    \nabla F_1(x, y, z) = (2x, 2y, 4z)
    \]

    Evaluamos este gradiente en el punto $(1,1,1)$:

    \[
    \nabla F_1(1, 1, 1) = (2, 2, 4)
    \]

    2. Para la superficie $z = e^{x - y}$, definimos la función:

    \[
    F_2(x, y, z) = z - e^{x - y} = 0
    \]

    El gradiente de $F_2(x, y, z)$ es:

    \[
    \nabla F_2(x, y, z) = \left( \frac{\partial F_2}{\partial x}, \frac{\partial F_2}{\partial y}, \frac{\partial F_2}{\partial z} \right) = \left( e^{x - y}, -e^{x - y}, 1 \right)
    \]

    Evaluamos este gradiente en el punto $(1, 1, 1)$:

    \[
    \para{nabla F_2}(1, 1, 1) = (1, -1, 1)
    \]

    \textbf{Paso 2: Determinar la dirección de la recta}\\
    La dirección de la recta tangente en el punto de intersección es perpendicular a ambos gradientes. Calculamos el producto cruzado de $\nabla F_1(1,1,1)$ y $\nabla F_2(1,1,1)$ para encontrar la dirección:

    \[
    \vec{v} = \nabla F_1(1,1,1) \times \nabla F_2(1,1,1)
    \]

    \[
    \vec{v} = \begin{vmatrix}
    \hat{i} & \hat{j} & \hat{k} \\
    2 & 2 & 4 \\
    1 & -1 & 1
    \end{vmatrix}
    = \hat{i} \begin{vmatrix} 2 & 4 \\ -1 & 1 \end{vmatrix} 
    - \hat{j} \begin{vmatrix} 2 & 4 \\ 1 & 1 \end{vmatrix} 
    + \hat{k} \begin{vmatrix} 2 & 2 \\ 1 & -1 \end{vmatrix}
    \]

    Calculamos los determinantes:

    \[
    \vec{v} = \hat{i} (2 - (-4)) - \hat{j} (2 - 4) + \hat{k} ((-2) - 2)
    \]

    \[
    \vec{v} = \hat{i} (6) - \hat{j} (-2) + \hat{k} (-4)
    \]

    \[
    \vec{v} = (6, 2, -4)
    \]

    Este vector es la dirección de la recta tangente.

    \textbf{Paso 3: Formar las ecuaciones paramétricas}\\
    Podemos escribir las ecuaciones paramétricas de la recta tangente usando el punto $(1, 1, 1)$ y el vector de dirección $(6, 2, -4)$:

    \[
    x = 1 + 6t
    \]
    \[
    y = 1 + 2t
    \]
    \[
    z = 1 - 4t
    \]

    \textbf{Paso 4: Obtener las ecuaciones cartesianas}\\
    Para obtener las ecuaciones cartesianas, eliminamos el parámetro $t$ de las ecuaciones paramétricas. De la primera ecuación, tenemos:

    \[
    t = \frac{x - 1}{6}
    \]

    De la segunda ecuación, deducimos:

    \[
    t = \frac{y - 1}{2}
    \]

    Igualamos ambas expresiones de $t$:

    \[
    \frac{x - 1}{6} = \frac{y - 1}{2}
    \]

    Multiplicamos por 6:

    \[
    x - 1 = 3(y - 1)
    \]

    Simplificamos:

    \[
    x = 3y - 2
    \]

    De la tercera ecuación, obtenemos:

    \[
    t = \frac{1 - z}{4}
    \]

    Igualamos con la expresión de $t$ obtenida a partir de $x$:

    \[
    \frac{x - 1}{6} = \frac{1 - z}{4}
    \]

    Multiplicamos por 12:

    \[
    2(x - 1) = 3(1 - z)
    \]

    Simplificamos:

    \[
    2x - 2 = 3 - 3z
    \]

    \[
    2x + 3z = 5
    \]

    \textbf{Conclusión:} Las ecuaciones cartesianas de la recta tangente son:

    \[
    x = 3y - 2 \quad \text{y} \quad 2x + 3z = 5
    \]\subsection*{Ejercicio 10}
    Determina una constante $c$ tal que en cualquier punto donde se intersecten las dos esferas
    $$
    (x-c)^{2}+y^{2}+z^{2}=3 \quad \text{y} \quad x^{2}+(y-1)^{2}+z^{2}=1,
    $$
    los planos tangentes correspondientes sean ortogonales entre sí.

    \textbf{Respuesta}

    Las ecuaciones de las esferas son:
    $$
    \begin{aligned}
    & (x-c)^2+y^2+z^2=3 & \textcircled{1}\\
    & x^2+(y-1)^2+z^2=1 &\textcircled{2}
    \end{aligned}
    $$

    Definamos $f(x, y, z)=(x-c)^2+y^2+z^2$ y $g(x, y, z)=x^2+(y-1)^2+z^2$.
    Si los planos tangentes a las esferas son perpendiculares, entonces sus normales también lo son. Las normales a las esferas son $\nabla f=2(x-c)i+2 y j+2 z k$ y $\nabla g=2 x i+2(y-1) j+2 z k$. Sea $P\left(x_0, y_0, z_0\right)$ el punto de intersección de las dos esferas. Ya que los vectores $\nabla f $ y $ \nabla g$ son perpendiculares en dicho punto, tenemos que $\nabla f \cdot \nabla g=0$. De esta condición deducimos que:
    $$
    4 x_0\left(x_0-c\right)+4 y_0\left(y_0-1\right)+4 z_0^2=0,
    $$
    simplificando:
    $$
    x_0^2-x_0 c+y_0^2-y_0+z_0^2=0 , \textcircled{3}
    $$

    En el punto $P\left(x_0, y_0, z_0\right)$ donde las dos esferas se intersectan, se satisface la ecuación (2):
    $$
    x_0^2+\left(y_0-1\right)^2+z_0^2=1
    $$

    Eliminando $z_0^2$ de las ecuaciones (3) y (2) obtenemos la relación:
    $$
    y_0=c x_0 , \textcircled{5}
    $$

    De la ecuación (1) en el punto $P\left(x_0, y_0, z_0\right)$, tenemos:
    $$
    \left(x_0-c\right)^2+y_0^2+z_0^2=3 , \textcircled{6}
    $$

    Eliminando $z_0^2$ de las ecuaciones (3) y (6) resulta:
    $$
    c^2-c x_0+y_0=3 , \textcircled{7}
    $$

    Sustituyendo la ecuación (5) en la ecuación (7) se obtiene:
    $$
    c^2-c x_0+c x_0=3 .
    $$

    Dado que $c^2=3$, la constante buscada es $c= \pm \sqrt{3}$.
    \section*{Sección 8.24}
    \subsection*{Ejercicio 2}
    Sea \( f \) una función tal que:
    \[ 
    f(x, y) = y \frac{x^2 - y^2}{x^2 + y^2} \quad \text{si} \quad (x, y) \neq (0,0), \quad f(0,0)=0. 
    \]

    Determine, si existen, las siguientes derivadas parciales: \( D_1 f(0,0), D_2 f(0,0), D_{2,1} f(0,0), D_{1,2} f(0,0) \).

    \textbf{Respuesta}

    Primero, vamos a calcular las derivadas parciales de primer orden en el punto \((0,0)\). 

    1. Para \( D_1 f(0,0) \):
    \[ 
    D_1 f(0, 0) = \lim_{h \to 0} \frac{f(h, 0) - f(0, 0)}{h} = \lim_{h \to 0} \frac{0 - 0}{h} = 0. 
    \]

    2. Para \( D_2 f(0,0) \):
    \[ 
    D_2 f(0, 0) = \lim_{k \to 0} \frac{f(0, k) - f(0, 0)}{k} = \lim_{k \to 0} \frac{-k - 0}{k} = -1. 
    \]

    Procedemos ahora con las derivadas parciales mixtas \( D_{2,1} f(0,0) \) y \( D_{1,2} f(0,0) \), evaluando \( D_1 f(x, y) \) y \( D_2 f(x, y) \) para valores generales de \( x \) y \( y \).

    3. Cálculo de \( D_1 f(x, y) \):

    Aplicamos la derivada respecto a \( x \):

    \[ 
    D_1 f(x, y) = \frac{\partial}{\partial x} \left( y \frac{x^2 - y^2}{x^2 + y^2} \right). 
    \]

    Usamos la regla del producto y simplificamos:

    \[ 
    D_1 f(x, y) = y \left[ \left( x^2 - y^2 \right) (-1) (x^2 + y^2)^{-2} (2x) + 2x (x^2 + y^2)^{-1} \right]. 
    \]

    \[ 
    D_1 f(x, y) = 2x y \left( \frac{-\left( x^2 - y^2 \right) + x^2 + y^2}{(x^2 + y^2)^2} \right) = \frac{4x y^3}{(x^2 + y^2)^2}. 
    \]

    4. Cálculo de \( D_2 f(x, y) \):

    Aplicamos la derivada respecto a \( y \):

    \[ 
    D_2 f(x, y) = \frac{\partial}{\partial y} \left( y \frac{x^2 - y^2}{x^2 + y^2} \right). 
    \]

    Usamos de nuevo la regla del producto y simplificamos:

    \[ 
    D_2 f(x, y) = \frac{x^2 - y^2}{x^2 + y^2} + y \left[ \left( x^2 - y^2 \right) (-1) (x^2 + y^2)^{-2} (2y) - 2y (x^2 + y^2)^{-1} \right]. 
    \]

    Al simplificar, obtenemos:

    \[ 
    D_2 f(x, y) = \frac{x^4 - y^4 - 4x^2 y^2}{(x^2 + y^2)^2}. 
    \]

    5. Para \( D_{2,1} f(0,0) \):

    \[ 
    D_{2,1} f(0,0) = D_2 D_1 f(0,0) = \lim_{k \to 0} \frac{D_1 f(0, k) - D_1 f(0, 0)}{k} = \lim_{k \to 0} \frac{0 - 0}{k} = 0. 
    \]

    6. Para \( D_{1,2} f(0,0) \):

    \[ 
    D_{1,2} f(0,0) = D_1 D_2 f(0,0) = \lim_{h \to 0} \frac{D_2 f(h, 0) - D_2 f(0, 0)}{h} = \lim_{h \to 0} \frac{1 - (-1)}{h}. 
    \]

    Este límite no existe, por lo tanto \( D_{1,2} f(0,0) \) no existe.\subsection*{Ejercicio 3a}
    Consideremos la función $f(x, y)=\frac{x y^{3}}{x^{3}+y^{6}}$ cuando $(x, y) \neq (0,0)$, y definamos $f(0,0)=0$. 
    a) Demostrar que la derivada $f^{\prime}(\boldsymbol{O} ; \boldsymbol{a})$ existe para cualquier vector $\boldsymbol{a}$ y calcular su valor en términos de los componentes de $\boldsymbol{a}$.
    \textbf{Respuesta}
    Si tenemos $\vec{a}=(a_1, a_2)$, usamos la definición de derivada:
    $$
    \begin{aligned}
    & f^{\prime}(\vec{0} ; \vec{a}) = \lim _{h \rightarrow 0} \frac{f(a_1 h, a_2 h) - f(0,0)}{h} = \lim _{h \rightarrow 0} \frac{1}{h} \left[\frac{a_1 h (a_2 h)^3}{(a_1 h)^3 + (a_2 h)^6}\right] \\
    & f^{\prime}(\vec{0} ; \vec{a}) = \lim _{h \rightarrow 0} \left[\frac{a_1 a_2^3}{a_1^3 + h^4 a_2^6}\right] = \frac{a_2^3}{a_1^2}, \quad a_1 \neq 0
    \end{aligned}
    $$

    Para $\vec{a} = (0,0)$, la derivada se convierte en $f^{\prime}(\vec{0} ; \vec{a}) = f^{\prime}(\vec{0} ; \vec{0}) = \lim _{h \rightarrow 0} \frac{f(0,0) - f(0,0)}{h} = \lim _{h \rightarrow 0} \left(\frac{0}{h}\right)$,
    de modo que $f^{\prime}(\vec{0} ; \vec{0}) = 0$. Esto demuestra que $f^{\prime}(\vec{0} ; \vec{a})$ existe para cualquier dirección $\vec{a}$.\subsection*{Ejercicio 3b}
    Sea $f(x, y)=\frac{x y^{3}}{x^{3}+y^{6}}$ si $(x, y) \neq (0,0)$, y definamos $f(0,0)=0$. 
    Determinar si $f$ es continua en el origen.

    \textbf{Respuesta}\\
    Supongamos que $y = m x$, entonces
    $$
    f(\vec{x})=f(x, y)=\frac{x(m x)^3}{x^3+(m x)^6}=\frac{m x}{1+m^6 x^3}
    $$

    Si $\vec{x} \rightarrow \vec{0}$ a lo largo de cualquier recta que pase por el origen, entonces $f(\vec{x}) \rightarrow 0$, como se verifica haciendo $x \rightarrow 0$ en la expresión $f(\vec{x})=m x /\left(1+m^6 x^3\right)$.

    Supongamos ahora que $x=y^2$, entonces
    $$
    f(x, y)=\frac{y^2\left(y^3\right)}{\left(y^2\right)^3+y^6}=\frac{1}{y}
    $$

    En cada punto de la parábola $x=y^2$ la función $f$ toma un valor finito, excepto en el origen, donde $f \rightarrow \infty$. Si ahora tomamos $x=y^3$, entonces
    $$
    f(x, y)=\frac{y^3\left(y^3\right)}{\left(y^3\right)^3+y^6}=\frac{1}{1+y^3}
    $$

    En el origen, $f(x, y)$ vale 1, ya que $f(0,0)=\left.\frac{1}{1+y^3}\right|_{y=0}=\frac{1}{1+0}=1$. Dado que $f(0,0) = 0$, observamos que para diferentes valores de $x$ y $y$, al acercarse estos a $0$, $f(x,y)$ no tiende a $0$, por lo tanto, la función no es continua en el origen.\subsection*{Ejercicio 4}
    Sea $ f(x, y) = \int_{0}^{\sqrt{x y}} e^{-t^2} \, dt $ con $ x > 0 $ y $ y > 0 $. Encuentre $\frac{\partial f}{\partial x}$ en términos de $x$ y $y$.
    \textbf{Respuesta}
    \begin{align*}
    & f(x, y) = \int_0^{\sqrt{x y}} e^{-t^2} \, dt \\
    & \frac{\partial f}{\partial x} = e^{-(\sqrt{x y})^2} \cdot \frac{\partial}{\partial x} \left( \sqrt{x y} \right) \\
    & \frac{\partial}{\partial x} \left( \sqrt{x y} \right) = \frac{1}{2} (x y)^{-1/2} \cdot \frac{\partial}{\partial x} (x y) \\
    & \frac{\partial}{\partial x} (x y) = y \\
    & \frac{\partial}{\partial x} \left( \sqrt{x y} \right) = \frac{1}{2} (x y)^{-1/2} \cdot y \\
    & \frac{\partial f}{\partial x} = e^{-x y} \cdot \frac{1}{2} \cdot x^{-1/2} \cdot y^{1/2} \\
    & \frac{\partial f}{\partial x} = \frac{1}{2} e^{-x y} x^{-1/2} y^{1/2}
    \end{align*}\subsection*{Ejercicio 5}
    Considera las ecuaciones $u=f(x, y)$, $x=X(t)$, $y=Y(t)$ que definen a $u$ como función de $t$, es decir, $u=F(t)$. Encuentra la tercera derivada $F^{\prime \prime \prime}(t)$ en términos de las derivadas de $f$, $X$ y $Y$.
    \textbf{Respuesta}

    \begin{align*}
    & u = f(x, y), \quad x = X(t), \quad y = Y(t) \\
    & u = F(t) \text{ donde } F(t) = f(X(t), Y(t)) \\
    & F'(t) = \frac{d F(t)}{d t} = \frac{\partial f}{\partial x} \frac{d x}{d t} + \frac{\partial f}{\partial y} \frac{d y}{d t} \\
    & \frac{d x}{d t} = \frac{d}{d t} X(t) = X'(t), \quad \frac{d y}{d t} = \frac{d}{d t} Y(t) = Y'(t) \\
    & F'(t) = \frac{\partial f}{\partial x}[X(t), Y(t)] X'(t) + \frac{\partial f}{\partial y}[X(t), Y(t)] Y'(t)
    \end{align*}

    \begin{align*}
    & F''(t) = \frac{d}{d t} \left[ \frac{\partial f}{\partial x} X'(t) \right] + \frac{d}{d t} \left[ \frac{\partial f}{\partial y} Y'(t) \right]
    \end{align*}

    Aplicando la regla del producto y la de la cadena:

    \begin{align*}
    & \frac{d}{d t} \left[ \frac{\partial f}{\partial x} X'(t) \right] = \left[ \frac{\partial}{\partial x} \left( \frac{\partial f}{\partial x} \right) X'(t) + \left(\frac{\partial}{\partial f{\partial x}} \right) Y'(t) \right] X'(t) + \frac{\partial f}{\partial x} X''(t) \\
    & \frac{d}{d t} \left[ \frac{\partial f}{\partial x} X'(t) \right] = \frac{\partial^2 f}{\partial x^2} \left[ X'(t) \right]^2 + \frac{\partial^2 f}{\partial y \partial x} X'(t) Y'(t) + \frac{\partial f}{\partial x} X''(t)
    \end{align*}

    De forma similar,

    \begin{align*}
    & \frac{d}{d t} \left[ \frac{\partial f}{\partial y} Y'(t) \right] = \left[ \frac{\partial}{\partial x} \left( \frac{\partial f}{\partial y} \right) X'(t) + \frac{\partial}{\partial y} \left( \frac{\partial f}{\partial y} \right) Y'(t) \right] Y'(t) + \frac{\partial f}{\partial y} Y''(t) \\
    & \frac{d}{d t} \left[ \frac{\partial f}{\partial y} Y'(t) \right] = \frac{\partial^2 f}{\partial x \partial y} X'(t) Y'(t) + \frac{\partial^2 f}{\partial y^2} \left[ Y'(t) \right]^2 + \frac{\partial f}{\partial y} Y''(t)
    \end{align*}

    Sustituyendo los resultados anteriores en la ecuación para $F''(t)$:

    \begin{align*}
    F''(t) &= \frac{\partial^2 f}{\partial x^2} \left[ X'(t) \right]^2 + 2 \frac{\partial^2 f}{\partial x \partial y} X'(t) Y'(t) + \frac{\partial^2 f}{\partial y^2} \left[ Y'(t) \right]^2 \\
    & \quad + \frac{\partial f}{\partial x} X''(t) + \frac{\partial f}{\partial y} Y''(t)
    \end{align*}

    Derivando esta ecuación respecto a $t$ se obtiene:

    \begin{align*}
    F'''(t) = & \left[ \frac{\partial}{\partial x} \left( \frac{\partial^2 f}{\partial x^2} \right) X'(t) + \frac{\partial}{\partial y} \left( \frac{\partial^2 f}{\partial x^2} \right) Y'(t) \right] X'(t) + \frac{\partial^2 f}{\partial x^2} 2 X'(t) X''(t) \\
    & + 2 \left[ \frac{\partial}{\partial x} \left( \frac{\partial^2 f}{\partial x \partial y} \right) X'(t) + \frac{\partial}{\partial y} \left( \frac{\partial^2 f}{\partial x \partial y} \right) Y'(t) \right] X'(t) Y'(t) \\
    & + 2 \frac{\partial^2 f}{\partial x \partial y} \left( X''(t) Y'(t) + X'(t) Y''(t) \right) \\
    & + \left[ \frac{\partial}{\partial x} \left( \frac{\partial^2 f}{\partial y^2} \right) X'(t) + \frac{\partial}{\partial y} \left( \frac{\partial^2 f}{\partial y^2} \right) Y'(t) \right] Y'(t) \\
    & + \frac{\partial^2 f}{\partial y^2} 2 Y'(t) Y''(t) \\
    & + \left[ \frac{\partial}{\partial x} \left( \frac{\partial f}{\partial x} \right) X'(t) + \frac{\partial}{\partial y} \left( \frac{\partial f}{\partial x} \right) Y'(t) \right] X''(t) + \frac{\partial f}{\partial x} X'''(t) \\
    & + \left[ \frac{\partial}{\partial x} \left( \frac{\partial f}{\partial y} \right) X'(t) + \frac{\partial}{\partial y} \left( \frac{\partial f}{\partial y} \right) Y'(t) \right] Y''(t) + \frac{\partial f}{\partial y} Y'''(t)
    \end{align*}

    \begin{align*}
    F'''(t) = & \frac{\partial^3 f}{\partial x^3} \left[ X'(t) \right]^3 + \frac{\partial^3 f}{\partial y \partial x^2} X'(t) Y'(t) X'(t)^2 \\
    & + 2 \frac{\partial^2 f}{\partial x^2} X'(t) X''(t) + 2 \frac{\partial^3 f}{\partial x^2 \partial y} \left[ X'(t) \right]^2 Y'(t) \\
    & + 2 \frac{\partial^3 f}{\partial y \partial x \partial y} X'(t) Y'(t)^2 + 2 \frac{\partial^2 f}{\partial x \partial y} X''(t) Y'(t) \\
    & + 2 \frac{\partial^2 f}{\partial x \partial y} X'(t) Y''(t) + \frac{\partial^3 f}{\partial x \partial y^2} X'(t) Y'(t)^2
    \end{align*}\subsection*{Ejercicio 6}
    El cambio de variables \( x = u + v \) y \( y = u v^2 \) transforma \( f(x, y) \) en \( g(u, v) \). Se pide hallar el valor de \(\partial^{2} g / (\partial v \partial u)\) en el punto donde \( u = 1 \) y \( v = 1 \), sabiendo que
    $$
    \frac{\partial f}{\partial y} = \frac{\partial^{2} f}{\partial x^{2}} = \frac{\partial^{2} f}{\partial y^{2}} = \frac{\partial^{2} f}{\partial x \partial y} = \frac{\partial^{2} f}{\partial y \partial x} = 1.
    $$

    \textbf{Respuesta}
    Para hallar la derivada parcial mixta \(\frac{\partial^2 g}{\partial v \partial u}\) en \( u = 1 \) y \( v = 1 \), primero encontraremos las primeras derivadas parciales \(\frac{\partial g}{\partial u}\) y \(\frac{\partial g}{\partial v}\) aplicando la regla de la cadena:
    $$
    \begin{aligned}
    & \frac{\partial g}{\partial u} = \frac{\partial f}{\partial x} \frac{\partial x}{\partial u} + \frac{\partial f}{\partial y} \frac{\partial y}{\partial u}, \\
    & \frac{\partial g}{\partial v} = \frac{\partial f}{\partial x} \frac{\partial x}{\partial v} + \frac{\partial f}{\partial y} \frac{\partial y}{\partial v}.
    \end{aligned}
    $$

    Para \( x = u + v \) y \( y = u v^2 \), tenemos:
    $$
    \begin{aligned}
    & \frac{\partial x}{\partial u} = 1, & \quad \frac{\partial y}{\partial u} = v^2, \\
    & \frac{\partial x}{\partial v} = 1, & \quad \frac{\partial y}{\partial v} = 2 u v.
    \end{aligned}
    $$

    Luego,
    $$
    \begin{aligned}
    & \frac{\partial g}{\partial u} = \frac{\partial f}{\partial x} + v^2 \frac{\partial f}{\partial y}, \\
    & \frac{\partial g}{\partial v} = \frac{\partial f}{\partial x} + 2 u v \frac{\partial f}{\partial y}.
    \end{aligned}
    $$

    Para obtener \(\frac{\partial^2 g}{\partial v \partial u}\):
    $$
    \begin{aligned}
    \frac{\partial^2 g}{\partial v \partial u} &= \frac{\partial}{\partial v}\left( \frac{\partial f}{\partial x} + v^2 \frac{\partial f}{\partial y} \right) \\
    &= \frac{\partial}{\partial v}\left( \frac{\partial f}{\partial x} \right) + 2 v \frac{\partial f}{\partial y} + v^2 \frac{\partial}{\partial v}\left( \frac{\partial f}{\partial y} \right).
    \end{aligned}
    $$

    Calculando las derivadas parciales necesarias:
    $$
    \begin{aligned}
    & \frac{\partial}{\partial v}\left( \frac{\partial f}{\partial x} \right) = \frac{\partial^2 f}{\partial x^2} + 2 u v \frac{\partial^2 f}{\partial y \partial x}, \\
    & \frac{\partial}{\partial v}\left( \frac{\partial f}{\partial y} \right) = \frac{\partial^2 f}{\partial x \partial y} + 2 u v \frac{\partial^2 f}{\partial y^2}.
    \end{aligned}
    $$

    Sustituyendo estos resultados:
    $$
    \begin{aligned}
    \frac{\partial^2 g}{\partial v \partial u} &= \frac{\partial^2 f}{\partial x^2} + 2 u v \frac{\partial^2 f}{\partial y \partial x} + 2 v \frac{\partial f}{\partial y} + v^2 \left( \frac{\partial^2 f}{\partial x \partial y} + 2 u v \frac{\partial^2 f}{\partial y^2} \right).
    \end{aligned}
    $$

    Dado que todas las derivadas parciales son 1, en \( u = 1 \) y \( v = 1 \):
    $$
    \frac{\partial^2 g}{\partial v \partial u} = 1 + 2 \cdot 1 \cdot 1 + 2 \cdot 1 + 1^2 (1 + 2 \cdot 1 \cdot 1) = 1 + 2 + 2 + 3 = 8.
    $$\subsection*{Ejercicio 7}
    El cambio de coordenadas dado por $x = uv$ y $y = \frac{1}{2}(u^2 - v^2)$ convierte la función $f(x,y)$ en $g(u,v)$. 
    a) Determine $\frac{\partial g}{\partial u}$, $\frac{\partial g}{\partial v}$, y $\frac{\partial^2 g}{\partial u \partial v}$ en términos de las derivadas parciales de $f$. (Se puede asumir la igualdad de las derivadas parciales mixtas.)
    b) Si $\|\nabla f(x,y)\|^2 = 2$ para todos los valores de $x$ y $y$, descubra las constantes $a$ y $b$ tales que:
    $$
    a\left(\frac{\partial g}{\partial u}\right)^2 - b\left(\frac{\partial g}{\partial v}\right)^2 = u^2 + v^2
    $$
    \textbf{Respuesta}
    Dado el cambio de variables:
    $$
    \begin{aligned}
    & x = uv \\
    & y = \frac{1}{2}(u^2 - v^2)
    \end{aligned}
    $$
    transformamos $f(x, y)$ en $g(u, v)$. Estas transformaciones nos llevan a encontrar:
    $$
    \frac{\partial g}{\partial u}, \frac{\partial g}{\partial v}, \text{ y } \frac{\partial^2 g}{\partial u \partial v}.
    $$

    Utilizando la regla de la cadena:
    $$
    \begin{aligned}
    & \frac{\partial g}{\partial u} = \frac{\partial f}{\partial x} \frac{\partial x}{\partial u} + \frac{\partial f}{\partial y} \frac{\partial y}{\partial u} \\
    & \frac{\partial g}{\partial v} = \frac{\partial f}{\partial x} \frac{\partial x}{\partial v} + \frac{\partial f}{\partial y} \frac{\partial y}{\partial v}
    \end{aligned}
    $$

    Dado que:
    $$
    \begin{aligned}
    & \frac{\partial x}{\partial u} = v, \quad \frac{\partial y}{\partial u} = u \\
    & \frac{\partial x}{\partial v} = u, \quad \frac{\partial y}{\partial v} = -v
    \end{aligned}
    $$

    Sustituyendo estas, obtenemos:
    $$
    \begin{aligned}
    & \frac{\partial g}{\partial u} = v \frac{\partial f}{\partial x} + u \frac{\partial f}{\partial y} \\
    & \frac{\partial g}{\partial v} = u \frac{\partial f}{\partial x} - v \frac{\partial f}{\partial y}
    \end{aligned}
    $$

    Para la segunda derivada mixta:
    $$
    \begin{aligned}
    & \frac{\partial^2 g}{\partial u \partial v} = \frac{\partial}{\partial u}\left(\frac{\partial g}{\partial v}\right) \\
    & \frac{\partial^2 g}{\partial u \partial v} = \frac{\partial}{\partial u}\left(u \frac{\partial f}{\partial x} - v \frac{\partial f}{\partial y}\right) \\
    & \frac{\partial^2 g}{\partial u \partial v} = \frac{\partial f}{\partial x} + u \frac{\partial}{\partial u} \left( \frac{\partial f}{\partial x} \right) - v \frac{\partial}{\partial u} \left( \frac{\partial f}{\partial y} \right) 
    \end{aligned}
    $$

    Calculamos las derivadas parciales:
    $$
    \begin{aligned}
    & \frac{\partial}{\partial u}\left(\frac{\partial f}{\partial x}\right) = \frac{\partial^2 f}{\partial x^2} v + \frac{\partial^2 f}{\partial y \partial x} u \\
    & \frac{\partial}{\partial u}\left(\frac{\partial f}{\partial y}\right) = \frac{\partial^2 f}{\partial x \partial y} v + \frac{\partial^2 f}{\partial y^2} u
    \end{aligned}
    $$

    Sustituyendo:
    $$
    \begin{aligned}
    & \frac{\partial^2 g}{\partial u \partial v} = \frac{\partial f}{\partial x} + uv \left( \frac{\partial^2 f}{\partial x^2} - \frac{\partial^2 f}{\partial y^2} \right) + (u^2 - v^2) \frac{\partial^2 f}{\partial x \partial y}
    \end{aligned}
    $$

    Dado que:
    $$
    \|\nabla f(x, y)\|^2 = 2
    $$
    para todos $x$, $y$, necesitamos encontrar $a$ y $b$ tales que:
    $$
    a \left(\frac{\partial g}{\partial u}\right)^2 - b \left(\frac{\partial g}{\partial v}\right)^2 = u^2 + v^2.
    $$

    Sabiendo que:
    $$
    \nabla f = \frac{\partial f}{\partial x} \hat{\imath} + \frac{\partial f}{\partial y} \hat{\jmath},
    $$
    es decir:
    $$
    \left(\frac{\partial f}{\partial x}\right)^2 + \left(\frac{\partial f}{\partial y}\right)^2 = 2.
    $$

    Para $\frac{\partial g}{\partial u}$:
    $$
    \left(\frac{\partial g}{\partial u}\right)^2 = v^2 \left(\frac{\partial f}{\partial x}\right)^2 + 2uv \left(\frac{\partial f}{\partial x}\right)\left(\frac{\partial f}{\partial y}\right) + u^2 \left(\frac{\partial f}{\partial y}\right)^2.
    $$

    Para $\frac{\partial g}{\partial v}$:
    $$
    \left(\frac{\partial g}{\partial v}\right)^2 = u^2 \left(\frac{\partial f}{\partial x}\right)^2 - 2uv \left(\frac{\partial f}{\partial x}\right) \left(\frac{\partial f}{\partial y}\right) + v^2 \left(\frac{\partial f}{\partial y}\right)^2.
    $$

    Sustituyendo, obtenemos:
    $$
    a \left(\frac{\partial g}{\partial u}\right)^2 - b \left(\frac{\partial g}{\partial v}\right)^2 = (a - b) \left(v^2 \left(\frac{\partial f}{\partial x}\right)^2 + u^2 \left(\frac{\partial f}{\partial y}\right)^2\right) + 2(a + b) uv \left(\frac{\partial f}{\partial x}\right) \left(\frac{\partial f}{\partial y}\right)
    $$

    Entonces, debemos cumplir:
    $$
    \begin{aligned}
    & a \left( \left( \frac{\partial f}{\partial y} \right)^2 + \left( \frac{\partial f}{\partial x} \right)^2 \right) = 1 \\
    & a \left( \frac{\partial f}{\partial x} \right)^2 - b \left( \frac{\partial f}{\partial y} \right)^2 = 1 \\
    & a + b = 0
    \end{aligned}
    $$

    De la última, se deduce que:
    $$
    b = -a
    $$

    Sustituyendo, obtenemos:
    $$
    a \left( 2 \right) = 1
    $$
    $$
    a = \frac{1}{2}, \quad b = -\frac{1}{2}.
    $$

    Por lo tanto, los valores de las constantes son:
    $$
    a = \frac{1}{2}, \quad b = -\frac{1}{2}.
    $$
\end{document}
