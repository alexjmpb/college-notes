\documentclass{report}
\usepackage[spanish]{babel}



\input{setup.tex}

\begin{document}
    \coverPage{ Matemáticas }{ Cálculo Integral y Series }{ Taller 1 }{  }{ Alexander Mendoza }{\today}

    \chapter*{Spivak Capítulo 13}

    \begin{enumerate}
        \item Demostrar que $\int_{0}^{b}x^3dx = \dfrac{b^4}{4}$.

        Consideremos la partición de [0, b] $P_n$ tal que $t_i - t_{i-1} = \dfrac{b}{n}$, así $t_0 = 0, t_1 = \dfrac{b}{n}, \dots , t_i =\dfrac{ib}{n}$. Con esto, sabemos que
        $$m_i = t_{i-1}^3 = \left((i-1)\dfrac{b}{n}\right)^3$$
        $$M_i = t_{i}^3 = \left(i\dfrac{b}{n}\right)^3$$

        Luego

        \begin{align*}
            L(f, P_n) &= \sum_{i=1}^{n}\left((i-1)\dfrac{b}{n}\right)^3\dfrac{b}{n}\\
            &= \sum_{i=1}^{n}(i-1)^3\dfrac{b^4}{n^4}\\
            &= \dfrac{b^4}{n^4} \sum_{i=1}^{n}(i-1)^3\\
            &= \dfrac{b^4}{n^4} \sum_{i=1}^{n-1}i^3\\
            &= \dfrac{b^4}{n^4} \left[\dfrac{n(n-1)}{2}\right]^2\\
            &= \dfrac{b^4}{4} \dfrac{(n-1)^2}{n^2}
        \end{align*}

        Con esto podemos observar que cuando $n$ se hace tan grande cuanto se quiera, $L(f, P_n)$ tiende a $\dfrac{b^4}{4}$. Continuamos de manera similar para $U(f, P_n)$

        \begin{align*}
            U(f, P_n) &= \sum_{i=1}^{n}\left(i\dfrac{b}{n}\right)^3\dfrac{b}{n}\\
            &= \sum_{i=1}^{n}i^3\dfrac{b^4}{n^4}\\
            &= \dfrac{b^4}{n^4} \sum_{i=1}^{n}i^3\\
            &= \dfrac{b^4}{n^4} \left[\dfrac{n(n+1)}{2}\right]^2\\
            &= \dfrac{b^4}{4} \dfrac{(n+1)^2}{n^2}
        \end{align*}

        Ahora calculemos la diferencia de las sumas

        \begin{align*}
            U(f, P_n) - L(f, P_n) &= \dfrac{b^4}{4} \dfrac{(n+1)^2}{n^2} - \dfrac{b^4}{4} \dfrac{(n-1)^2}{n^2}\\
            &= \dfrac{b^4}{4} \left[\dfrac{(n+1)^2}{n^2} - \dfrac{(n-1)^2}{n^2}\right]\\
            &= \dfrac{b^4}{4} \left(\dfrac{n^2+2}{n^2}\right)\\
            &= \dfrac{b^4}{n}
        \end{align*}

        De esta manera, dado $\epsilon > 0$ si $n > \dfrac{b^4}{\epsilon}$, entonces

        $$U(f, P_n) - L(f, P_n) = \dfrac{b^4}{n} < \epsilon$$

        Así, dado cualquier $\epsilon > 0$, podemos encontrar una partición $P_n$ de $[0,b]$ tal que $U(f, P_n) - L(f, P_n) < \epsilon$. Por lo tanto, $f$ es integrable. Luego como tanto $U(f, P_n)$ y $L(f, P_n)$ se aproximan a $\frac{b^4}{4}$ dado un $n$ lo suficiente mente grande, $\frac{b^4}{4}$ es el único número tal que

        $$L(f, P_n) =  \leq \dfrac{b^4}{4} \leq U(f, P_n)$$

        Por lo tanto $\int_{0}^{b} x^3 dx = \dfrac{b^4}{4}$.

        \item Demostrar que $\int_{0}^{b}x^4dx = \dfrac{b^5}{5}$.

        Consideremos la partición de [0, b] $P_n$ tal que $t_i - t_{i-1} = \dfrac{b}{n}$, así $t_0 = 0, t_1 = \dfrac{b}{n}, \dots , t_i =\dfrac{ib}{n}$. Con esto, sabemos que
        $$m_i = t_{i-1}^3 = \left((i-1)\dfrac{b}{n}\right)^3$$
        $$M_i = t_{i}^3 = \left(i\dfrac{b}{n}\right)^3$$

        Luego

        \begin{align*}
            L(f, P_n) &= \sum_{i=1}^{n}\left((i-1)\dfrac{b}{n}\right)^4\dfrac{b}{n}\\
            &= \sum_{i=1}^{n}(i-1)^4\dfrac{b^5}{n^5}\\
            &= \dfrac{b^5}{n^5} \sum_{i=1}^{n}(i-1)^4\\
            &= \dfrac{b^5}{n^5} \sum_{i=1}^{n-1}i^4\\
            &= \dfrac{b^5}{n^5} \dfrac{n(n-1)(2n-1)(3n^2-3n-1)}{30}\\
            &= \dfrac{b^5}{30} \dfrac{n(n-1)(2n-1)(3n^2-3n-1)}{n^5}\\
            &= \dfrac{b^5}{30} \left[6 + \dfrac{5(2-3n)n^2-1}{n^4}\right]\\
            &= \dfrac{b^5}{5} + \dfrac{d^5 5(2-3n)n^2-1}{30n^4}
        \end{align*}

        Con esto podemos observar que cuando $n$ se hace tan grande cuanto se quiera, $L(f, P_n)$ tiende a $\dfrac{b^4}{4}$. Continuamos de manera similar para $U(f, P_n)$

        \begin{align*}
            U(f, P_n) &= \sum_{i=1}^{n}\left(i\dfrac{b}{n}\right)^4\dfrac{b}{n}\\
            &= \sum_{i=1}^{n}i^4\dfrac{b^5}{n^5}\\
            &= \dfrac{b^5}{n^5} \sum_{i=1}^{n}i^4\\
            &= \dfrac{b^5}{n^5} \dfrac{n(n+1)(2n+1)(3n^2+3n-1)}{30}\\
            &= \dfrac{b^5}{30} \dfrac{n(n+1)(2n+1)(3n^2+3n-1)}{n^5}\\
            &= \dfrac{b^5}{30} \left[6 + \dfrac{5(2+3n)n^2-1}{n^4}\right]\\
            &= \dfrac{b^5}{5} + \dfrac{d^5 5(2+3n)n^2-1}{30n^4}
        \end{align*}

        Ahora calculemos la diferencia de las sumas

        \begin{align*}
            U(f, P_n) - L(f, P_n) &= \dfrac{b^5}{5} + \dfrac{d^5 5(2+3n)n^2-1}{30n^4} \dfrac{(n+1)^2}{n^2} - \dfrac{b^4}{4} \dfrac{(n-1)^2}{n^2}\\
            &= \dfrac{b^4}{4} \left[\dfrac{(n+1)^2}{n^2} - \dfrac{(n-1)^2}{n^2}\right]\\
            &= \dfrac{b^4}{4} \left(\dfrac{n^2+2}{n^2}\right)\\
            &= \dfrac{b^4}{n}
        \end{align*}

        De esta manera, dado $\epsilon > 0$ si $n > \dfrac{b^4}{\epsilon}$, entonces

        $$U(f, P_n) - L(f, P_n) = \dfrac{b^4}{n} < \epsilon$$

        Así, dado cualquier $\epsilon > 0$, podemos encontrar una partición $P_n$ de $[0,b]$ tal que $U(f, P_n) - L(f, P_n) < \epsilon$. Por lo tanto, $f$ es integrable. Luego como tanto $U(f, P_n)$ y $L(f, P_n)$ se aproximan a $\frac{b^4}{4}$ dado un $n$ lo suficiente mente grande, $\frac{b^4}{4}$ es el único número tal que

        $$L(f, P_n)  \leq \dfrac{b^4}{4} \leq U(f, P_n)$$

        Por lo tanto $\int_{0}^{b} x^3 dx = \dfrac{b^4}{4}$.
    \end{enumerate}
\end{document}
