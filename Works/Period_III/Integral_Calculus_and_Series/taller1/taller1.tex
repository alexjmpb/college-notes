\documentclass{report}
\usepackage[spanish]{babel}
\newtheorem{lemma}{Lema}
\usepackage{array}
\usepackage{float}
\usepackage[dvipsnames]{xcolor}

\input{setup.tex}

\begin{document}
    \coverPage{ Matemáticas }{ Cálculo Integral y Series }{ Taller 1 }{  }{ Alexander Mendoza }{\today}

    \textbf{\textcolor{NavyBlue}{Las respuestas para el capítulo 13 del Spivak se encuentran señaladas en azul}}\\
    \textbf{\textcolor{Red}{Las respuestas para el capítulo 14 del Spivak se encuentran señaladas en rojo}}\\
    \textbf{\textcolor{OliveGreen}{Las respuestas para los ejercicios complementarios se encuentran señaladas en verde}}\\

    \section*{\textcolor{NavyBlue}{Spivak Capítulo 13}}

    \begin{enumerate}[label=\textcolor{NavyBlue}{\textbf{\arabic*.}}]
        \item Demostrar que $\int_{0}^{b}x^3dx = \dfrac{b^4}{4}$.

        Consideremos la partición de [0, b] $P_n$ tal que $t_i - t_{i-1} = \dfrac{b}{n}$, así $t_0 = 0, t_1 = \dfrac{b}{n}, \dots , t_i =\dfrac{ib}{n}$. Con esto, sabemos que
        $$m_i = t_{i-1}^3 = \left((i-1)\dfrac{b}{n}\right)^3$$
        $$M_i = t_{i}^3 = \left(i\dfrac{b}{n}\right)^3$$

        Luego

        \begin{align*}
            L(f, P_n) &= \sum_{i=1}^{n}\left((i-1)\dfrac{b}{n}\right)^3\dfrac{b}{n}\\
            &= \sum_{i=1}^{n}(i-1)^3\dfrac{b^4}{n^4}\\
            &= \dfrac{b^4}{n^4} \sum_{i=1}^{n}(i-1)^3\\
            &= \dfrac{b^4}{n^4} \sum_{i=1}^{n-1}i^3\\
            &= \dfrac{b^4}{n^4} \left[\dfrac{n(n-1)}{2}\right]^2\\
            &= \dfrac{b^4}{4} \dfrac{(n-1)^2}{n^2}
        \end{align*}

        Con esto podemos observar que cuando $n$ se hace tan grande cuanto se quiera, $L(f, P_n)$ tiende a $\dfrac{b^4}{4}$. Continuamos de manera similar para $U(f, P_n)$

        \begin{align*}
            U(f, P_n) &= \sum_{i=1}^{n}\left(i\dfrac{b}{n}\right)^3\dfrac{b}{n}\\
            &= \sum_{i=1}^{n}i^3\dfrac{b^4}{n^4}\\
            &= \dfrac{b^4}{n^4} \sum_{i=1}^{n}i^3\\
            &= \dfrac{b^4}{n^4} \left[\dfrac{n(n+1)}{2}\right]^2\\
            &= \dfrac{b^4}{4} \dfrac{(n+1)^2}{n^2}
        \end{align*}

        Ahora calculemos la diferencia de las sumas

        \begin{align*}
            U(f, P_n) - L(f, P_n) &= \dfrac{b^4}{4} \dfrac{(n+1)^2}{n^2} - \dfrac{b^4}{4} \dfrac{(n-1)^2}{n^2}\\
            &= \dfrac{b^4}{4} \left[\dfrac{(n+1)^2}{n^2} - \dfrac{(n-1)^2}{n^2}\right]\\
            &= \dfrac{b^4}{4} \left(\dfrac{n^2+2}{n^2}\right)\\
            &= \dfrac{b^4}{n}
        \end{align*}

        De esta manera, dado $\epsilon > 0$ si $n > \dfrac{b^4}{\epsilon}$, entonces

        $$U(f, P_n) - L(f, P_n) = \dfrac{b^4}{n} < \epsilon$$

        Así, dado cualquier $\epsilon > 0$, podemos encontrar una partición $P_n$ de $[0,b]$ tal que $U(f, P_n) - L(f, P_n) < \epsilon$. Por lo tanto, $f$ es integrable. Luego como tanto $U(f, P_n)$ y $L(f, P_n)$ se aproximan a $\dfrac{b^4}{4}$ dado un $n$ lo suficiente mente grande, $\dfrac{b^4}{4}$ es el único número tal que

        $$L(f, P_n) = \dfrac{b^4}{4} \dfrac{(n-1)^2}{n^2} \leq \dfrac{b^4}{4} \leq \dfrac{b^4}{4} \dfrac{(n+1)^2}{n^2} = U(f, P_n)$$

        Por lo tanto $\int_{0}^{b} x^3 dx = \dfrac{b^4}{4}$.

        \item Demostrar que $\int_{0}^{b}x^4dx = \dfrac{b^5}{5}$.

        Consideremos la partición de [0, b] $P_n$ tal que $t_i - t_{i-1} = \dfrac{b}{n}$, así $t_0 = 0, t_1 = \dfrac{b}{n}, \dots , t_i =\dfrac{ib}{n}$. Con esto, sabemos que
        $$m_i = t_{i-1}^4 = \left((i-1)\dfrac{b}{n}\right)^4$$
        $$M_i = t_{i}^4 = \left(i\dfrac{b}{n}\right)^4$$

        Luego

        \begin{align*}
            L(f, P_n) &= \sum_{i=1}^{n}\left((i-1)\dfrac{b}{n}\right)^4\dfrac{b}{n}\\
            &= \sum_{i=1}^{n}(i-1)^4\dfrac{b^5}{n^5}\\
            &= \dfrac{b^5}{n^5} \sum_{i=1}^{n}(i-1)^4\\
            &= \dfrac{b^5}{n^5} \sum_{i=1}^{n-1}i^4\\
            &= \dfrac{b^5}{n^5} \dfrac{n(n-1)(2n-1)(3n^2-3n-1)}{30}\\
            &= \dfrac{b^5}{30} \dfrac{n(n-1)(2n-1)(3n^2-3n-1)}{n^5}\\
            &= \dfrac{b^5}{30} \left[6 + \dfrac{5(2-3n)n^2-1}{n^4}\right]\\
            &= \dfrac{b^5}{5} + \dfrac{b^5 5(2-3n)n^2-1}{30n^4}
        \end{align*}

        Con esto podemos observar que cuando $n$ se hace tan grande cuanto se quiera, $L(f, P_n)$ tiende a $\dfrac{b^4}{4}$. Continuamos de manera similar para $U(f, P_n)$

        \begin{align*}
            U(f, P_n) &= \sum_{i=1}^{n}\left(i\dfrac{b}{n}\right)^4\dfrac{b}{n}\\
            &= \sum_{i=1}^{n}i^4\dfrac{b^5}{n^5}\\
            &= \dfrac{b^5}{n^5} \sum_{i=1}^{n}i^4\\
            &= \dfrac{b^5}{n^5} \dfrac{n(n+1)(2n+1)(3n^2+3n-1)}{30}\\
            &= \dfrac{b^5}{30} \dfrac{n(n+1)(2n+1)(3n^2+3n-1)}{n^5}\\
            &= \dfrac{b^5}{30} \left[6 + \dfrac{5(2+3n)n^2-1}{n^4}\right]\\
            &= \dfrac{b^5}{5} + \dfrac{b^5 5(2+3n)n^2-1}{30n^4}
        \end{align*}

        Ahora calculemos la diferencia de las sumas

        \begin{align*}
            U(f, P_n) - L(f, P_n) &= \dfrac{b^5}{5} + \dfrac{b^5 5(2+3n)n^2-1}{30n^4} - \dfrac{b^5}{5} + \dfrac{b^5 5(2-3n)n^2-1}{30n^4}\\
            &= \dfrac{b^5}{n}
        \end{align*}

        De esta manera, dado $\epsilon > 0$ si $n > \dfrac{b^4}{\epsilon}$, entonces

        $$U(f, P_n) - L(f, P_n) = \dfrac{b^5}{n} < \epsilon$$

        Así, dado cualquier $\epsilon > 0$, podemos encontrar una partición $P_n$ de $[0,b]$ tal que $U(f, P_n) - L(f, P_n) < \epsilon$. Por lo tanto, $f$ es integrable. Luego como tanto $U(f, P_n)$ y $L(f, P_n)$ se aproximan a $\dfrac{b^5}{5}$ dado un $n$ lo suficiente mente grande, $\dfrac{b^5}{5}$ es el único número tal que

        $$L(f, P_n) = \dfrac{b^5}{5} + \dfrac{b^5 5(2-3n)n^2-1}{30n^4} \leq \dfrac{b^5}{5} \leq \dfrac{b^5}{5} + \dfrac{b^5 5(2+3n)n^2-1}{30n^4} = U(f, P_n)$$

        Por lo tanto $\int_{0}^{b} x^3 dx = \dfrac{b^5}{5}$.

        \setcounter{enumi}{4}
        \item Obtener sin cálculos
            \begin{enumerate}[label=\textcolor{NavyBlue}{\textbf{\roman*.}}]
                \item $\int_{-1}^{1}x^3\sqrt{1-x^2}dx$

                La gráfica de la función es la siguiente:

                \begin{center}
                    \fbox{\adjustbox{trim=1 1 1 1, width=.5\textwidth}{\includegraphics{images/grafica1.png}}}
                \end{center}

                Observando la función, podemos observar que es impar, esto se puede verificar ya que

                $$ - (x^3\sqrt{1-x^2}) = (-x)^3\sqrt{1-(-x)^2}$$

                Luego $\int_{-1}^{1}x^3\sqrt{1-x^2}dx = 0$, esto se puede observar gráficamente ya que el área debajo del eje $x$ es igual al área sobre el eje, por lo cual se cancelan.

                \item $\int_{-1}^{1}(x^5+3)\sqrt{1-x} dx$

                Primero manipulemos la expresión para que sea más fácil de trabajar,

                $$\int_{-1}^{1}(x^5+3)\sqrt{1-x}dx = \int_{-1}^{1}x^5\sqrt{1-x}dx + \int_{-1}^{1}3\sqrt{1-xdx}$$

                Observemos que $x^5\sqrt{1-x}$ es impar, por lo tanto $\int_{-1}^{1}x^5\sqrt{1-x} = 0$. Además observemos que $\sqrt{1-xdx}$ es una función par, por lo tanto  $\int_{-1}^{1}\sqrt{1-xdx} = 2\int_{0}^{1}\sqrt{1-xdx}$. Por último observemos que $\sqrt{1-x}$ es la función del círculo unitario, por lo tanto $\int_{0}^{1}\sqrt{1-xdx} = \pi/4$  así tenemos los siguiente

                \begin{align*}
                    \int_{-1}^{1}(x^5+3)\sqrt{1-x}dx &= \int_{-1}^{1}x^5\sqrt{1-x}dx + \int_{-1}^{1}3\sqrt{1-xdx}\\
                    &= 0 + 3\int_{-1}^{1}\sqrt{1-xdx}\\
                    &= 6\int_{0}^{1}\sqrt{1-xdx}\\
                    &= 6 \left(\dfrac{\pi}{4}\right)\\
                    &= \dfrac{3\pi}{4}
                \end{align*}
            \end{enumerate}
            \setcounter{enumi}{6}
            \item Decidir cuáles de las siguientes funciones siguientes son integrables sobre $[0,2]$, y calcular la integral cuando sea posible.
            \begin{enumerate}[label=\textcolor{NavyBlue}{\textbf{\roman*.}}]
                \item $f(x) = \begin{cases}
                    x, & 0\leq x < 1\\
                    x-2, & 1\leq x \leq 2
                \end{cases}$

                La función es integrable y su integral se calcula de la siguiente manera

                \begin{align*}
                    \int_{0}^{2}f(x)dx &= \int_{0}^{1}xdx + \int_{1}^{2}x-2dx\\
                    &= \int_{0}^{1}xdx + \int_{1}^{2}x -\int_{1}^{2}2dx\\
                    &= \left(\dfrac{1^2}{2} - \dfrac{0^2}{2}\right) + \left(\dfrac{2^2}{2} - \dfrac{1^2}{2}-2(2-1)\right)\\
                    &= \dfrac{1}{2} + 2 - \dfrac{1}{2} - 4 + 2\\
                    &= 0
                \end{align*}

                \setcounter{enumii}{2}
                \item $f(x) = x + [x]$

                La función es integrable y su integral se calcula de la siguiente manera

                \begin{align*}
                    \int_{0}^{2}x+[x]dx &= \int_{0}^{2} xdx + \int_{0}^{2}[x]dx\\
                    &= \int_{0}^{2} xdx + \int_{0}^{1}[x]dx + \int_{1}^{2} [x] dx\\
                    &= \left(\dfrac{2^2}{2} - \dfrac{0^2}{2}\right) + 0(1-0) + 1(2-1)\\
                    &= 3
                \end{align*}

                \setcounter{enumii}{5}
                \item $f(x) = \begin{cases}
                    \dfrac{1}{\left[\dfrac{1}{x}\right]}, & 0 < x \leq 1\\
                    0, & x=0 \text{ o } x > 1
                \end{cases}$

                Analicemos los valores que toma nuestra función

                \begin{table}[H]
                    \setlength{\extrarowheight}{20pt}
                    \centering
                    \begin{adjustbox}{valign=c}
                    \begin{tabular}{|c|c|}
                        \hline
                        $\left[\dfrac{1}{x}\right]$ & Condición                           \\ \hline
                        $\dfrac{1}{1}$              & $1 \leq \dfrac{1}{x} < 2 \leftrightarrow \dfrac{1}{2} < x \leq 1$               \\ \hline
                        $\dfrac{1}{2}$              & $2 \leq \dfrac{1}{x} < 3 \leftrightarrow \dfrac{1}{3} < x \leq \dfrac{1}{2}$     \\ \hline
                        $\dfrac{1}{2}$              & $3 \leq \dfrac{1}{x} < 4 \leftrightarrow \dfrac{1}{4} < x \leq \dfrac{1}{3}$     \\ \hline
                        $\vdots$                    & $\vdots$                                                                       \\ \hline
                        $\dfrac{1}{n}$                & $n \leq \dfrac{1}{x} < n+1 \leftrightarrow \dfrac{1}{n+1} < x \leq \dfrac{1}{n}$ \\ \hline
                    \end{tabular}
                \end{adjustbox}
                \end{table}

                Con esta tabla de valores podemos obtener la siguiente gráfica\\
                \begin{center}
                    \fbox{\adjustbox{trim=1 1 1 1, width=.5\textwidth}{\includegraphics{images/grafica2.png}}}
                \end{center}

                Así, podemos dividir $(0, 1]$ en intervalos, obtener el área de esos intervalos y sumar las áreas. Haremos esto de la siguiente manera, primero veamos cuál es el área de los intervalos

                $$\int_{\dfrac{1}{2}}^{1} f(x)dx = 1(1-\dfrac{1}{2}) = \dfrac{1}{2}$$
                $$\int_{\dfrac{1}{3}}^{\dfrac{1}{2}} f(x)dx = \dfrac{1}{2}(\dfrac{1}{2} - \dfrac{1}{3}) = \dfrac{1}{12}$$
                $$\int_{\dfrac{1}{4}}^{\dfrac{1}{3}} f(x)dx = \dfrac{1}{3}(\dfrac{1}{3} - \dfrac{1}{4}) = \dfrac{1}{36}$$

                Luego, nuestra integral toma el siguiente valor

                \begin{align*}
                    \int_{0}^{2}f(x)dx &= \int_{0}^{1}f(x)dx + \int_{1}^{2}f(x)dx\\
                    &= \int_{0}^{1}f(x)dx + 0\\
                    &= 1(1-\dfrac{1}{2}) + \dfrac{1}{2}(\dfrac{1}{2} - \dfrac{1}{3}) + \dfrac{1}{3}(\dfrac{1}{3} - \dfrac{1}{4}) + \cdots\\
                    &= \sum_{n=1}^{\infty} \dfrac{1}{n^2(n+1)}\\
                    &= \frac{1}{6}(\pi^2-6)
                \end{align*}

                \item $f$ es la función indicada en la siguiente figura
                \begin{center}
                    \fbox{\adjustbox{trim=1 1 1 1, clip, width=0.7\textwidth}{\includegraphics{images/grafica3.png}}}
                \end{center}

                Analicemos el área en cada intervalo en el cambia la función. En cada intervalo la función forma un triángulo, algunos intervalos en los que podemos observar que cambia la función son $[1,2]$, $[\frac{1}{2}, 1$], $[\frac{1}{4}, \frac{1}{2}]$, $[\frac{1}{8}, \frac{1}{4}]$.

                $\int_{1}^{2}f(x)dx = \frac{1(2-1)}{2} = \frac{1}{2}$
                Para el intervalo $[\frac{1}{2}, 1]$, el área del triángulo es:

                \[
                \int_{\frac{1}{2}}^{1} f(x) \, dx = \frac{1 (1-\frac{1}{2})}{2} = \frac{1}{4}
                \]

                Para el intervalo $[\frac{1}{4}, \frac{1}{2}]$, el área del triángulo es:

                \[
                \int_{\frac{1}{4}}^{\frac{1}{2}} f(x) \, dx = \frac{1 (\frac{1}{2}-\frac{1}{4})}{2} = \frac{1}{8}
                \]

                Finalmente, para el intervalo $[\frac{1}{8}, \frac{1}{4}]$, el área del triángulo es:

                \[
                \int_{\frac{1}{8}}^{\frac{1}{4}} f(x) \, dx = \frac{1 (\frac{1}{4}-\frac{1}{8})}{2} = \frac{1}{16}
                \]

                Con esto tenemos que

                \begin{align*}
                    \int_{0}^{2} f(x)dx &= \int_{1}^{2}f(x)dx + \int_{\frac{1}{2}}^{1}f(x)\,dx + \int_{\frac{1}{4}}^{\frac{1}{2}}f(x)\,dx + \cdots\\
                    &= \frac{1}{2} + \frac{1}{4} + \frac{1}{8} + \cdots\\
                    &= \sum_{n=1}^{\infty}\frac{1}{2^n}\\
                    &= 1
                \end{align*}
            \end{enumerate}

            \item Hallar las áreas de las regiones limitadas por
            \begin{enumerate}[label=\textcolor{NavyBlue}{\textbf{\roman*.}}]
                \setcounter{enumii}{2}
                \item Las gráficas de $f(x) = x^2$ y $g(x) = 1 - x^2$
                \begin{center}
                    \fbox{\adjustbox{trim=1 1 1 1, width=.5\textwidth}{\includegraphics{images/grafica4.png}}}
                \end{center}

                Consideraremos el intervalo $[-1, 1]$, sabemos que la altura va a ser $h = (2-x^2)-x^2$, luego

                \begin{align*}
                    A &= \int_{-1}^{1}1-2x^2\,dx\\
                    &= \int_{-1}^{1}1\,dx - \int_{-1}^{1}2x^2\,dx\\
                    &= \left. x - \frac{2^3}{x} \right|_{-1}^{1}\\
                    &= \frac{2}{3}
                \end{align*}
                \setcounter{enumii}{5}
                \item La gráfica de $f(x) = \sqrt{x}$, el eje horizontal y la vertical por $(2, 0)$.
                La gráfica de la función es la siguiente

                \begin{center}
                    \fbox{\adjustbox{trim=1 1 1 1, width=.7\textwidth}{\includegraphics{images/grafica5.png}}}
                \end{center}

            \end{enumerate}
            \item Hallar
            $$\int_{a}^{b}\left(\int_{c}^{d}f(x)g(y)dy\right)$$
            en términos de $\int_{a}^{b}f$ y $\int_{a}^{b}g$.

            Para esto, consideraremos las constantes que están dentro de las integrales, observemos que en integral de adentro, $f(x)$ sería una constante ya que se está integrando con respecto a $y$ y no a $x$. Así, tenemos

            \begin{align*}
                \int_{a}^{b}\left(\int_{c}^{d}f(x)g(y)dy\right)\, dx &= \int_{a}^{b}f(x)\left(\int_{c}^{d}g(y)\, dy\right)\, dx
            \end{align*}

            Observemos ahora que la integral de adentro es constante con respecto a la integral de afuera, así se puede concluir lo siguiente

            \begin{align*}
                \int_{a}^{b}f(x)\left(\int_{c}^{d}g(y)\, dy\right)\, dx &= \int_{c}^{d}g(y)\,dy \int_{a}^{b}f(x)\,dx
            \end{align*}

            \setcounter{enumi}{12}
            \item Si $a < b < c < d$ y $f$ es integrable sobre $[a, d]$, demostrar que $f$ es integrable sobre $[b,c]$.

            Para el ejercicio usaremos el Teorema 4. Como $a < b < c < d$, tenemos que $a<c<d$ y como $f$ es integrable sobre $[a, d]$, por el Teorema 4 tenemos que $f$ es integrable sobre $[a,c]$ y sobre $[c,d]$. Luego, como $a<b<c$, aplicando nuevamente el Teorema 4 tenemos que $f$ es integrable sobre $[b, c]$.

            \setcounter{enumi}{19}
            \item Supongamos que $f$ está acotada sobre $[a,b]$ y que $f$ es continua en todo punto de $[a,b]$ con la excepción de $x_0$ de $(a,b)$. Demostrar que $f$ es integrable sobre $[a,b]$.

            Primero definamos las siguientes funciones

            $$f_1(x) = \begin{cases}
                f(x), & x<x_0\\
                0, & x>x_0\\
                m, & x = x_0
            \end{cases}$$

            Donde $m = \lim_{x \to x_0^-} f(x)$

            $$f_2(x) = \begin{cases}
                f(x), & x>x_0\\
                0, & x<x_0\\
                n, & x = x_0
            \end{cases}$$

            Donde $n = \lim_{x \to x_0^+} f(x)$. Y sea

            $$f_3(x) = \begin{cases}
                0, & x\not=x_0\\
                f(x_0)-m-n, & x=x_0\\
            \end{cases}$$

            Luego $f = f_1 + f_2 + f_3$, como $f_1, f_2, f_3$ son integrables, $f$ es integrable por el Teorema 5.
    \end{enumerate}
    \section*{\textcolor{Red}{Spivak Capítulo 14}}
    \begin{enumerate}[label=\textcolor{Red}{\textbf{\arabic*.}}]
        \item Hallar las derivadas de cada una de las funciones siguientes:
        \begin{enumerate}[label=\textcolor{Red}{\textbf{\roman*.}}]
            \setcounter{enumii}{2}
            \item $$F(x) = \int_{15}^{x}\left(\int_{8}^{y}\dfrac{1}{1+t^2+\sin^2t}\,dt\right)dy$$
            Sea $f(y) = \int_{8}^{y}\dfrac{1}{1+t^2+\sin^2t}\,dt$, así por el teorema fundamental del cálculo

            \begin{align*}
                F'(x) &= -f(x)\\
                &= -\int_{8}^{x}\dfrac{1}{1+t^2+\sin^2t}\,dt\\
                &= \int_{x}^{8}\dfrac{1}{1+t^2+sin^2t}
            \end{align*}
            \setcounter{enumii}{4}
            \item
            $$F(x) = \int_{a}^{b}\dfrac{x}{1+t^2+sin^2t}dt$$
            Como $x$ es constante, tenemos
            $$x\int_{a}^{b}\dfrac{1}{1+t^2+sin^2t}dt$$
            Además, la integral es una integral definida por lo cual sabemos que tomará algún valor fijo, por lo cual la función se puede ver como una variable $x$ por una constante, así

            $$F'(x) = \int_{a}^{b}\dfrac{1}{1+t^2+sin^2t}$$
            \setcounter{enumii}{7}
            \item $F^{-1}$, donde $$F(x) = \int_{0}^{x}\dfrac{1}{\sqrt{1-t^2}}dt$$

            Tenemos que

            \begin{align*}
                (F^{-1})'(x) &= \dfrac{1}{F'(F^{-1}(x))}\\
                &= \dfrac{1}{\dfrac{1}{\sqrt{1-(F^{-1}(x))^2}}}\\
                &= \sqrt{1-(F^{-1}(x))^2}
            \end{align*}

        \end{enumerate}

        \item Para cada una de las $f$ siguientes, si $F(x) = \int_{0}^{x}f$, ¿En qué puntos $x$ es $F'(x) = f(x)$?

        \begin{enumerate}[label=\textcolor{Red}{\textbf{\roman*.}}]
            \setcounter{enumii}{1}
            \item $f(x) = 0$ si $x < 1$, $f(x) = 1$ si $x \geq 1$.\\
            Como la función es continua en todos los puntos excepto en $1$, por el primer teorema fundamental del cálculo tenemos que $F'(x) = f(x)$ en todo $x \not = 1$.
            \item $f(x) = 0$ si $x \not = 1$, $f(x) = 1$ si $x = 1$.\\
            Nuevamente, como la función es continua en todos los puntos excepto en $1$, por el primer teorema fundamental del cálculo tenemos que $F'(x) = f(x)$ en todo $x \not = 1$.
            \setcounter{enumii}{5}
            \item $f(x) = 0$ si $x \leq 0$, $f(x) = \frac{1}{\left[\frac{1}{x}\right]}$ si $x \geq 0$.\\
            $F'(x) = f(x)$ para todo $x$ irracional.
        \end{enumerate}

        \setcounter{enumi}{3}
        \item Demostrar que los valores de las expresiones siguientes no dependen de $x$.

        \begin{enumerate}[label=\textcolor{Red}{\textbf{\roman*.}}]
            \item $$\int_{0}^{x}\dfrac{1}{1+t^2}dt + \int_{0}^{\frac{1}{x}}\dfrac{1}{1+t^2}dt$$
            Consideremos $f(x)$ como la expresión de arriba, luego

            \begin{align*}
                f'(x) &= \dfrac{1}{1+x^2} + \dfrac{1}{1+(\frac{1}{x})^2}\left(-\dfrac{1}{x^2}\right)\\
                &= \dfrac{1}{1+x^2}-\dfrac{1}{x^2+1}\\
                &= 0
            \end{align*}

            Sabemos que si $f'(x) = 0$, entonces $f(x) = c$ para alguna constante $c$, de esta manera, $f$ no depende de $x$.

            \item $$$$
        \end{enumerate}
        \setcounter{enumi}{5}
        \item Hallar una función $g$ tal que
        \begin{enumerate}[label=\textcolor{Red}{\textbf{\roman*.}}]
            \item $$ \int_{0}^{x} tg(t)dt = x + x^2$$
            Aplicando el teorema fundamental del cálculo tenemos que
            $$g(t) = \frac{1}{t}+2$$
        \end{enumerate}

        \setcounter{enumi}{11}
        \item Hallar  $F'(x)$ si $F(x)=\int_{0 }^{x}xf(t)dt$.
        \begin{align*}
            F'(x) &= \left( \int_{0}^{x} xf(t) \, dt \right)' \\
            &= \left( 1 \cdot \int_{0}^{x} f(t) \, dt + x \cdot \left( \int_{0}^{x} f(t) \, dt \right)' \right) \\
            &= \int_{0}^{x} f(t) \, dt + x \cdot f(x).
        \end{align*}

        \item Demostrar que si $f$ es continua, entonces
        $$\int_{0}^{x}f(u)(x-u)du=\int_{0}^{x}\left( \int_{0}^{u}f(t)dt\right) du$$

        Sea $F(x) = \int_{0}^{x}f(u)(x-u)du$ y $G(x) = \int_{0}^{x}\left( \int_{0}^{u}f(t)dt\right) du$, luego derivemos ambas funciones y tenemos

        $F'(x) = \int_{0 }^{x}f(t)dt+x f(x)- xf(x)$

        Usando el ejercicio anterior tenemos que $F'(x) = \int_{0}^{x}f(t)dt$

        De manera similar

        $G'(x) = \int_{0}^{x}f(t)dt$

        Así, como sus derivadas son iguales, entonces $F(x) = G(x)$\pagebreak

        \section*{\textcolor{OliveGreen}{Ejercicios Complementarios}}

        \begin{enumerate}[label=\textcolor{OliveGreen}{\textbf{\arabic*.}}]
            \item Dibuje las regiones encerradas por cada una de las curvas dadas. Decida si integra respecto a $x 0$ y. Trace un rectángulo representativo de aproximación e indique su altura y su ancho. Luego determine el área de la región.
            \begin{enumerate}[label=\textcolor{OliveGreen}{\textbf{\roman*.}}]
                \item $$
                y=\frac{1}{x}, \quad y=x, \quad y=\frac{1}{4} x, \quad x>0
                $$

                \begin{center}
                \fbox{\adjustbox{trim=1 1 1 1, width=.5\textwidth}{\includegraphics{images/grafica6.png}}}
                \end{center}

                Los puntos de intersección son $x = 0, x=2, x=1$, así

                $$\int_{-2}^{-1}\left(-\frac{1}{x}+\frac{x}{4}\right) d x+\int_{-1}^0-\frac{3 x}{4} d x+\int_0^2 0 d x=\log (2) \approx 0.693147$$

                \item $$
                y=\frac{1}{4} x^{2}, \quad y=2 x^{2}, \quad x+y=3, \quad x \geq 0
                $$

                Los puntos de intersección son $x=0, x=2, x=1$, luego

                $$\int_{-6}^{-\frac{3}{2}}\left(3-x-\frac{x^2}{4}\right) d x+\int_{-\frac{3}{2}}^0 \frac{7 x^2}{4} d x+\int_0^1 0 d x+\int_1^2 0 d x=\frac{117}{8}$$
            \end{enumerate}
            \item Plantee una integral para el volumen del solido obtenido al hacer girar cada una de las regiones delimitadas por las curvas dadas, alrededor de la recta especificada. Después utilice su calculadora para evaluar la integral con una aproximación a cinco cifras decimales.

            \begin{enumerate}[label=\textcolor{OliveGreen}{\textbf{\roman*.}}]
                \item $y = e^{-x^2}, y=0, x=-1, x=1$

                Alrededor del eje $x$. Procederemos por el métodos de la arandela, así

                $$
                V=\pi \int_{-1}^{1}\left(e^{-x^{2}}\right)^{2} dx \approx 3.75825
                $$

                Alrededor de $y=-1$. Procederemos de la misma forma por el método de la arandela.

                $$
                V=\pi \int_{-1}^{1}\left(e^{-x^{2}}-1\right)^{2}-(1)^{2} dx \approx 6.93975
                $$

                \item $y = 0, y = cos^2x, \frac{2\pi}{2} \leq x \leq \frac{\pi}{2}$
                
                Alrededor del eje $x$. De igual forma procederemos por el método de las arandelas

                $$
                V=\pi \int_{-\pi / 2}^{\pi / 2}\left(\cos ^{2} x\right)^{2} dx \approx 3.70110
                $$

                Alrededor de $y=1$. Procederemos nuevamente por el método de las arandelas

                $$V = \pi \int_{-\pi / 2}^{\pi / 2}(1)^{2}-\left(1-\cos ^{2} x\right)^{2} dx \approx 1.9635 $$

                \item $x^2 + 4y^2 = 4$
                
                Alrededor de $y=2$. Procederemos nuevamente por el método de la arandela

                $$V = \pi \int_{-1}^{1} \left(2(1 + \sqrt{1-y^2})\right)^{2} - \left(2(1-\sqrt{1-y^2})\right)^{2} \, dy \approx 39.47841$$

                Alrededor de $x = 2$.Procederemos por el método de los cascarones de cilindro.

                $$4 \pi \sqrt{1 - \frac{(2 + x)^2}{4 x (4 + x)}} \cdot |x| \approx 121.7486
                $$

                \item $$
                y=x^{2}, \quad x^{2}+y^{2}=1, \quad y \geq 0
                $$
            \end{enumerate}

            \item
            a. Plantee una integral para el volumen del solido que se genera al hacer rotar la región que definen las curvas dadas alrededor del eje especificado.
            b. Utilice su calculadora para evaluar la integral con una aproximación de cinco decimales.
            \begin{enumerate}[label=\textcolor{OliveGreen}{\textbf{\roman*.}}]
                \item $$
                y=x e^{-x}, \quad y=0, \quad x=2
                $$

                Alrededor del eje $y$ tenemos que

                $$
                V=2 \pi \int_{0}^{2} x^{2} e^{-x} d x \approx 4.06300
                $$

                \item $y = tanx, y= 0, x=\dfrac{pi}{4}$
                Alrededor de $x = \dfrac{\pi}{2}$. Tenemos que

                $$V = 2\pi\int_{0}^{\frac{\pi}{4}\left(\frac{\pi}{2}-x\right)\tan x\,dx} \approx 2.25323$$
            \end{enumerate}
            \item Determine la longitud exacta de las siguientes curvas
            $$
            y=1+6 x^{3 / 2}, \quad 0 \leq x \leq 1, \quad y^{\prime}=9(x)^{1 / 2}
            $$
            \begin{align*}
                L &= \int_{0}^{1} \sqrt{1+\left(9(x)^{1 / 2}\right)^{2}} dx \\
                &= \int_{0}^{1} \sqrt{1+81 x} dx \\
                &= \int_{0}^{1}(1+81 x)^{1 / 2} dx\\
                &= 6.10322
            \end{align*}

            \item $$
            y^{2}=4(x+4)^{3}, \quad 0 \leq x \leq 2, \quad y>0
            $$

            \begin{align*}
                L &= \int_{0}^{2} \sqrt{1 + \left(3(x+4)^{1/2}\right)^{2}} dx \\
                &= \int_{0}^{2} \sqrt{1 + 9(x+4)} dx \\
                &\approx 13.54287
            \end{align*}
            \item $$
            y=\frac{x^{3}}{3}+\frac{1}{4 x}, \quad 1 \leq x \leq 2, \quad y^{\prime}=\frac{4 x^{4}-1}{4 x^{2}}
            $$
            \begin{align*}
                L &= \int_{1}^{2} \sqrt{1+\left(\frac{4 x^{4}-1}{4 x^{2}}\right)^{2}} d x \\
                &= \frac{59}{24} \approx 2.45833
            \end{align*}
            \item $$
            x=\frac{y^{4}}{8}+\frac{1}{4 y^{2}}, \quad 1 \leq y \leq 2
            $$

            $$L=\int_{1}^{2} \sqrt{1+\left(\frac{y^{3}}{2}-\frac{1}{2 y^{3}}\right)^{2}} dy \approx 2.0625 $$

            \item Determine el area de la superficie obtenida al hacer girar la curva en torno al eje $x$
            $$y=x^{3}, \quad 0 \leq x \leq 2 $$

            $$A=2 \pi \int_{0}^{2} x^{3} \sqrt{1+\left(3 x^{2}\right)^{2}} dx = \frac{\pi}{27}(145 \sqrt{145}-1)$$

            \item $$
            9 x=y^{2}+18, \quad 2 \leq x \leq 6
            $$
            \begin{align*}
                A &= 2 \pi \int_{0}^{6} y \sqrt{1+\left(\frac{d x}{d y}\right)^{2}} d y \\
                &= 49\pi
            \end{align*}

            \item $$y=\sqrt{1+4 x}, \quad 1 \leq x \leq 5, \quad f^{\prime}(x)=\frac{2}{\sqrt{1+4 x}}$$
            
            $$A =2 \pi \int_{1}^{5} \sqrt{1+4 x} \sqrt{1+\left(\frac{2}{\sqrt{1+4 x}}\right)^{2}} dx = \frac{98\pi}{3}$$
        \end{enumerate}
    \end{enumerate}
\end{document}
