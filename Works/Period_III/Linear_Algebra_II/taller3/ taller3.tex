\documentclass{report}
\usepackage[spanish]{babel}



\input{setup.tex}

\begin{document}
    \coverPage{ Matemáticas }{ Álgebra Lineal II }{ Taller 3 }{  }{ Alexander Mendoza }{\today}

    \section*{Taller}

    \begin{enumerate}
        \item Demuestra que
        \begin{align*}
            \|x\|_1 := \sum_{i=1}^n \left| x_i \right|
        \end{align*}
        para cada $x \in \mathbb{R}^n$.
    
        Demostración:
        \begin{enumerate}
            \item Para cada $x \in \mathbb{R}^n$ y $\alpha \in \mathbb{R}$, $\|x\| \geq 0$. La norma está definida como la suma de los valores absolutos de cada $x_i$, es decir,
            \begin{align*}
                \|x\|_1 = \left| x_1 \right| + \left| x_2 \right| + \ldots + \left| x_n \right|
            \end{align*}
            Cada $\left| x_i \right| \geqslant 0$, entonces
            \begin{align*}
                \left| x_1 \right| + \left| x_2 \right| + \ldots + \left| x_n \right| = \sum_{i=1}^n \left| x_i \right| \geq 0
            \end{align*}
            
            \item $\|x\|_1 = 0$ si y solo si $x = (0,0, \ldots, 0)$.
            \begin{enumerate}
                \item Primera parte de la implicación: Si $\|x\|_1 = 0$, entonces $\left| x_1 \right| + \left| x_2 \right| + \ldots + \left| x_n \right| = 0$. Como $\left| x_1 \right| \geq 0$, en la suma no existe un $\left| x_k \right|$ que sea el inverso aditivo de algún $\left| x_i \right|$. Por lo tanto,
                \begin{align*}
                    \sum_{j=1}^n \left| x_i \right| = \sum_{i=1}^n \left| 0_i \right| = (0,0, \ldots, 0)
                \end{align*}
                En otras palabras, $x_i$ tiene que ser igual a $|0|$.
                \item Si $x = (0,3)$, entonces $\left| x_i \right| = 0$, por lo que
                \begin{align*}
                    \left| x_1 \right| + \left| x_2 \right| + \ldots + \left| x_n \right| = 0
                \end{align*}
            \end{enumerate}
            
            \item Para $\alpha \in \mathbb{R}$, se cumple $\| \alpha x \|_1 = | \alpha | \| x \|_1$.
            \begin{align*}
                \sum_{i=1}^n \| \alpha x \| &= \left\| \alpha x_1 \right\| + \ldots + \left\| \alpha x_n \right\| \\
                &= | \alpha | \left( \| x_1 \| + \ldots + \| x_n \| \right) \\
                &= | \alpha | \sum_{j=1}^n \| x_i \| \\
                &= | \alpha | \cdot \| x \|_1
            \end{align*}
            
            \item $\| x + y \|_1 \leq \| x \|_1 + \| y \|_1$
            \begin{align*}
                & \| x + y \|_1 = \sum_{j=1}^n \mid \left( x_j + y_j \|_1 \right| = \sum_{j=1}^n \left| x_j + x_j \right| \leq \\
                & \leq \sum_{j=1}^n \left( \left| x_j \right| + \left| y_j \right| \right) = \sum_{j=1}^n \left| x_j \right| + \sum_{j=1}^n \left| y_j \right| = \| x \|_1 + \| y \|_1
            \end{align*}
        \end{enumerate}
        \item Demuestre:
            $\|x\|_{\infty}=\max \left\{\left|x_1,\right| x_2|, \ldots,| x_n \mid\right\}$
            Por definición, la norma infinito de un vector $x$ se define como:

            \[
            \|x\|_{\infty}=\max \left\{\left|x_1\right|,\left|x_2\right|, \ldots, \left|x_n\right|\right\}
            \]

            Esto significa que $\|x\|_{\infty}$ es igual al máximo valor absoluto de todas las componentes del vector $x$. Por lo tanto, necesitamos demostrar que $\|x\|_{\infty}$ es de hecho el máximo de dichos valores absolutos.

            \begin{enumerate}
                \item $\|x\|_{\infty} \geq 0$:
                
                Notamos que cada componente $\left|x_i\right|$ del vector $x$ es un valor absoluto, lo que implica que es no negativo ($\left|x_i\right| \geq 0$). Por lo tanto, el máximo de estos valores absolutos, que es $\|x\|_{\infty}$, también es no negativo. Por lo tanto, $\|x\|_{\infty} \geq 0$.
                
                \item $\|x\|_{\infty}=0$ si y solo si $x=(0, \ldots, 0)$:
                
                Si $\|x\|_{\infty}=0$, entonces el máximo de los valores absolutos de las componentes de $x$ es cero. Esto implica que cada componente $\left|x_i\right|$ de $x$ debe ser cero, ya que no puede ser negativo y el máximo es cero. Por lo tanto, $x$ debe ser el vector cuyas componentes son todas cero, es decir, $x=(0, \ldots, 0)$.
                
                Por otro lado, si $x=(0, \ldots, 0)$, entonces cada componente $\left|x_i\right|$ es cero, y por lo tanto, el máximo de estos valores absolutos es cero. Por lo tanto, $\|x\|_{\infty}=0$.
            \end{enumerate}

            Por lo tanto, hemos demostrado que $\|x\|_{\infty}=\max \left\{\left|x_1\right|,\left|x_2\right|, \ldots, \left|x_n\right|\right\}$.
        
        \item Halle $\langle x, y\rangle = \sum_{i, j=1}^3 \alpha_{ij} \alpha_i \beta_j \quad x \in \mathbb{R}^3 \quad \text{donde}$
        \begin{align*}
            A=\left[\alpha_{ij}\right] &= \left[\begin{array}{lll}
            2 & 1 & 0 \\
            1 & 2 & 0 \\
            0 & 1 & 4
            \end{array}\right]
        \end{align*}

        y $\alpha=\left(\alpha_1, \alpha_2, \alpha_3\right)$ y $\beta=\left(\beta_1, \beta_2, \beta_3\right)$, luego expandimos $\langle x, y\rangle$ utilizando las definiciones de $\alpha_{ij}$:
        \begin{align*}
            \langle x, y\rangle &= \sum_{i=1}^3 \sum_{j=1}^3 \alpha_{ij} \alpha_i \beta_j \\
            \text{Cuando } i=1, &\ \alpha_{i1}=2, \alpha_{i2}=1, \alpha_{i3}=0 \\
            \text{Cuando } i=2, &\ \alpha_{i1}=1, \alpha_{i2}=2, \alpha_{i3}=0 \\
            \text{Cuando } i=3, &\ \alpha_{i1}=0, \alpha_{i2}=1, \alpha_{i3}=4
        \end{align*}

        Sustituimos estos valores en la expresión de $\langle x, y\rangle$:
        \begin{align*}
            \langle x, y\rangle &= 2 \alpha_1 \beta_1 + 1 \alpha_1 \beta_2 + 0 \alpha_1 \beta_3 \\
            &\quad + 1 \alpha_2 \beta_1 + 2 \alpha_2 \beta_2 + 0 \alpha_2 \beta_3 \\
            &\quad + 0 \alpha_3 \beta_1 + 1 \alpha_3 \beta_2 + 4 \alpha_3 \beta_3 \\
            &= \alpha_1(2 \beta_1 + \beta_2) + \alpha_2(\beta_1 + 2 \beta_2) + \alpha_3(\beta_2 + 4 \beta_3)
        \end{align*}
            \end{enumerate}
    
\end{document}
