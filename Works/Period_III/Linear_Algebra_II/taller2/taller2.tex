\documentclass{report}
\usepackage[spanish]{babel}



\input{setup.tex}

\begin{document}
    \coverPage{ Matemáticas }{ Álgebra Lineal II }{ Taller 2 }{  }{ Alexander Mendoza }{\today}

    \section*{Taller 2}
    
    \begin{enumerate}
        \item Si $A \in M_{nxn}(F)$ es invertible y $deg m_A(x) = P$. Demostrar que $A^{-1}$ es una combinación de $I_n, A, A^2, \dots A^{P-1}$.

        Sea $\mathrm{A}$ una matriz invertible de $n \times n$. Dado que $A$ es invertible, $\det(A) \neq 0$, lo que implica que el polinomio característico $P_{A}(\lambda) \neq 0$, es decir, $\lambda \neq 0$. Por el teorema de Cayley-Hamilton, $P(A)=0$. Si escribimos $P(A)$ de forma general como $b_{0} I + b_{1} A + \ldots + b_{n} A^{n} = 0$, entonces $A^{-1}$ se puede expresar como:

        $$
        A^{-1} = \left(\frac{-b_{1}}{b_{0}} \cdot I - \frac{b_{2}}{b_{0}} \cdot A - \ldots - \frac{b_{n}}{b_{0}} A^{n-1}\right)
        $$

        \item Determine el polinomio característico y el polinomio mínimo de cada una de las siguientes matrices.

        \begin{enumerate}
            \item $$\begin{bmatrix}
                1 & 2 & 3\\
                0 & 1 & 2\\
                0 & 0 & 1
            \end{bmatrix}$$

            $$
            A = \begin{bmatrix}
            1 & 2 & 3 \\
            0 & 1 & 2 \\
            0 & 0 & 1
            \end{bmatrix}
            $$

            $P_{A}(\lambda) = -(1-\lambda)(1-\lambda)(1-\lambda)$

            $P_{A} = -(A-I)^{3} = 0$

            \begin{enumerate}
                \item $(I-A)^{2} = \begin{bmatrix} 0 & -2 & -3 \\ 0 & 0 & -2 \\ 0 & 0 & 0 \end{bmatrix}$
                \item $(I-A)^{2} = \begin{bmatrix} 0 & 2 & 3 \\ 0 & 0 & 2 \\ 0 & 0 & 0 \end{bmatrix} \cdot \begin{bmatrix} 0 & 2 & 3 \\ 0 & 0 & 2 \\ 0 & 0 & 0 \end{bmatrix} = \begin{bmatrix} 0 & 0 & 4 \\ 0 & 0 & 0 \\ 0 & 0 & 0 \end{bmatrix}$
            \end{enumerate}

            Por el teorema, $(I-A)^{3} = 0$, por lo tanto, el polinomio característico es el mismo que el minimal.

            \item Sea la matriz $$\begin{bmatrix}
                1 & 1 & 0\\
                -1 & 1 & 1\\
                0 & 1 & -1
            \end{bmatrix}$$

            \[
                A = \begin{bmatrix}
                1 & 1 & 0 \\
                -1 & 1 & 1 \\
                0 & 1 & -1
                \end{bmatrix}, \quad
                D = \begin{bmatrix}
                -\lambda & 0 & 0 \\
                0 & -\lambda & 0 \\
                0 & 0 & -\lambda
                \end{bmatrix}
                \]
                
                La suma $A + D$ es:
                
                \[
                A + D = \begin{bmatrix}
                1-\lambda & 1 & 0 \\
                -1 & 1-\lambda & 1 \\
                0 & 1 & -1-\lambda
                \end{bmatrix}
                \]
                
                El polinomio característico se calcula como:
                
                \[
                \begin{aligned}
                \det(A + D) &= (1-\lambda)((\lambda^{2}-2)-1-\lambda) \\
                &= -\lambda^{3}+\lambda^{2}+\lambda-3 \\
                &= (\lambda+1,3593)(\lambda-1,1797+0,9030 i)(\lambda-1,1797-0,90301 i)
                \end{aligned}
                \]
                
                donde $\alpha = 1+1,3593$, $\beta = 1-1,1797+0,9030 i$, y $\gamma = 1-1,1797-0,90301 i$.
                
                Para la posibilidad 1:
                
                \[
                (A+1,3593 I)(A-1,1797 I+0,90301 i I) = \begin{bmatrix}
                2,36 & 0 & 0 \\
                -1 & 2,36 & 0 \\
                0 & 0 & -0,64
                \end{bmatrix}
                \]
                
                \[
                \begin{aligned}
                & \begin{bmatrix}
                1 & 1 & 0 \\
                -1 & 1 & 1 \\
                0 & 1 & -1
                \end{bmatrix} + \begin{bmatrix}
                -0,1797 & 1 & 0 \\
                -1 & -0,8571 & 1 \\
                0 & 1 & -2,1797
                \end{bmatrix} \\
                & = \text{Resultado de la suma}
                \end{aligned}
                \]

            \item Sea la matriz $$\begin{bmatrix}
                0 & 0 & 2\\
                1 & 0 & -1\\
                0 & 1 & 1
            \end{bmatrix}$$

            \item Sea la matriz y sean $a, b \in \mathbb{R}$
            $$\begin{bmatrix}
                b & a & 0\\
                0 & b & a\\
                0 & 0 & b
            \end{bmatrix}$$

            \[
                A = \begin{bmatrix}
                b & a & 0 \\
                0 & b & a \\
                0 & 0 & b
                \end{bmatrix}
                \]
                
                Dado que $A$ es triangular superior, su polinomio característico es $(b-\lambda)^3$.
                
                Para encontrar el polinomio minimal, consideramos las potencias de $(b-\lambda)$:
                
                1) $(b-\lambda)$:
                \[
                (bI - A) = \begin{bmatrix}
                b-\lambda & -a & 0 \\
                0 & b-\lambda & -a \\
                0 & 0 & b-\lambda
                \end{bmatrix}
                \]
                Si $a=0$, el polinomio minimal es $(b-\lambda)$. Si $a\neq0$, continuamos.
                
                2) $(b-\lambda)^2$:
                \[
                (bI - A)^2 = \begin{bmatrix}
                0 & -a & 0 \\
                0 & 0 & -a \\
                0 & 0 & 0
                \end{bmatrix} \begin{bmatrix}
                0 & -a & 0 \\
                0 & 0 & -a \\
                0 & 0 & 0
                \end{bmatrix} = \begin{bmatrix}
                0 & 0 & a^2 \\
                0 & 0 & 0 \\
                0 & 0 & 0
                \end{bmatrix}
                \]
                
                3) $(b-\lambda)^3 = (b-\lambda)^2 \cdot (bI - A)$:
                \[
                (bI - A)^3 = \begin{bmatrix}
                0 & 0 & a^2 \\
                0 & 0 & 0 \\
                0 & 0 & 0
                \end{bmatrix} \begin{bmatrix}
                b-\lambda & -a & 0 \\
                0 & b-\lambda & -a \\
                0 & 0 & b-\lambda
                \end{bmatrix} = \begin{bmatrix}
                0 & 0 & 0 \\
                0 & 0 & 0 \\
                0 & 0 & 0
                \end{bmatrix}
                \]

            Con esto, el polinomio minimal de $A$ es el mismo que su polinomio característico.

            \item Sea la matriz  $$\begin{bmatrix}
                2 & 8 & 0 & 0 & 0 & 0 & 0\\
                0 & 2 & 0 & 0 & 0 & 0 & 0\\
                0 & 0 & 4 & 2 & 0 & 0 & 0\\
                0 & 0 & 1 & 3 & 0 & 0 & 0\\
                0 & 0 & 0 & 0 & 0 & 3 & 0\\
                0 & 0 & 0 & 0 & 0 & 0 & 0\\
                0 & 0 & 0 & 0 & 0 & 0 & 5\\
            \end{bmatrix}$$

            Calcular el polinomio característico de $A$ es equivalente a calcular el de las submatrices que lo conforman.

            \[
            \begin{aligned}
            &\begin{bmatrix}
            2-\lambda & 8 \\
            0 & 2-\lambda
            \end{bmatrix} = (2-\lambda)^{2}-8(0) = (2-\lambda)^{2} \\
            &\begin{bmatrix}
            4-\lambda & 2 \\
            1 & 3-\lambda
            \end{bmatrix} = (4-\lambda)(3-\lambda)-(1)(2) = (\lambda-5)(\lambda-2) \\
            &\begin{bmatrix}
            0-\lambda & 3 & 0 \\
            0 & 0-\lambda & 0 \\
            0 & 0 & 5-\lambda
            \end{bmatrix} = -\lambda
            \begin{bmatrix}
            -\lambda & 0 \\
            0 & 5-\lambda
            \end{bmatrix} - 3
            \begin{bmatrix}
            0 & 0 \\
            0 & 5-\lambda
            \end{bmatrix}
            \end{aligned}
            \]
            
            Realizando los cálculos, obtenemos que:
            
            \[
            P_{A}(\lambda) = (2-\lambda)^{2}(\lambda-5)^{2}(\lambda-2)(-\lambda)^{2}
            \]
            
            Posibilidades:
            
            \[
            \begin{bmatrix}
            4 & 32 & 0 & 0 & 0 & 0 & 0 \\
            0 & 4 & 0 & 0 & 0 & 0 & 0 \\
            0 & 0 & 18 & 14 & 0 & 0 & 0 \\
            0 & 0 & 7 & 11 & 0 & 0 & 0 \\
            0 & 0 & 0 & 0 & 0 & 0 & 0 \\
            0 & 0 & 0 & 0 & 0 & 0 & 0 \\
            0 & 0 & 0 & 0 & 0 & 0 & 25
            \end{bmatrix}
            \]
            
            Minimal: $(-\lambda)^{2}(\lambda-5)^{2}(\lambda-2)$
            
            \[
            \begin{aligned}
            &\text{Minimal: }(-\lambda)^{2}(\lambda-5)^{2}(\lambda-2) \\
            &\begin{bmatrix}
            4 & 32 & 0 & 0 & 0 & 0 & 0 \\
            0 & 4 & 0 & 0 & 0 & 0 & 0 \\
            0 & 0 & 18 & 14 & 0 & 0 & 0 \\
            0 & 0 & 7 & 11 & 0 & 0 & 0 \\
            0 & 0 & 0 & 0 & 0 & 0 & 0 \\
            0 & 0 & 0 & 0 & 0 & 0 & 0 \\
            0 & 0 & 0 & 0 & 0 & 0 & 25
            \end{bmatrix}
            \times
            \begin{bmatrix}
            0 & 8 & 0 & 0 & 0 & 0 & 0 \\
            0 & 0 & 0 & 0 & 0 & 0 & 0 \\
            0 & 0 & 2 & 2 & 0 & 0 & 0 \\
            0 & 0 & 1 & 1 & 0 & 0 & 0 \\
            0 & 0 & 0 & 0 & -2 & 3 & 0 \\
            0 & 0 & 0 & 0 & 0 & -2 & 0 \\
            0 & 0 & 0 & 0 & 0 & 0 & 3
            \end{bmatrix} \\
            &=
            \begin{bmatrix}
            0 & 32 & 0 & 0 & 0 & 0 & 0 \\
            0 & 0 & 0 & 0 & 0 & 0 & 0 \\
            0 & 0 & 50 & 50 & 0 & 0 & 0 \\
            0 & 0 & 25 & 25 & 0 & 0 & 0 \\
            0 & 0 & 0 & 0 & 0 & 0 & 0 \\
            0 & 0 & 0 & 0 & 0 & 0 & 0 \\
            0 & 0 & 0 & 0 & 0 & 0 & 75
            \end{bmatrix} \\
            &= (\lambda^{2})(\lambda-2)
            \end{aligned}
            \]
            
            $(\lambda-5)$
            
            \[
            \begin{bmatrix}
            -3 & 8 & 0 & 0 & 0 & 0 & 0 \\
            0 & -3 & 0 & 0 & 0 & 0 & 0 \\
            0 & 0 & -1 & 2 & 0 & 0 & 0 \\
            0 & 0 & 1 & -2 & 0 & 0 & 0 \\
            0 & 0 & 0 & 0 & 0 & 3 & 0 \\
            0 & 0 & 0 & 0 & 0 & 0 & 0 \\
            0 & 0 & 0 & 0 & 0 & 0 & 0
            \end{bmatrix}
            = (\lambda-5) \Rightarrow
            \begin{bmatrix}
            9 & -48 & 0 & 0 & 0 & 0 & 0 \\
            0 & 9 & 0 & 0 & 0 & 0 & 0 \\
            0 & 0 & 3 & -6 & 0 & 0 & 0 \\
            0 & 0 & -3 & 6 & 0 & 0 & 0 \\
            0 & 0 & 0 & 0 & 0 & 0 & 0 \\
            0 & 0 & 0 & 0 & 0 & 0 & 0 \\
            0 & 0 & 0 & 0 & 0 & 0 & 0
            \end{bmatrix}
            = (\lambda-5)^{2}
            \]
            
            Por lo tanto, el polinomio minimal es $(-\lambda)^{2}(\lambda-5)^{2}(\lambda-2)$.
        \end{enumerate}

        \item Demuestre que la transpuesta $A^t$ de una matriz $A$ tienen el mismo polinomio minimal.
        
        Para demostrar que la matriz transpuesta $A^T$ de una matriz $A$ tiene el mismo polinomio minimal, observamos que

        \[
        \begin{aligned}
        \lambda_{A}(\lambda) & = \operatorname{det}(\lambda \operatorname{In}-A) \\
        & = \operatorname{det}(\lambda(\operatorname{In}^T)-A) \\
        & = \operatorname{det}(\lambda \operatorname{In}-A^{T}) = \lambda^{t}(A)(\lambda)
        \end{aligned}
        \]

        Esto muestra que el polinomio minimal de $A$ es igual al de $A^T$. Por lo tanto, $A$ y $A^T$ tienen el mismo polinomio minimal.

        \item Demuestre que el término constante en el polinomio característico de $A$ es $\det(A)$
        
        Según la definición, tenemos:
        \[
        P(\lambda) = \det\left[\begin{array}{cccc}
        a_{11}-\lambda & a_{12} & \cdots & a_{1n} \\
        a_{21} & a_{22}-\lambda & \cdots & a_{2n} \\
        \vdots & \vdots & \ddots & \vdots \\
        a_{n1} & a_{n2} & \cdots & a_{nn}-\lambda
        \end{array}\right] = (-1)^n(\lambda^n + c_1\lambda^{n-1} + c_2\lambda^{n-2} + \cdots + c_n)
        \]

        Sustituimos \( \lambda = 0 \):
        \[
        P(0) = \det\left[\begin{array}{cccc}
        a_{11} & a_{12} & \cdots & a_{1n} \\
        a_{21} & a_{22} & \cdots & a_{2n} \\
        \vdots & \vdots & \ddots & \vdots \\
        a_{n1} & a_{n2} & \cdots & a_{nn}
        \end{array}\right] = (-1)^n(0^n + c_1\cdot 0^{n-1} + c_2\cdot 0^{n-2} + \cdots + c_n)
        \]

        Simplificando, obtenemos:
        \[
        \det(A) = c_n
        \]
    \end{enumerate}
\end{document}
