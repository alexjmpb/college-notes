\documentclass{report}
\usepackage[spanish]{babel}
\newtheorem{theorem}{Teorema}


\input{setup.tex}

\begin{document}
    \coverPage{ Matemáticas }{ Teoría de Números }{ Corrección Parcial }{  }{ Alexander Mendoza }{\today}

    \section*{Parcial teoría de números}

    \begin{enumerate}
        \item Demuestre que todo número de la forma $n^3 + 2n$ con $n \in \mathbb{N}$ es divisible por $3$.
        Procederemos por inducción. Consideremos primero cuando $n = 1$
        $$3 \mid 1^3 + 2\cdot 1 = 3$$
        Con esto, se tiene que la propiedad se cumple para el caso base $n = 1$. Supongamos ahora que $3 \mid n^3 + 2n$ para algún $n$. Esta será nuestra hipótesis de inducción. Luego, por hipótesis de inducción, $3 \mid n^3$ y $3 \mid 2n$. Como $\gcd(3, 2) = 1$, entonces tenemos que $3 \mid n$, luego $3 \mid n^3$, $3 \mid 3n^2$, $3 \mid 5n$, $3 \mid 3$ por lo tanto $3 \mid n^3+3n^2+5n+3$ lo que implica que $3 \mid (n+1)^3 + 2(n+1)$.

        \item Sean $a,b \in \mathbb{Z}^+$, demostrar que $lcm(a, b)\gcd(a,b) = ab$.

        Consideremos los siguientes casos:

        \begin{enumerate}
            \item Cuando $a,b$ son ambos primos:
            Como $a, b$ son primos, tenemos que $\gcd(a,b) = 1$. Además sabemos que $lcm(a,b) = ab$. Con esto podemos concluir que $lcm(a, b)gcd(a,b) = ab \cdot 1 = ab$.
            \item Cuando $a$ es primo y $b$ no lo es:
            Sabemos que $b$ no es primo, por lo tanto $b$ es producto de primos, así $p_1^{m_1}p_2^{m_2}\cdot \cdots \cdot p_n^{m_n}$ si $a = P_k^{m_k}$ para algún $k$, concluimos que $\gcd(a,b) = a$ y $lcm(a,b) = b$, por lo tanto $lcm(a, b)gcd(a,b) = ab$. De manera análoga se puede demostrar cuando $a$ no es primo y $b$ es primo
            \item Cuando $a, b$ son no primos:
            Entonces, tenemos:
            \begin{align*}
            & \gcd(a, b) = p_1^{\min(m_1, n_1)} \cdot p_2^{\min(m_2, n_2)} \cdot \ldots \cdot p_k^{\min(m_k, n_k)} \\
            & \operatorname{lcm}(a, b) = p_1^{\max(m_1, n_1)} \cdot p_2^{\max(m_2, n_2)} \cdot \ldots \cdot p_k^{\max(m_k, n_k)}
            \end{align*}

            Por lo tanto,
            \begin{align*}
            & \gcd(a, b) \cdot \operatorname{lcm}(a, b) \\
            & = \left( p_1^{\min(m_1, n_1)} \cdot p_2^{\min(m_2, n_2)} \cdot \ldots \cdot p_k^{\min(m_k, n_k)} \right) \cdot \\
            & \left( p_1^{\max(m_1, n_1)} \cdot p_2^{\max(m_2, n_2)} \cdot \ldots \cdot p_k^{\max(m_k, n_k)} \right) \\
            & = \left( p_1^{m_1} \cdot p_1^{n_1} \right) \cdot \left( p_2^{m_2} \cdot p_2^{n_2} \right) \cdot \ldots \cdot \left( p_k^{m_k} \cdot p_k^{n_k} \right) \\
            & = \left( p_1^{m_1 + n_1} \right) \cdot \left( p_2^{m_2 + n_2} \right) \cdot \ldots \cdot \left( p_k^{m_k + n_k} \right) \\
            & = a \cdot b
            \end{align*}
        \end{enumerate}

        \item Sea $p$ primo. Demostrar que $\phi(p^r) = p^r-p^{r-1}$. Emplee este resultado para probar que si $2 \mid n$ entonces $\phi(2n) = 2\phi(n)$.

        Sea $N = \left\{ n \mid 1 \leq n \leq p^a \right\}$, si logramos construir un conjunto $P$ tal que $P$ contenga todos los enteros positivos menores que $p^a$ que no son primos relativos de $p^a$, entonces $\left|N\right| - \left|P\right| = \phi(p^r)$. Note que todo número de la forma $xp$ no es primo relativo de $p^a$, ya que $\gcd(xp, p^a)$ es como mínimo $p$. De esta manera tenemos que $1 p, 2 p, ..., p^{a-1}p$ no son primos relativos de $p^a$, así tenemos que $P = \left\{ mp \mid 1 \leq m \leq p^{a-1} \right\}$ y con esto, $\left|P\right| = p^{a-1}$, además sabemos que $\left|N\right| = p^a$, por lo tanto $\phi(p^r) = p^a - p^{a-1}$.

        \begin{theorem}
            Si $\gcd(m,n) = 1$, entonces $\phi(mn) = \phi(m)\phi(n)$
        \end{theorem}

        Para demostrar que si $2 \mid n$ entonces $\phi(2n) = 2\phi(n)$, procederemos de la siguiente manera. Sabemos que como $2 \mid n$, $n = 2^rm$ donde $\gcd(2, m) = 1$, así
        \begin{align*}
            \phi(2n) &= \phi(2^rm)\\
            &= \phi(2^{r+1})\phi(m) && \text{ Por T.1 y $\gcd(2^r,m) = 1$ }\\
            &= (2^{r+1}-2^r)\phi(m) && \text{ Por demostración anterior}\\
            &= (2(2^r-2^{r-1}))\phi(m) && \text{ Propiedad distributiva}\\
            &= 2\phi(2^r)\phi(m) && \text{ Por demostración anterior}\\
            &= 2\phi(2^rm) && \text{ Por T.1}\\
            &= 2\phi(n)
        \end{align*}

        \item Hallar el menor número positivo $n$ con 42 divisores.

        Sabemos que la descomposición en primos de 42 es $\left\{ 2,3,7 \right\}$, con esto, por el teorema de la función de divisores, tenemos que $n = p_1^1p_2^2p_3^6$ tiene 42 divisores, para $p_1, p_2, p_3$. Si tomamos los menores primos y los ordenamos de manera que $n = 5^1*3^2*2^6 = 2880$, con esto $2880$ tiene 42 divisores y es el menor número con dicha propiedad.

        \item Si $p$ y $q$ son primos tales que $p, q \geq 5$ demostrar que $24 \mid p^2 - q^2$.\\
        Dado que $p$ y $q$ son primos mayores o iguales que $5$, ambos son impares. Podemos representarlos como $p = 2n + 1$ y $q = 2m + 1$, donde $n$ y $m$ son enteros no negativos.

        La diferencia de cuadrados $p^2 - q^2$ es:

        \begin{align*}
            p^2 - q^2 &= (2n + 1)^2 - (2m + 1)^2 \\
            &= 4(n^2 - m^2) \\
            &= 4(n - m)(n + m)
        \end{align*}

        Como $p$ y $q$ son primos distintos, al menos uno de ellos es congruente a $\pm 1$ módulo $3$. Sin pérdida de generalidad, supongamos que $p$ es congruente a $1$ o $-1$ módulo $3$. Entonces, $p^2$ es congruente a $1$ módulo $3$, al igual que $q^2$.

        Por lo tanto, $p^2 - q^2$ es la diferencia de dos números congruentes a $1$ módulo $3$, lo que implica que $p^2 - q^2$ es divisible por $3$.

        Además, como $p$ y $q$ son impares, $p^2$ y $q^2$ son congruentes a $1$ módulo $8$, lo que significa que $p^2 - q^2$ es divisible por $8$.

        Por lo tanto, $p^2 - q^2$ es divisible por tanto $3$ como $8$, y por ende, por $24$. Por lo tanto, si $p$ y $q$ son primos mayores o iguales que $5$, entonces $24$ divide a $p^2 - q^2$.
    \end{enumerate}
\end{document}
